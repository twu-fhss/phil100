% Options for packages loaded elsewhere
\PassOptionsToPackage{unicode}{hyperref}
\PassOptionsToPackage{hyphens}{url}
%
\documentclass[
]{book}
\usepackage{amsmath,amssymb}
\usepackage{iftex}
\ifPDFTeX
  \usepackage[T1]{fontenc}
  \usepackage[utf8]{inputenc}
  \usepackage{textcomp} % provide euro and other symbols
\else % if luatex or xetex
  \usepackage{unicode-math} % this also loads fontspec
  \defaultfontfeatures{Scale=MatchLowercase}
  \defaultfontfeatures[\rmfamily]{Ligatures=TeX,Scale=1}
\fi
\usepackage{lmodern}
\ifPDFTeX\else
  % xetex/luatex font selection
\fi
% Use upquote if available, for straight quotes in verbatim environments
\IfFileExists{upquote.sty}{\usepackage{upquote}}{}
\IfFileExists{microtype.sty}{% use microtype if available
  \usepackage[]{microtype}
  \UseMicrotypeSet[protrusion]{basicmath} % disable protrusion for tt fonts
}{}
\makeatletter
\@ifundefined{KOMAClassName}{% if non-KOMA class
  \IfFileExists{parskip.sty}{%
    \usepackage{parskip}
  }{% else
    \setlength{\parindent}{0pt}
    \setlength{\parskip}{6pt plus 2pt minus 1pt}}
}{% if KOMA class
  \KOMAoptions{parskip=half}}
\makeatother
\usepackage{xcolor}
\usepackage{longtable,booktabs,array}
\usepackage{calc} % for calculating minipage widths
% Correct order of tables after \paragraph or \subparagraph
\usepackage{etoolbox}
\makeatletter
\patchcmd\longtable{\par}{\if@noskipsec\mbox{}\fi\par}{}{}
\makeatother
% Allow footnotes in longtable head/foot
\IfFileExists{footnotehyper.sty}{\usepackage{footnotehyper}}{\usepackage{footnote}}
\makesavenoteenv{longtable}
\usepackage{graphicx}
\makeatletter
\def\maxwidth{\ifdim\Gin@nat@width>\linewidth\linewidth\else\Gin@nat@width\fi}
\def\maxheight{\ifdim\Gin@nat@height>\textheight\textheight\else\Gin@nat@height\fi}
\makeatother
% Scale images if necessary, so that they will not overflow the page
% margins by default, and it is still possible to overwrite the defaults
% using explicit options in \includegraphics[width, height, ...]{}
\setkeys{Gin}{width=\maxwidth,height=\maxheight,keepaspectratio}
% Set default figure placement to htbp
\makeatletter
\def\fps@figure{htbp}
\makeatother
\setlength{\emergencystretch}{3em} % prevent overfull lines
\providecommand{\tightlist}{%
  \setlength{\itemsep}{0pt}\setlength{\parskip}{0pt}}
\setcounter{secnumdepth}{5}
\usepackage{booktabs}
\usepackage{amsthm}
\makeatletter
\def\thm@space@setup{%
  \thm@preskip=8pt plus 2pt minus 4pt
  \thm@postskip=\thm@preskip
}
\makeatother

\usepackage{tcolorbox}


\newtcolorbox{blackbox}{
  colback=black,
  coltext=white,
  colframe=black,
  boxsep=5pt,
  arc=4pt}
\newtcolorbox{bonus}{
  colback=blue!15,
  colframe=blue!15,
  coltext=black!80,
  boxsep=5pt,
  arc=4pt}
\newtcolorbox{reflect}{
  colback=green!5,
  colframe=green!5,
  coltext=black!80,
  boxsep=5pt,
  arc=4pt}
\newtcolorbox{assessment}{
  colback=blue!5,
  colframe=blue!5,
  coltext=black!80,
  boxsep=5pt,
  arc=4pt}
\newtcolorbox{progress}{
  colback=purple!10,
  colframe=purple!10,
  coltext=black!80,
  boxsep=5pt,
  arc=4pt}
\newtcolorbox{video}{
  colback=yellow!5,
  colframe=yellow!5,
  coltext=black!80,
  boxsep=5pt,
  arc=4pt}
\newtcolorbox{caution}{
  colback=red!5,
  colframe=red!5,
  coltext=black!80,
  boxsep=5pt,
  arc=4pt}
\newtcolorbox{feedback}{
  colback=black!5,
  colframe=black!5,
  coltext=black!80,
  boxsep=5pt,
  arc=4pt}
\ifLuaTeX
  \usepackage{selnolig}  % disable illegal ligatures
\fi
\usepackage[]{natbib}
\bibliographystyle{apalike}
\IfFileExists{bookmark.sty}{\usepackage{bookmark}}{\usepackage{hyperref}}
\IfFileExists{xurl.sty}{\usepackage{xurl}}{} % add URL line breaks if available
\urlstyle{same}
\hypersetup{
  pdftitle={Philosophy for Life},
  hidelinks,
  pdfcreator={LaTeX via pandoc}}

\title{Philosophy for Life}
\author{}
\date{\vspace{-2.5em}}

\begin{document}
\maketitle

{
\setcounter{tocdepth}{1}
\tableofcontents
}
\hypertarget{welcome}{%
\chapter*{Welcome}\label{welcome}}
\addcontentsline{toc}{chapter}{Welcome}

This is the course book for Philosophy for Life. This book is divided into thematic units of study to help you engage with the materials. The course resources and learning activities are designed not only to help prepare you for the course assessments, but also to give you opportunities to practice various skills.

\begin{quote}
Below you will find information about how to navigate this book. Please read the full course syllabus located on the Course Home page in Moodle. It includes key information about the course schedule, assignments, and policies.
\end{quote}

\hypertarget{course-notes}{%
\section*{Course Notes}\label{course-notes}}
\addcontentsline{toc}{section}{Course Notes}

You should be reading this information in the context of a Trinity Western University course offered via Moodle. If this is not the case, then this may be an unauthorized reproduction of the course. Please contact \href{mailto:elearning@twu.ca}{\nolinkurl{elearning@twu.ca}} if you have concerns.

These notes will be your guide through the learning activities and assessment strategies necessary for you to succeed in the course, so it is important for you to engage to the best of your ability and take advantage of the resources available to you through Trinity Western University.

\begin{quote}
Assessment tasks are managed in other sections of the Moodle course, so be sure to familiarize yourself with those requirements and resources.
\end{quote}

\hypertarget{how-to-navigate-this-book}{%
\section*{How To Navigate This Book}\label{how-to-navigate-this-book}}
\addcontentsline{toc}{section}{How To Navigate This Book}

To move quickly to different portions of the book, click on the appropriate chapter or section in the table of contents on the left. The buttons at the top of the page allow you to show/hide the table of contents, search the book, change font settings, download a pdf or ebook copy of this book, or get hints on various sections of the book.

\includegraphics{assets/course-intro/menu.png}

The faint left and right arrows at the sides of each page (or bottom of the page if it's narrow enough) allow you to step to the next/previous section. Here's what they look like:

\includegraphics{assets/course-intro/left_arrow.png} \includegraphics{assets/course-intro/right_arrow.png}

You can also download an offline copy of this book in various formats, such as pdf or an ebook. If you are having any accessibility or navigation issues with this book, please reach out to your instructor or our online team at \href{mailto:elearning@twu.ca}{\nolinkurl{elearning@twu.ca}}.

\hypertarget{course-units}{%
\subsection*{Course Units}\label{course-units}}
\addcontentsline{toc}{subsection}{Course Units}

This course is organized into thematic units. Each unit of the course will provide you with the following information:

\begin{itemize}
\tightlist
\item
  A general overview of the key concepts that will be addressed during the unit.\\
\item
  Specific learning outcomes and topics for the unit.\\
\item
  Learning activities to help you engage with the concepts. These often include key readings, videos, and reflective prompts.\\
\item
  The Assessment section provides details on assignments you will need to complete throughout the course to demonstrate your understanding of the course learning outcomes.
\end{itemize}

\begin{quote}
Note that assessments, including assignments and discussion posts will be submitted in Moodle. See the Assessment section(s) in Moodle for full assignment details.
\end{quote}

\hypertarget{course-activities}{%
\subsection*{Course Activities}\label{course-activities}}
\addcontentsline{toc}{subsection}{Course Activities}

Below is some key information on features you will see throughout the course.

\begin{reflect}
\textbf{\emph{Learning Activity}}

This box will prompt you to engage in course concepts, often by viewing resources and reflecting on your experience and/or learning. Most learning activities are ungraded and are designed to help prepare you for the assessment in this course.
\end{reflect}

\begin{assessment}
\textbf{\emph{Assessment}}

This box will signify an assignment or discussion post you will submit in Moodle. Note that these demonstrate your understanding of the course learning outcomes. Be sure to review the grading rubrics for each assignment.
\end{assessment}

\begin{progress}
\textbf{\emph{Checking Your Learning}}

This box is for checking your understanding, to make sure you are ready for what follows. Ways to check your learning might include self-check quizzes or questions for discussion. These activities are not graded but are critical for you to be able to begin to develop evaluative judgement in this domain of knowledge.
\end{progress}

\begin{caution}
\textbf{\emph{Note}}

This box signifies key notes. It may also warn you of possible problems or pitfalls you may encounter!
\end{caution}

If you have any questions, do not hesitate to ask. We are here to help and be your guide on this journey.

\hypertarget{wisdom}{%
\chapter{Wisdom}\label{wisdom}}

In this video, we provide a concise overview of the initial unit's four key themes: knowledge, humility, friendship, and rhetoric. While the first topic examines some fundamental concepts regarding the essence of wisdom, the majority of this unit is dedicated to unveiling practical skills essential for leading a wise and fulfilling life. We explore how to cultivate and apply these skills, emphasizing a proactive approach to embodying the principles of wisdom in our daily experiences.

\textbf{Wisdom Unit 1 Introduction} Video (6 min 10 sec)

\hypertarget{overview}{%
\section*{Overview}\label{overview}}
\addcontentsline{toc}{section}{Overview}

\hypertarget{topics}{%
\subsection*{Topics}\label{topics}}
\addcontentsline{toc}{subsection}{Topics}

This unit is divided into the following four topics:

\begin{enumerate}
\def\labelenumi{\arabic{enumi}.}
\tightlist
\item
  Wisdom and Knowledge
\item
  Wisdom and Humility
\item
  Wisdom and Friendship
\item
  Wisdom and Rhetoric
\end{enumerate}

\hypertarget{learning-outcomes}{%
\subsection*{Learning Outcomes}\label{learning-outcomes}}
\addcontentsline{toc}{subsection}{Learning Outcomes}

When you've completed this unit, you will have learned how to:

\begin{itemize}
\tightlist
\item
  Identify and understand the role wisdom plays in knowledge, humiloity, friendship and rhetoric
\item
  Practice and apply the skills of wisdom to knowledge, humility, friendship and rhetoric
\item
  Apply these skills to the types of actions a wise person may take
\item
  Enhance your skills of wisdom and reason to live life within the context of justice and faith
\item
  Develop the skills to become a well-rounded person and citizen and live a wise and just life
\end{itemize}

\hypertarget{wisdom-and-knowledge}{%
\section{Wisdom and Knowledge}\label{wisdom-and-knowledge}}

Follow and complete the steps below to accomplish your learning for this topic.

\begin{enumerate}
\def\labelenumi{\arabic{enumi}.}
\tightlist
\item
  Read Topic Notes
\item
  Watch Topic Video
\item
  Read Topic Reading
\item
  Complete Topic Exercise
\item
  Complete Topic Questionnaire
\item
  Watch Optional Video
\end{enumerate}

\hypertarget{topic-notes}{%
\subsection*{Topic Notes}\label{topic-notes}}
\addcontentsline{toc}{subsection}{Topic Notes}

In this topic, we delve into the essence of wisdom and its intricate relationship with knowledge. Wisdom transcends mere information; it is a practical virtue essential for navigating life's complexities, confronting challenges, and ideally, circumventing adversity. While wisdom is rooted in practicality, it necessitates a foundation of knowledge. To lead a well-lived life, wisdom calls for an understanding of the right beliefs and the skill to apply them judiciously and timely.

Philosopher Robert Nozick contends that a wise individual possesses beliefs spanning various facets of life, encompassing crucial values, strategies for goal achievement, and the requisite steps to realize these objectives. Furthermore, wise actions, integral to a fulfilling life, stem from the application of these foundational beliefs. Mere possession of knowledge does not equate to wisdom; the discerning application of beliefs in one's behavior is paramount.

Nozick acknowledges that, despite meticulous planning and application of appropriate beliefs and actions, unforeseen challenges may arise, disrupting our endeavors. A wise individual anticipates this inherent uncertainty, preparing for potential setbacks, acquiring the skills to avert failure, and developing resilience to respond effectively when faced with adversity. In essence, wisdom involves not only knowing what is right but also navigating the unpredictable nature of life with foresight, adaptability, and a resilient spirit.

\hypertarget{watch-and-reflect}{%
\subsection*{Watch and Reflect}\label{watch-and-reflect}}
\addcontentsline{toc}{subsection}{Watch and Reflect}

In this video, you will learn about wisdom and knowledge from Robert Nozick's piece, ``\emph{What is Wisdom and Why do Philosophers Love it So?}''

\textbf{Wisdom and Knowledge Unit 1 Topic 1} Video (11 min 15 sec)

\hypertarget{read-and-reflect}{%
\subsection*{Read and Reflect}\label{read-and-reflect}}
\addcontentsline{toc}{subsection}{Read and Reflect}

\begin{itemize}
\tightlist
\item
  Robert Nozick - \href{assets/u1/PHIL-100-Nozick-What-is-Wisdom.pdf}{\emph{What is Wisdom and Why Do Philosophers Love It So?}}
\end{itemize}

\hypertarget{required-nozick-reading-note-taking-exercise}{%
\subsection*{Required Nozick Reading Note-Taking Exercise}\label{required-nozick-reading-note-taking-exercise}}
\addcontentsline{toc}{subsection}{Required Nozick Reading Note-Taking Exercise}

\begin{reflect}
\begin{itemize}
\tightlist
\item
  Answering these questions will help you understand the Nozick reading and prepare you for your Unit 1 Reflection Assignment.
\item
  Click the Download button on the left after you've completed all the notes for this reading and save the answers to your computer.
\item
  You will be required to submit your downloaded notes to this reading as part of your Unit 1 Reflection Assignment.
\end{itemize}
\end{reflect}

\hypertarget{required-note-taking-questionnaire}{%
\subsection*{Required Note-Taking Questionnaire}\label{required-note-taking-questionnaire}}
\addcontentsline{toc}{subsection}{Required Note-Taking Questionnaire}

\begin{reflect}
\begin{itemize}
\tightlist
\item
  Answering these questions will help you reflect on the topic content and prepare you for your Unit 1 Reflection Assignment.
\item
  Click the Download button on the left after you've completed all the questions for this unit and save the answers to your computer.
\item
  You will be required to submit the downloaded answers to these questions as part of your Unit 1 Reflection Assignment.
\end{itemize}
\end{reflect}

\hypertarget{watch-and-reflect-1}{%
\subsection*{Watch and Reflect}\label{watch-and-reflect-1}}
\addcontentsline{toc}{subsection}{Watch and Reflect}

\begin{reflect}
In this 5 minute 55 second optional video, philosophers Valerie Tiberius and Philip Kitcher, as well as psychologist Lisa Feldman Barret discuss their definitions of wisdom.

\href{https://youtu.be/obqedyeUcwk}{Watch: \emph{What is Wisdom?}}
\end{reflect}

\hypertarget{wisdom-and-humility}{%
\section{Wisdom and Humility}\label{wisdom-and-humility}}

Follow and complete the steps below to accomplish your learning for this topic.

\begin{enumerate}
\def\labelenumi{\arabic{enumi}.}
\tightlist
\item
  Read Topic Notes
\item
  Watch Topic Video
\item
  Read Topic Reading
\item
  Complete Topic Exercise
\item
  Complete Topic Questionnaire
\item
  Watch Optional Video
\end{enumerate}

\hypertarget{topic-notes-1}{%
\subsection*{Topic Notes}\label{topic-notes-1}}
\addcontentsline{toc}{subsection}{Topic Notes}

Within this topic, our attention shifts to the intertwining realms of wisdom and humility. A wise individual possesses a keen awareness of the boundaries of their knowledge, acknowledging both what they comprehend and what lies beyond their understanding. Here, ``ignorance'' is not disparaging but denotes a specific lack of knowledge in a given domain.

The dialogue unfolds in Plato's ``the Apology,'' where Socrates, perplexed by the Oracle of Delphi's proclamation that no one is wiser than he, embarks on a quest to unravel the meaning of this paradox. Through dialogues with politicians, poets, and artisans, Socrates discovers that those who claim profound knowledge often lack true understanding. The pivotal distinction lies in Socrates' awareness of his ignorance, particularly regarding the nature of beauty and what is good, in contrast to others who falsely believe they possess knowledge. This self-awareness becomes the hallmark of Socrates' wisdom.

Applying the principle of acknowledging ignorance to the pursuit of wisdom involves cultivating a humble stance towards life's intricate questions. Whether grappling with the interpretation of the Bible, contemplating evolutionary theories, exploring the nature of sexuality, evaluating the impact of social media, or assessing the effectiveness of political leadership, a wise person approaches these complex issues with humility. This involves posing thoughtful questions, recognizing the limits of one's own understanding, refraining from hasty judgments, and maintaining a receptivity to evolving perspectives. In essence, humility becomes a guiding virtue in the pursuit of wisdom, fostering a mindset that values curiosity, open-mindedness, and the continual quest for understanding.

\hypertarget{watch-and-reflect-2}{%
\subsection*{Watch and Reflect}\label{watch-and-reflect-2}}
\addcontentsline{toc}{subsection}{Watch and Reflect}

In this video, you will learn about wisdom and humility from Plato's \emph{Apology}.

\textbf{Wisdom and Humility Unit 1 Topic 2} Video (13 min 54 sec)

\hypertarget{read-and-reflect-1}{%
\subsection*{Read and Reflect}\label{read-and-reflect-1}}
\addcontentsline{toc}{subsection}{Read and Reflect}

\begin{itemize}
\tightlist
\item
  Plato - \href{https://classics.mit.edu/Plato/apology.html}{\emph{Apology}}
\end{itemize}

\hypertarget{required-plato-reading-note-taking-exercise}{%
\subsection*{Required Plato Reading Note-Taking Exercise}\label{required-plato-reading-note-taking-exercise}}
\addcontentsline{toc}{subsection}{Required Plato Reading Note-Taking Exercise}

\begin{reflect}
\begin{itemize}
\tightlist
\item
  Answering these questions will help you understand the Plato reading and prepare you for your Unit 1 Reflection Assignment.
\item
  Click the Download button on the left after you've completed all the notes for this reading and save the answers to your computer.
\item
  You will be required to submit your downloaded notes to this reading as part of your Unit 1 Reflection Assignment.
\end{itemize}
\end{reflect}

\hypertarget{required-note-taking-questionnaire-1}{%
\subsection*{Required Note-Taking Questionnaire}\label{required-note-taking-questionnaire-1}}
\addcontentsline{toc}{subsection}{Required Note-Taking Questionnaire}

\begin{reflect}
\begin{itemize}
\tightlist
\item
  Answering these questions will help you reflect on the topic content and prepare you for your Unit 1 Reflection Assignment.
\item
  Click the Download button on the left after you've completed all the questions for this unit and save the answers to your computer.
\item
  You will be required to submit the downloaded answers to these questions as part of your Unit 1 Reflection Assignment.
\end{itemize}
\end{reflect}

\hypertarget{watch-and-reflect-3}{%
\subsection*{Watch and Reflect}\label{watch-and-reflect-3}}
\addcontentsline{toc}{subsection}{Watch and Reflect}

\begin{reflect}
In this 2 minute 59 second optional video with Dr.~Ian Church, \emph{What is Intellectual Humility}? A doxastic account. Here is the first of three parts. If you have the time, you should watch all three parts.

\href{https://www.youtube.com/watch?v=8CZIkGEJYRY}{Watch: \emph{What is Intellectual Humility?}}
\end{reflect}

\hypertarget{wisdom-and-friendship}{%
\section{Wisdom and Friendship}\label{wisdom-and-friendship}}

Follow and complete the steps below to accomplish your learning for this topic.

\begin{enumerate}
\def\labelenumi{\arabic{enumi}.}
\tightlist
\item
  Read Topic Notes\\
\item
  Watch Topic Video\\
\item
  Read Topic Reading\\
\item
  Complete Topic Exercise\\
\item
  Complete Topic Questionnaire\\
\item
  Watch Optional Video
\end{enumerate}

\hypertarget{topic-notes-2}{%
\subsection*{Topic Notes}\label{topic-notes-2}}
\addcontentsline{toc}{subsection}{Topic Notes}

The third topic explores the intricate relationship between wisdom and friendship, emphasizing the discerning approach a wise individual takes in understanding the various dimensions of companionship. To navigate this terrain effectively, it is imperative to comprehend the diverse types of friendships, as expounded by Aristotle.

Aristotle categorizes friendships into three distinct types: those based on utility, pleasure, and virtue. Friends of utility serve as instrumental connections, such as business partners or teammates in sports. Friends of pleasure share enjoyable activities, while friends of virtue embody individuals with admirable qualities, actively contributing to our moral development. Aristotle contends that friendships rooted in virtue are the most profound, albeit challenging to cultivate. Such friendships extend beyond mere utility or pleasure; they transcend difficulties, demonstrating a genuine concern for one another's character development.

Thomas Aquinas further enriches this discourse by cautioning against qualities that undermine the pursuit of virtue in friendships. Envy, manifesting as displeasure at a friend's success, and arrogant pride, reflected in an inability to handle jests or an inflated sense of self-worth, are identified as detrimental vices. Aquinas urges us to be vigilant in recognizing and avoiding these pitfalls, both in our friends and within ourselves.

Crucially, this exploration prompts introspection. Are our friendships solely utilitarian or pleasure-seeking? Do we harbor envy towards our friends' accomplishments? Do we exhibit arrogant pride? The video introduces these insights from Aristotle and Aquinas, providing valuable guidance to nurture robust and virtuous friendships conducive to a life well-lived.

\hypertarget{watch-and-reflect-4}{%
\subsection*{Watch and Reflect}\label{watch-and-reflect-4}}
\addcontentsline{toc}{subsection}{Watch and Reflect}

In this video, you will learn about wisdom and friendship from Aristotle's \emph{Nicomachean Ethics}, Books 8, 9 and 10, and some additional points from Thomas Aquinas.

\textbf{Wisdom and Friendship Unit 1 Topic 3} Video (24 min 18 sec)

\hypertarget{read-and-reflect-2}{%
\subsection*{Read and Reflect}\label{read-and-reflect-2}}
\addcontentsline{toc}{subsection}{Read and Reflect}

\begin{itemize}
\tightlist
\item
  Aristotle - \href{assets/u1/PHIL-100-Aristotle-NE-VIII-IX-X.pdf}{\emph{Nicomachean Ethics}, Books VIII, IX and X}
\end{itemize}

\hypertarget{required-aristotle-reading-note-taking-exercise}{%
\subsection*{Required Aristotle Reading Note-Taking Exercise}\label{required-aristotle-reading-note-taking-exercise}}
\addcontentsline{toc}{subsection}{Required Aristotle Reading Note-Taking Exercise}

\begin{reflect}
\begin{itemize}
\tightlist
\item
  Answering these questions will help you understand the Aristotle reading and prepare you for your Unit 1 Reflection Assignment.
\item
  Click the Download button on the left after you've completed all the notes for this reading and save the answers to your computer.
\item
  You will be required to submit your downloaded notes to this reading as part of your Unit 1 Reflection Assignment.
\end{itemize}
\end{reflect}

\hypertarget{required-note-taking-questionnaire-2}{%
\subsection*{Required Note-Taking Questionnaire}\label{required-note-taking-questionnaire-2}}
\addcontentsline{toc}{subsection}{Required Note-Taking Questionnaire}

\begin{reflect}
\begin{itemize}
\tightlist
\item
  Answering these questions will help you reflect on the topic content and prepare you for your Unit 1 Reflection Assignment.
\item
  Click the Download button on the left after you've completed all the questions for this unit and save the answers to your computer.
\item
  You will be required to submit the downloaded answers to these questions as part of your Unit 1 Reflection Assignment.
\end{itemize}
\end{reflect}

\hypertarget{watch-and-reflect-5}{%
\subsection*{Watch and Reflect}\label{watch-and-reflect-5}}
\addcontentsline{toc}{subsection}{Watch and Reflect}

\begin{reflect}
In this 7 minute 21 second optional video, you will learn further details about Aristotle's view of friendship.

\href{https://www.youtube.com/watch?v=F18kSA8OxqY}{Watch: \emph{Aristotle's Timeless Advice on What Real Friendship Is and Why It Matters}}
\end{reflect}

\hypertarget{wisdom-and-rhetoric}{%
\section{Wisdom and Rhetoric}\label{wisdom-and-rhetoric}}

Follow and complete the steps below to accomplish your learning for this topic.

\begin{enumerate}
\def\labelenumi{\arabic{enumi}.}
\tightlist
\item
  Read Topic Notes
\item
  Watch Topic Video
\item
  Read Topic Reading
\item
  Complete Topic Exercise
\item
  Complete Topic Questionnaire
\item
  Watch Optional Video
\end{enumerate}

\hypertarget{topic-notes-3}{%
\subsection*{Topic Notes}\label{topic-notes-3}}
\addcontentsline{toc}{subsection}{Topic Notes}

The final topic examines the intersection of wisdom and rhetoric, recognizing that engaging in arguments with others is an inherent aspect of daily life. A wise individual possesses the skill to navigate these discussions without descending into conflict. According to Jay Heinrichs, productive argumentation requires a clear understanding of the desired outcome.

The crucial skill in constructive argumentation involves defining the appropriate goal for the situation. Are you seeking to influence a decision, aiming for a sense of victory, or hoping to shift your opponent's mood or perspective? Perhaps your goal is to efficiently conclude the argument while maintaining a positive relationship with the other person. Each goal requires a tailored approach, prompting consideration of the feasibility of your objectives and the methods to achieve them.

Moreover, one ought to consider reflecting on the post-argument scenario. Do you wish to uphold the relationship after the disagreement? What compromises are you willing to make, and are you prepared to acknowledge defeat from your opponent's perspective if it serves your overarching goal? Heinrichs encourages a thoughtful examination of the argument's purpose, emphasizing the strategic choices and compromises necessary to align with your desired outcome.

In essence, this exploration aims to equip individuals with the skills and insights to engage in persuasive and constructive discussions, fostering an environment where arguments serve as tools for understanding and collaboration rather than sources of conflict.

\hypertarget{watch-and-reflect-6}{%
\subsection*{Watch and Reflect}\label{watch-and-reflect-6}}
\addcontentsline{toc}{subsection}{Watch and Reflect}

In this video, you will learn about wisdom and rhetoric from Jay Heinrich's \emph{Thank You for Arguing}, Chapter 2 ``Set Your Goals''.

\textbf{Wisdom and Rhetoric Unit 1 Topic 4} Video (12 min 19 sec)

\hypertarget{read-and-reflect-3}{%
\subsection*{Read and Reflect}\label{read-and-reflect-3}}
\addcontentsline{toc}{subsection}{Read and Reflect}

\begin{itemize}
\tightlist
\item
  Jay Heinrichs - \href{assets/u1/PHIL-100-Heinrichs-Thank-You-for-Arguing.pdf}{\emph{Thank You for Arguing}}
\end{itemize}

\hypertarget{required-heinrichs-reading-note-taking-exercise}{%
\subsection*{Required Heinrichs Reading Note-Taking Exercise}\label{required-heinrichs-reading-note-taking-exercise}}
\addcontentsline{toc}{subsection}{Required Heinrichs Reading Note-Taking Exercise}

\begin{reflect}
\begin{itemize}
\tightlist
\item
  Answering these questions will help you understand the Heinrichs reading and prepare you for your Unit 1 Reflection Assignment.
\item
  Click the Download button on the left after you've completed all the notes for this reading and save the answers to your computer.
\item
  You will be required to submit your downloaded notes to this reading as part of your Unit 1 Reflection Assignment.
\end{itemize}
\end{reflect}

\hypertarget{required-note-taking-questionnaire-3}{%
\subsection*{Required Note-Taking Questionnaire}\label{required-note-taking-questionnaire-3}}
\addcontentsline{toc}{subsection}{Required Note-Taking Questionnaire}

\begin{reflect}
\begin{itemize}
\tightlist
\item
  Answering these questions will help you reflect on the topic content and prepare you for your Unit 1 Reflection Assignment.\\
\item
  Click the Download button on the left after you've completed all the questions for this unit and save the answers to your computer.\\
\item
  You will be required to submit the downloaded answers to these questions as part of your Unit 1 Reflection Assignment.
\end{itemize}
\end{reflect}

\hypertarget{watch-and-reflect-7}{%
\subsection*{Watch and Reflect}\label{watch-and-reflect-7}}
\addcontentsline{toc}{subsection}{Watch and Reflect}

\begin{reflect}
In this 4 minute 29 second optional video. Here, Camille A. Langston, a specialist in rhetoric and nineteenth-century women, explains the basics of deliberative rhetoric and shares tips for appealing to the ethos, logos, and pathos of your audience.

\href{https://www.youtube.com/watch?v=3klMM9BkW5o}{Watch: \emph{How to Use Rhetoric to Get What You Want}}
\end{reflect}

\hypertarget{faith}{%
\chapter{Faith}\label{faith}}

In this video, we provide a succinct overview of the four key themes in the second unit: definitions, hope, action, and rationality. Our exploration will center around the intricate nature of faith, encompassing not only the attitudes associated with faith, such as beliefs and trust, but also delving into the actions inspired by faith, with a brief emphasis on ethical considerations. Lastly, our journey will lead us to scrutinize the rationality of faith, contemplating its alignment with evidence and the broader spectrum of reasoned thought. Join us as we navigate these nuanced dimensions of faith, seeking a deeper understanding of its essence and implications.

\textbf{Faith Unit 2 Introduction} Video (7 min 34 sec)

\hypertarget{overview-1}{%
\section*{Overview}\label{overview-1}}
\addcontentsline{toc}{section}{Overview}

\hypertarget{topics-1}{%
\subsection*{Topics}\label{topics-1}}
\addcontentsline{toc}{subsection}{Topics}

This unit is divided into the following four topics:

\begin{enumerate}
\def\labelenumi{\arabic{enumi}.}
\tightlist
\item
  Faith and Definitions\\
\item
  Faith and Hope\\
\item
  Faith and Action\\
\item
  Faith and Rationality
\end{enumerate}

\hypertarget{learning-outcomes-1}{%
\subsection*{Learning Outcomes}\label{learning-outcomes-1}}
\addcontentsline{toc}{subsection}{Learning Outcomes}

When you've completed this unit, you will have learned how to:

\begin{itemize}
\tightlist
\item
  Identify and understand the nature of faith, including definitions, ethical implications, and its relationship to rationality
\item
  Develop a greater understanding of the ethical behavior required by faith
\item
  Identify briefly how Christianity applies to living life within the context of justice and faith
\item
  Enhance a greater understanding about faith and rationality
\item
  Navigate some of the objections to the nature of faith, particularly the beliefs about faith
\end{itemize}

\hypertarget{faith-and-definitions}{%
\section{Faith and Definitions}\label{faith-and-definitions}}

Follow and complete the steps below to accomplish your learning for this topic.

\begin{enumerate}
\def\labelenumi{\arabic{enumi}.}
\tightlist
\item
  Read Topic Notes
\item
  Watch Topic Video
\item
  Read Topic Reading
\item
  Complete Note-Taking Exercise
\item
  Complete Topic Questionnaire
\item
  Watch Optional Video
\end{enumerate}

\hypertarget{topic-notes-4}{%
\subsection*{Topic Notes}\label{topic-notes-4}}
\addcontentsline{toc}{subsection}{Topic Notes}

The first topic includes considering the multifaceted definitions of the term ``faith.'' A pivotal distinction surfaces when considering faith both as an attitude and as an action. Faith as an attitude encapsulates the cognitive realm, residing in our thoughts and beliefs, encompassing notions of trust and hope. This conceptualization aligns with common perceptions of faith as a mental and cognitive experience.

In contrast, faith as an action introduces an ethical dimension to the term. Rooted in the Greek word ``pistis,'' faith, in this context, pertains to ethical behavior directed towards a person or an ideal. This perspective invites us to view faith not merely as a mental state but as a manifestation of ethical considerations in our actions and conduct.

Further nuances emerge within faith as an attitude, where we explore the differentiation between doxastic faith and non-doxastic faith. The term ``doxastic'' centers on belief, indicating a cognitive dimension to faith. Conversely, ``non-doxastic'' faith extends beyond mere belief, encompassing faith that operates independently of explicit cognitive assent.

This exploration aims to unravel the intricate layers of faith, encouraging a comprehensive understanding that spans both the cognitive and ethical dimensions of this complex and significant concept.

\hypertarget{topic-video}{%
\subsection*{Topic Video}\label{topic-video}}
\addcontentsline{toc}{subsection}{Topic Video}

In this video, you will learn about the various definitions of faith from Elizabeth Jackson's article, ``Faith: Contemporary Perspectives''. The main focus is on learning about various distinctions with the concept of faith, such as attitude faith and action faith, and doxastic and non-doxastic view of faith.

\textbf{Faith and Definitions Unit 2 Topic 1} Video (8 min 01 sec)

\hypertarget{topic-reading}{%
\subsection*{Topic Reading}\label{topic-reading}}
\addcontentsline{toc}{subsection}{Topic Reading}

\begin{itemize}
\tightlist
\item
  Jackson - \emph{Types of Faith} - \url{https://iep.utm.edu/faith-contemporary-perspectives/\#SSH1ai\%7Btarget=\%22_blank}``\}
\end{itemize}

\hypertarget{note-taking-exercise}{%
\subsection*{Note-Taking Exercise}\label{note-taking-exercise}}
\addcontentsline{toc}{subsection}{Note-Taking Exercise}

\begin{reflect}
\begin{itemize}
\tightlist
\item
  Answering these questions will help you understand the Jackson reading and prepare you for your Unit 2 Reflection Assignment.
\item
  Click the Download button on the left after you've completed all the notes for this reading and save the answers to your computer.
\item
  You will be required to submit your downloaded notes to this reading as part of your Unit 2 Reflection Assignment.
\end{itemize}
\end{reflect}

\hypertarget{topic-questionnaire}{%
\subsection*{Topic Questionnaire}\label{topic-questionnaire}}
\addcontentsline{toc}{subsection}{Topic Questionnaire}

\begin{reflect}
\begin{itemize}
\tightlist
\item
  Answering these questions will help you reflect on the topic content and prepare you for your Unit 2 Reflection Assignment.
\item
  Click the Download button on the left after you've completed all the questions for this unit and save the answers to your computer.
\item
  You will be required to submit the downloaded answers to these questions as part of your Unit 2 Reflection Assignment.
\end{itemize}
\end{reflect}

\hypertarget{optional-video}{%
\subsection*{Optional Video}\label{optional-video}}
\addcontentsline{toc}{subsection}{Optional Video}

\begin{reflect}
In this 10 minute 06 second optional video, you will learn some further distinctions between the terms faith and belief, especially in religious contexts.

\href{https://www.youtube.com/watch?v=o8QKkHWUSu0}{Watch: \emph{Belief vs Faith (Philosophical Distinction)}}
\end{reflect}

\hypertarget{faith-and-hope}{%
\section{Faith and Hope}\label{faith-and-hope}}

Follow and complete the steps below to accomplish your learning for this topic.

\begin{enumerate}
\def\labelenumi{\arabic{enumi}.}
\tightlist
\item
  Read Topic Notes
\item
  Watch Topic Video
\item
  Read Topic Reading
\item
  Complete Topic Exercise
\item
  Complete Topic Questionnaire
\item
  Watch Optional Video
\end{enumerate}

\hypertarget{topic-notes-5}{%
\subsection*{Topic Notes}\label{topic-notes-5}}
\addcontentsline{toc}{subsection}{Topic Notes}

Building on our exploration of faith as an attitude, particularly focusing on belief and trust in the previous topic, we now inquire into a non-doxastic facet of faith: hope. Unlike belief and trust, hope doesn't necessarily hinge on cognitive assent, marking it as a non-doxastic attitude within the realm of faith. Louis Pojman contends that understanding faith as hope is particularly relevant, especially in the context of Christian faith and the doctrine of salvation.

The concept of salvation suggests that humanity is in jeopardy, and God is endeavoring to rescue them. Traditional perspectives posit that belief in God is a prerequisite for salvation. However, Pojman challenges this notion, recognizing that humans don't have absolute control over their beliefs, including doubts about God. In this context, the idea that salvation is contingent on something beyond one's control raises questions. Pojman argues that faith as hope in God and his redemptive plan provides a more accurate and inclusive perspective on faith and salvation.

By emphasizing hope as a non-doxastic attitude within faith, this discussion broadens our understanding of faith beyond mere cognitive elements, offering a nuanced perspective that aligns with the complexities of belief, doubt, and the pursuit of salvation.

\hypertarget{watch-and-reflect-8}{%
\subsection*{Watch and Reflect}\label{watch-and-reflect-8}}
\addcontentsline{toc}{subsection}{Watch and Reflect}

In this video, you will learn about faith as hope in the midst of doubt from Louis Pojman's article, ``Faith, Hope, Doubt''. According to Pojman, faith is a matter of hope and not belief. Thus, faith is non-doxastic.

\textbf{Faith and Hope Unit 2 Topic 2} Video (10 min 26 sec)

\hypertarget{read-and-reflect-4}{%
\subsection*{Read and Reflect}\label{read-and-reflect-4}}
\addcontentsline{toc}{subsection}{Read and Reflect}

\begin{itemize}
\tightlist
\item
  Pojman - \href{assets/u2/PHIL-100-Pojman-Faith-Hope-and-Doubt.pdf}{\emph{Faith, Hope and Doubt}}
\end{itemize}

\hypertarget{required-pojman-reading-note-taking-exercise}{%
\subsection*{Required Pojman Reading Note-Taking Exercise}\label{required-pojman-reading-note-taking-exercise}}
\addcontentsline{toc}{subsection}{Required Pojman Reading Note-Taking Exercise}

\begin{reflect}
\begin{itemize}
\tightlist
\item
  Answering these questions will help you understand the Pojman reading and prepare you for your Unit 2 Reflection Assignment.
\item
  Click the Download button on the left after you've completed all the notes for this reading and save the answers to your computer.
\item
  You will be required to submit your downloaded notes to this reading as part of your Unit 2 Reflection Assignment.
\end{itemize}
\end{reflect}

\hypertarget{required-note-taking-questionnaire-4}{%
\subsection*{Required Note-Taking Questionnaire}\label{required-note-taking-questionnaire-4}}
\addcontentsline{toc}{subsection}{Required Note-Taking Questionnaire}

\begin{reflect}
\begin{itemize}
\tightlist
\item
  Answering these questions will help you reflect on the topic content and prepare you for your Unit 2 Reflection Assignment.
\item
  Click the Download button on the left after you've completed all the questions for this unit and save the answers to your computer.
\item
  You will be required to submit the downloaded answers to these questions as part of your Unit 2 Reflection Assignment.
\end{itemize}
\end{reflect}

\hypertarget{watch-and-reflect-9}{%
\subsection*{Watch and Reflect}\label{watch-and-reflect-9}}
\addcontentsline{toc}{subsection}{Watch and Reflect}

\begin{reflect}
In this 6 minute 17 second optional video, you will learn about faith and doubt by Richard Swinburne.

\href{https://www.youtube.com/watch?v=exsmSlxnbHQ}{Watch: \emph{Swinburne: On Doubt and Faith}}
\end{reflect}

\hypertarget{faith-and-action}{%
\section{Faith and Action}\label{faith-and-action}}

Follow and complete the steps below to accomplish your learning for this topic.

\begin{enumerate}
\def\labelenumi{\arabic{enumi}.}
\tightlist
\item
  Read Topic Notes
\item
  Watch Topic Video
\item
  Read Topic Reading
\item
  Complete Topic Exercise
\item
  Complete Topic Questionnaire
\item
  Watch Optional Video
\end{enumerate}

\hypertarget{topic-notes-6}{%
\subsection*{Topic Notes}\label{topic-notes-6}}
\addcontentsline{toc}{subsection}{Topic Notes}

In our recent discussions, we've examined faith as an attitude, encompassing beliefs, trust, and hope---mental and cognitive experiences that reside within an individual. Shifting the focus, we now explore faith as action, centering on behavioral aspects rather than solely on mental experiences. While beliefs, desires, and hopes may influence actions, this connection is not absolute. Consequently, faith as action directs attention to ethical behavior as the primary mode of expressing faith.

In this third topic, we briefly navigate a theological controversy within Christianity concerning the definition of faith. Traditionally, faith was construed as an attitudinal concept, emphasizing beliefs and trust. However, contemporary perspectives challenge this tradition, proposing that Christian faith is more intricately linked to Christian ethics and good works, de-emphasizing the importance of specific attitudinal beliefs and trust in Christianity.

This exploration invites us to reconsider the nature of faith within the Christian context, prompting reflection on whether faith should be predominantly characterized by mental attitudes or by the ethical conduct and good works that spring forth from one's beliefs.

\hypertarget{watch-and-reflect-10}{%
\subsection*{Watch and Reflect}\label{watch-and-reflect-10}}
\addcontentsline{toc}{subsection}{Watch and Reflect}

In this video, you will learn about faith and action from Boyd and Thorsen's chapter, ``Ethics in the Christian Scriptures.'' Unlike faith as belief, trust, and hope, faith as action focuses on behavior and ethics, rather than on mental content. You will also be introduced to the debate in Christian theology about faith and works.

\textbf{Faith and Action Unit 2 Topic 3} Video (8 min 54 sec)

\hypertarget{read-and-reflect-5}{%
\subsection*{Read and Reflect}\label{read-and-reflect-5}}
\addcontentsline{toc}{subsection}{Read and Reflect}

\begin{itemize}
\tightlist
\item
  Boyd and Thorsen - \href{assets/u2/PHIL-100-Boyd-and-Thorsen-Ethics-in-the-Christian-Scriptures.pdf}{\emph{Ethics in the Christian Scriptures}}
\end{itemize}

\hypertarget{required-boyd-thorsen-reading-note-taking-exercise}{%
\subsection*{Required Boyd \& Thorsen Reading Note-Taking Exercise}\label{required-boyd-thorsen-reading-note-taking-exercise}}
\addcontentsline{toc}{subsection}{Required Boyd \& Thorsen Reading Note-Taking Exercise}

\begin{reflect}
\begin{itemize}
\tightlist
\item
  Answering these questions will help you understand the Boyd \& Thorsen reading and prepare you for your Unit 2 Reflection Assignment.\\
\item
  Click the Download button on the left after you've completed all the notes for this reading and save the answers to your computer.\\
\item
  You will be required to submit your downloaded notes to this reading as part of your Unit 2 Reflection Assignment.
\end{itemize}
\end{reflect}

\hypertarget{required-note-taking-questionnaire-5}{%
\subsection*{Required Note-Taking Questionnaire}\label{required-note-taking-questionnaire-5}}
\addcontentsline{toc}{subsection}{Required Note-Taking Questionnaire}

\begin{reflect}
\begin{itemize}
\tightlist
\item
  Answering these questions will help you reflect on the topic content and prepare you for your Unit 2 Reflection Assignment.
\item
  Click the Download button on the left after you've completed all the questions for this unit and save the answers to your computer.
\item
  You will be required to submit the downloaded answers to these questions as part of your Unit 2 Reflection Assignment.
\end{itemize}
\end{reflect}

\hypertarget{watch-and-reflect-11}{%
\subsection*{Watch and Reflect}\label{watch-and-reflect-11}}
\addcontentsline{toc}{subsection}{Watch and Reflect}

\begin{reflect}
In this 9 minute 04 second optional video, Professor Peter Singer discusses the role of religion in ethics. Questions discussed include: Is something good because a divine being approves of it, or does the divine being approve of it because it is good? How do we know what is good without religion? How do we reconcile ethics from different religions coexisting in the same society?

\href{https://www.youtube.com/watch?v=w9QtjQ5Ow7Y}{Watch: \emph{Religion and Ethics}}
\end{reflect}

\hypertarget{faith-and-rationality}{%
\section{Faith and Rationality}\label{faith-and-rationality}}

Follow and complete the steps below to accomplish your learning for this topic.

\begin{enumerate}
\def\labelenumi{\arabic{enumi}.}
\tightlist
\item
  Read Topic Notes
\item
  Watch Topic Video
\item
  Read Topic Reading
\item
  Complete Topic Exercise
\item
  Complete Topic Questionnaire
\item
  Watch Optional Video
\end{enumerate}

\hypertarget{topic-notes-7}{%
\subsection*{Topic Notes}\label{topic-notes-7}}
\addcontentsline{toc}{subsection}{Topic Notes}

In our preceding discussions, we touched upon various definitions of faith, distinguishing between faith as attitudes and faith as actions. While these categories may overlap, they can also maintain distinct identities. However, a persistent challenge to these notions of faith arises from critics who question the rationality of faith, whether conceived as belief and trust or as action. This final topic examines this challenge.

The term ``rationality'' here pertains to the alignment of beliefs and actions with evidence. According to the evidentialist perspective, if beliefs and actions deviate from the evidence, they are deemed irrational. Hence, the objection posits that to exhibit rationality, one must align beliefs and actions according to the available evidence. The ensuing exploration assesses the strength of this objection, beginning with an examination of William Clifford's renowned work on the ethics of belief.

Clifford contends that it is ethically wrong to believe anything not proportional to the evidence. To evaluate this evidential objection, we consider several responses, including the potential incoherence of the objection, the observation that many of our rational beliefs are not strictly proportional to the evidence, and the recognition that, in certain cases of faith, there may be sufficient evidence to meet rational standards.

\hypertarget{watch-and-reflect-12}{%
\subsection*{Watch and Reflect}\label{watch-and-reflect-12}}
\addcontentsline{toc}{subsection}{Watch and Reflect}

In this video, you will learn about faith and rationality from W.K. Clifford's article, ``The Ethics of Belief.'' You will be introduced to the evidential objection to religious belief, which states that religious belief is rational only if there is evidence for religious belief, and some brief responses and replies to this evidential objection.

\textbf{Faith and Rationality Unit 2 Topic 4} Video (19 min 01 sec)

\hypertarget{read-and-reflect-6}{%
\subsection*{Read and Reflect}\label{read-and-reflect-6}}
\addcontentsline{toc}{subsection}{Read and Reflect}

\begin{itemize}
\tightlist
\item
  WK Clifford - \href{assets/u2/PHIL-100-Clifford-Ethics-of-Belief.pdf}{\emph{Ethics of Belief}}
\end{itemize}

\hypertarget{required-clifford-reading-note-taking-exercise}{%
\subsection*{Required Clifford Reading Note-Taking Exercise}\label{required-clifford-reading-note-taking-exercise}}
\addcontentsline{toc}{subsection}{Required Clifford Reading Note-Taking Exercise}

\begin{reflect}
\begin{itemize}
\tightlist
\item
  Answering these questions will help you understand the Clifford reading and prepare you for your Unit 2 Reflection Assignment.\\
\item
  Click the Download button on the left after you've completed all the notes for this reading and save the answers to your computer.\\
\item
  You will be required to submit your downloaded notes to this reading as part of your Unit 2 Reflection Assignment.
\end{itemize}
\end{reflect}

\hypertarget{required-note-taking-questionnaire-6}{%
\subsection*{Required Note-Taking Questionnaire}\label{required-note-taking-questionnaire-6}}
\addcontentsline{toc}{subsection}{Required Note-Taking Questionnaire}

\begin{reflect}
\begin{itemize}
\tightlist
\item
  Answering these questions will help you reflect on the topic content and prepare you for your Unit 2 Reflection Assignment.\\
\item
  Click the Download button on the left after you've completed all the questions for this unit and save the answers to your computer.\\
\item
  You will be required to submit the downloaded answers to these questions as part of your Unit 2 Reflection Assignment.
\end{itemize}
\end{reflect}

\hypertarget{watch-and-reflect-13}{%
\subsection*{Watch and Reflect}\label{watch-and-reflect-13}}
\addcontentsline{toc}{subsection}{Watch and Reflect}

\begin{reflect}
In this 8 minute 38 second optional video, you will learn about further details about the relationship between faith and reason.

\href{https://www.youtube.com/watch?v=MTPHXNMi9tA}{Watch: \emph{Religion: Reason and Faith}}
\end{reflect}

\hypertarget{reason}{%
\chapter{Reason}\label{reason}}

This final unit is about arguments. This unit is valuable because everything we've discussed so far, both about wisdom and faith, is built upon the foundation of arguments. We give arguments for our views about wisdom. We give arguments about issues of faith and rationality. This last unit provides the opportunity for you to learn a basic skill of how to identify an argument, whether the argument is located in a book, and more often on YouTube, a podcast, and in the news. Note that we do not cover how to evaluate arguments in this unit. Evaluating arguments is a skill taught in other philosophy courses.

In topic 1, we learn about the parts of the argument, such as the proper terms used to designate these different parts. If we wish to learn how to identify arguments, we must know what terms to look for. In topic 2, we learn about how to label and structure arguments. Often people do not organize their argument clearly and so we must learn how to reconstruct the argument so that we know the conclusion and the reasons for supporting the conclusion. In topic 3, we tackle slightly more complicated arguments and learn how to interpret what the author intended to say. As practiced in topic 2, people often organize their arguments (or, occasionally, non-arguments) in a confusing way. Moreover, we will learn about tricky terms often used in arguments, such as the terms ``if'' and ``then''. In the final topic, we practice identifying difficult arguments, which often include long texts, run on sentences, and additional irrelevant content to the argument being put forward. Our task in the final topic is to use our skills practiced in topics 1-3 to identify more difficult arguments.

\textbf{Reason Unit 3 Introduction} Video (5 min 41 sec)

\hypertarget{overview-2}{%
\section*{Overview}\label{overview-2}}
\addcontentsline{toc}{section}{Overview}

\hypertarget{topics-2}{%
\subsection*{Topics}\label{topics-2}}
\addcontentsline{toc}{subsection}{Topics}

This unit is divided into the following four topics:

\begin{enumerate}
\def\labelenumi{\arabic{enumi}.}
\tightlist
\item
  Parts and Keywords of an Argument\\
\item
  Labelling and Restructuring\\
\item
  Interpreting Arguments and Conditionals (If\ldots Then)\\
\item
  Identifying and Practicing Difficult Arguments
\end{enumerate}

\hypertarget{learning-outcomes-2}{%
\subsection*{Learning Outcomes}\label{learning-outcomes-2}}
\addcontentsline{toc}{subsection}{Learning Outcomes}

When you've completed this unit, you will have learned how to:

\begin{itemize}
\tightlist
\item
  Identify the parts of an argument and how to label and structure these parts\\
\item
  Interpret arguments with ambiguous terminology to be made more precise\\
\item
  Develop a greater understanding of how arguments work in contemporary culture, including issues of justice and faith\\
\item
  Practice the skills necessary for navigating arguments in literature, news, and podcasts\\
\item
  Appreciate the value and skill of knowing how to handle the basic features of an argument
\end{itemize}

\hypertarget{parts-and-keywords-of-an-argument}{%
\section{Parts and Keywords of an Argument}\label{parts-and-keywords-of-an-argument}}

Follow and complete the steps below to accomplish your learning for this topic.

\begin{enumerate}
\def\labelenumi{\arabic{enumi}.}
\tightlist
\item
  Read Topic Notes\\
\item
  Watch Topic Video\\
\item
  Complete Logic Exercise 1\\
\item
  Complete Logic Exercise 2\\
\item
  Watch Optional Video
\end{enumerate}

\hypertarget{topic-notes-8}{%
\subsection*{Topic Notes}\label{topic-notes-8}}
\addcontentsline{toc}{subsection}{Topic Notes}

In this topic, we learn about the parts of the argument and the words often used to designate these parts. Arguments are made up of statements. Statements are claims that are either true or false. Sometimes statements are whole sentences, while other times statements are phrases. For example, the question, ``Is there a God?'' is not a statement because the question does not state something that is either true or false. The answer to this question, ``yes there is a God,'' or ``no, there is no God'' are statements because they express claims that are either true or false. Other parts of the argument are the premises and conclusion. The premises are the statements that support the conclusion. The conclusion is the statement that is being argued for by the premises. The inference is the move from the premises to the conclusion. Imagine someone says they believe that God exists. This is their conclusion. You ask them, ``why think that God exists?'' Their response is going to be a reason (i.e.~a premise) or a set of reasons (i.e.~a set of premises) that are supposed to support their conclusion. The inference is the nature of the support from premise to conclusion.

Some of the primary keywords of an argument are terms that describe premises and conclusions. For example, the word ``therefore'' often picks out a conclusion. The statement that follows from the term ``therefore'' may act as a conclusion. Terms that describe premise can be ``all'' ``every'' ``some''. As we shall see, these types of words help us identify the premises of an argument.

\hypertarget{watch-and-reflect-14}{%
\subsection*{Watch and Reflect}\label{watch-and-reflect-14}}
\addcontentsline{toc}{subsection}{Watch and Reflect}

In this video, you will learn about wisdom and knowledge from Robert Nozick's piece, ``\emph{What is Wisdom and Why do Philosophers Love it So?}''

\textbf{Parts and Keywords of an Argument Unit 3 Topic 1} Video (6 min 46 sec)

\hypertarget{required-parts-of-an-argument-logic-exercise}{%
\subsection*{Required Parts of an Argument Logic Exercise}\label{required-parts-of-an-argument-logic-exercise}}
\addcontentsline{toc}{subsection}{Required Parts of an Argument Logic Exercise}

\begin{reflect}
\end{reflect}

\hypertarget{required-keywords-of-an-argument-logic-exercise}{%
\subsection*{Required Keywords of an Argument Logic Exercise}\label{required-keywords-of-an-argument-logic-exercise}}
\addcontentsline{toc}{subsection}{Required Keywords of an Argument Logic Exercise}

\begin{reflect}
\end{reflect}

\hypertarget{watch-and-reflect-15}{%
\subsection*{Watch and Reflect}\label{watch-and-reflect-15}}
\addcontentsline{toc}{subsection}{Watch and Reflect}

\begin{reflect}
In this 5 minute 24 second optional video, you will learn about the basic structure of an argument.

\href{https://www.youtube.com/watch?v=wbRxR53F3rI}{Watch: \emph{Critical Thinking \#3: Types of Arguments}}
\end{reflect}

\hypertarget{labelling-and-restructuring}{%
\section{Labelling and Restructuring}\label{labelling-and-restructuring}}

Follow and complete the steps below to accomplish your learning for this topic.

\begin{enumerate}
\def\labelenumi{\arabic{enumi}.}
\tightlist
\item
  Read Topic Notes\\
\item
  Watch Topic Video\\
\item
  Complete Logic Exercise 1\\
\item
  Complete Logic Exercise 2\\
\item
  Watch Optional Video
\end{enumerate}

\hypertarget{topic-notes-9}{%
\subsection*{Topic Notes}\label{topic-notes-9}}
\addcontentsline{toc}{subsection}{Topic Notes}

In the previous topic, we learned about the parts and keywords of an argument. In this topic, we learn how to label these parts of the argument and then restructure the argument vertically, with premises and conclusion, so that we may see the argument clearly. The steps for applying these tools follow this pattern: (i) focus on statements and ignore non-statements. Recall from the previous topic that arguments are made up of statements only and never non-statements. Thus, only focus on statements and cross out any non-statements. (ii) Locate the premises and conclusion amongst the set of statements using keywords. As discussed, premises and conclusions often have terms that designate them, such as ``therefore''.

Once the argument has been labelled, the next step is to restructure the argument vertically with the premises on top and the conclusion beneath. This way we can see what the author's argument is exactly. That is, we can see the conclusion -- the main point of their position -- and the reasons (i.e.~premises) for thinking the conclusion is true. Once these statements are ordered correctly, anyone can see the argument. The reader needn't labour over the text and the many confusing non-statements to locate the point of the passage.

\textbf{Labelling and Restructuring Unit 3 Topic 2} Video (8 min 56 sec)

\hypertarget{required-labelling-an-argument-exercise}{%
\subsection*{Required Labelling an Argument Exercise}\label{required-labelling-an-argument-exercise}}
\addcontentsline{toc}{subsection}{Required Labelling an Argument Exercise}

\begin{reflect}
topic-4/chromeless:true/hidepagetitle:true'' allowfullscreen=``allowfullscreen'' width=``100\%'' height=``12800''\textgreater{}
\end{reflect}

\hypertarget{required-restructuring-an-argument-exercise}{%
\subsection*{Required Restructuring an Argument Exercise}\label{required-restructuring-an-argument-exercise}}
\addcontentsline{toc}{subsection}{Required Restructuring an Argument Exercise}

\begin{reflect}
\end{reflect}

\hypertarget{watch-and-reflect-16}{%
\subsection*{Watch and Reflect}\label{watch-and-reflect-16}}
\addcontentsline{toc}{subsection}{Watch and Reflect}

\begin{reflect}
In this 1 minute 22 second optional video, you will learn about the basic structure of an argument, with brief attention to what makes a bad argument.

\href{https://www.youtube.com/watch?v=Z7f_uuy1JcM}{Watch: \emph{Ninety Second Philosophy: Arguments}}
\end{reflect}

\hypertarget{interpreting-arguments-and-conditionals-ifthen}{%
\section{Interpreting Arguments and Conditionals (If\ldots Then)}\label{interpreting-arguments-and-conditionals-ifthen}}

Follow and complete the steps below to accomplish your learning for this topic.

\begin{enumerate}
\def\labelenumi{\arabic{enumi}.}
\tightlist
\item
  Read Topic Notes\\
\item
  Watch Topic Video\\
\item
  Complete Logic Exercise 1\\
\item
  Complete Logic Exercise 2\\
\item
  Watch Optional Video
\end{enumerate}

\hypertarget{topic-notes-10}{%
\subsection*{Topic Notes}\label{topic-notes-10}}
\addcontentsline{toc}{subsection}{Topic Notes}

In the first two topics, we learned some foundational skills for identifying arguments, such as the parts and keywords of an argument, and the methods for labeling and restructuring arguments. The challenge, however, is that ordinary discourse -- both written and verbal -- often makes the argument difficult to identify. We often must interpret what we think the author intended to say about their view and then structure the argument according to this intention. In this topic, we practice the skill of interpretating arguments. One strategy for interpreting an argument is the ``Why/Because'' strategy. If it's difficult to identify the premises and conclusion, then look for statements that may follow with a hypothetical ``why''. Such statements may be the conclusion(s). The reason is that concluding statements have by definition reasons for them. And the term ``why'' may pick out these reasons. The term ``because'' often precedes a reason (i.e.~a premise) and so looking for statements that may be preceded with ``because'' may help with identifying the premises.

A second skill of interpretation is learning to navigate conditional statements. A conditional is a ``if \ldots{} then'' statement. For example, ``If it's raining outside, then we can't go to the park.'' Sentences with the words ``if'' and ``then'' may be interpreted in different ways depending on the order of these words. In this topic we practice some of these interpretive skills.

\textbf{Interpreting Arguments and Conditionals (If\ldots Then) Unit 3 Topic 3} Video (15 min 32 sec)

\hypertarget{required-interpreting-an-argument-exercise}{%
\subsection*{Required Interpreting an Argument Exercise}\label{required-interpreting-an-argument-exercise}}
\addcontentsline{toc}{subsection}{Required Interpreting an Argument Exercise}

\begin{reflect}
\end{reflect}

\hypertarget{required-arguments-with-ifthen-exercise}{%
\subsection*{Required Arguments with ``If\ldots Then'' Exercise}\label{required-arguments-with-ifthen-exercise}}
\addcontentsline{toc}{subsection}{Required Arguments with ``If\ldots Then'' Exercise}

\begin{reflect}
\end{reflect}

\hypertarget{watch-and-reflect-17}{%
\subsection*{Watch and Reflect}\label{watch-and-reflect-17}}
\addcontentsline{toc}{subsection}{Watch and Reflect}

\begin{reflect}
In this 6 minute optional video, you will be introduced to some further techniques for identifying arguments.

\href{https://www.youtube.com/watch?v=lYiEj5z8le8}{Watch: \emph{Lesson 2. Identifying Arguments}}
\end{reflect}

\hypertarget{identifying-and-practicing-difficult-arguments}{%
\section{Identifying and Practicing Difficult Arguments}\label{identifying-and-practicing-difficult-arguments}}

Follow and complete the steps below to accomplish your learning for this topic.

\begin{enumerate}
\def\labelenumi{\arabic{enumi}.}
\tightlist
\item
  Read Topic Notes\\
\item
  Watch Topic Video\\
\item
  Complete Logic Exercise\\
\item
  Complete Practice Exercises\\
\item
  Watch Optional Video
\end{enumerate}

\hypertarget{topic-notes-11}{%
\subsection*{Topic Notes}\label{topic-notes-11}}
\addcontentsline{toc}{subsection}{Topic Notes}

In the previous topics, we have attempted to build our foundation for identifying arguments. The real challenge is when we're confronted with long speeches, texts, reports, stories, and accusations whereby the argument may often seem unclear. Our task is to use the tools we've learned to identify the argument(s) in more difficult contexts. In this topic, we practice these skills by following these steps: (i) ignore non-statements and label the premises and conclusion; (ii) restructure the argument vertically with premises on top and the conclusion beneath. Notice that the difficult step is (i). We must sift through the content to locate keywords, interpret ambiguous statements, possibly use the ``Why/Because'' strategy in cases of doubt, decipher unusual ``if/then'' statements, and so on. Sometimes there is no argument at all; other times there may exist more than one argument, such that there is a main argument, and a secondary argument. The point of this entire exercise is to learn the skill that so many of us fail to properly master: to learn what the opposing person is trying to say. What is their position? What is their argument if there is one? Only after we've successfully identified the argument can we then properly raise objections and evaluate their argument.

\textbf{Identifying and Practicing Difficult Arguments Unit 3 Topic 4} Video (8 min 03 sec)

\hypertarget{required-identifying-difficult-arguments-exercise}{%
\subsection*{Required Identifying Difficult Arguments Exercise}\label{required-identifying-difficult-arguments-exercise}}
\addcontentsline{toc}{subsection}{Required Identifying Difficult Arguments Exercise}

\begin{reflect}
\end{reflect}

\hypertarget{optional-practice-exercises}{%
\subsection*{Optional Practice Exercises}\label{optional-practice-exercises}}
\addcontentsline{toc}{subsection}{Optional Practice Exercises}

\begin{reflect}
These practice exercises are optional and not graded. However, the more you practice and become familiar with identifying arguments, the more prepared you will be for your Final Course Reflection Assignment.

If you want to keep your answers for future reference, click the Download button on the left after you've completed the exercise and save the answers to your computer.
\end{reflect}

\hypertarget{watch-and-reflect-18}{%
\subsection*{Watch and Reflect}\label{watch-and-reflect-18}}
\addcontentsline{toc}{subsection}{Watch and Reflect}

\begin{reflect}
In this 5 minute 34 second optional video, you will be introduced to strategies for locating terms that describe parts of the argument in difficult passages, such as terms that pick out conclusions and premises.

\href{https://www.youtube.com/watch?v=07mehbgE5jc}{Watch: \emph{Identifying Premises and Conclusions}}
\end{reflect}

\hypertarget{assessment}{%
\chapter*{Assessment}\label{assessment}}
\addcontentsline{toc}{chapter}{Assessment}

The following assignments are opportunities for learners to demonstrate their understanding of the course outcomes. Please confirm assignment details with your instructor, referring to the course syllabus.

Note that Assignment dropboxes are located in Moodle. Also refer to the Course Schedule in Moodle for the specific due dates.

\hypertarget{assignment}{%
\section*{Assignment:}\label{assignment}}
\addcontentsline{toc}{section}{Assignment:}

\hypertarget{grading-criteria}{%
\subsection*{Grading Criteria}\label{grading-criteria}}
\addcontentsline{toc}{subsection}{Grading Criteria}

See the following rubric that explains how your assignment will be evaluated. Also available as a \href{assets/assessment/Identity-as-a-Teacher-RUBRIC.pdf}{pdf}

\textbf{APA/WRITING}

\textbf{Unsatisfactory:} Paper does not model language and conventions used in scholarly literature. Writing is not well-organized. Several errors in grammar or composition. Sources are not cited. APA citations are not appropriately formatted.

\textbf{Developing:} Paper partially models language and conventions used in scholarly literature. Writing is somewhat well organized and includes some errors in grammar or composition. Not all sources cited. APA citations are generally formatted correctly, with several errors.

\textbf{Proficient:} \emph{Paper consistently models language and conventions used in scholarly literature. Writing is well-organized and includes few (if any) errors in grammar or composition. All resources are appropriately cited (including in-text citations and bibliography information). Few (if any) errors in APA citations.}

\textbf{Exemplary:} Paper is an exemplar of language and conventions used in scholarly literature. Writing is well-organized and free of errors in grammar or composition. All resources are appropriately cited. No errors in APA format.

\hypertarget{statement-of-teaching-identity}{%
\subsubsection*{STATEMENT OF TEACHING IDENTITY}\label{statement-of-teaching-identity}}
\addcontentsline{toc}{subsubsection}{STATEMENT OF TEACHING IDENTITY}

\textbf{Unsatisfactory:} Does not provide a statement about identity as a teacher/facilitator

\textbf{Developing:} Provides an unclear statement about identity as a teacher/facilitator.

\textbf{Proficient:} \emph{Provides a clear, concise, and powerful statement about identity as a teacher/facilitator.}

\textbf{Exemplary:} Provides a clear, concise, and powerful statement about identity as a teacher/facilitator. Statement incorporates theory or research from course materials.

\hypertarget{developing-a-cohesive-and-logical-academic-argument}{%
\subsubsection*{DEVELOPING A COHESIVE AND LOGICAL ACADEMIC ARGUMENT}\label{developing-a-cohesive-and-logical-academic-argument}}
\addcontentsline{toc}{subsubsection}{DEVELOPING A COHESIVE AND LOGICAL ACADEMIC ARGUMENT}

\textbf{Unsatisfactory:} Does not make a focused, cohesive, or logical academic argument. Paper is confusing, and is missing an introduction, body, or conclusion. Transitions between sections and ideas are missing.

\textbf{Developing:} Makes an academic argument that is only partially focused, cohesive and logical. Paper is generally organized, but is missing an introduction, body, or conclusion. Transitions between sections and ideas are unclear.

\textbf{Proficient:} \emph{Makes a focused, cohesive, logical academic argument. Paper is effectively organized and includes an introduction, body, and conclusion. Transitions between sections and ideas are clear.}

\textbf{Exemplary:} Makes a focused, cohesive, logical and compelling academic argument. Paper is effectively organized and includes an introduction, body, and conclusion. Transitions between sections and ideas are clear, and build on each other.

\hypertarget{analysis-of-identity-as-a-teacher}{%
\subsubsection*{ANALYSIS OF IDENTITY AS A TEACHER}\label{analysis-of-identity-as-a-teacher}}
\addcontentsline{toc}{subsubsection}{ANALYSIS OF IDENTITY AS A TEACHER}

\textbf{Unsatisfactory:} Does not include three important aspects of identity as a teacher/facilitator. Does not include an analysis.

\textbf{Developing:} Lists but does not discuss three important aspects of identity as a teacher/facilitator. Includes a partial analysis.

\textbf{Proficient:} \emph{Includes a detailed discussion of three important aspects of identity as a teacher/facilitator. Includes thoughtful analysis of each of the three elements.}

\textbf{Exemplary:} Includes a detailed discussion of three important aspects of identity as a teacher/facilitator. Includes a thoughtful analysis, integrating scholarly literature to support analysis and furthering scholarly thinking related to teacher identity.

\hypertarget{scholarly-integration}{%
\subsubsection*{SCHOLARLY INTEGRATION}\label{scholarly-integration}}
\addcontentsline{toc}{subsubsection}{SCHOLARLY INTEGRATION}

\textbf{Unsatisfactory:} Does not integrate references to support claims and assertions made in the paper.

\textbf{Developing:} Integrates references to support some of the claims and assertions made in the paper.

\textbf{Proficient:} \emph{Integrates references to support claims and assertions made in the paper.}

\textbf{Exemplary:} Integrates references to support claims and assertions made in the paper, effectively synthesizing different perspectives and research results from scholarly sources.

\begin{longtable}[]{@{}
  >{\raggedright\arraybackslash}p{(\columnwidth - 8\tabcolsep) * \real{0.2000}}
  >{\raggedright\arraybackslash}p{(\columnwidth - 8\tabcolsep) * \real{0.2000}}
  >{\raggedright\arraybackslash}p{(\columnwidth - 8\tabcolsep) * \real{0.2000}}
  >{\raggedright\arraybackslash}p{(\columnwidth - 8\tabcolsep) * \real{0.2000}}
  >{\raggedright\arraybackslash}p{(\columnwidth - 8\tabcolsep) * \real{0.2000}}@{}}
\toprule\noalign{}
\begin{minipage}[b]{\linewidth}\raggedright
\textbf{TOTAL}
\end{minipage} & \begin{minipage}[b]{\linewidth}\raggedright
\textbf{0 = 0\% (F)}
\end{minipage} & \begin{minipage}[b]{\linewidth}\raggedright
\textbf{10 = 50\% (C)}
\end{minipage} & \begin{minipage}[b]{\linewidth}\raggedright
\textbf{15 = 75 (B)}
\end{minipage} & \begin{minipage}[b]{\linewidth}\raggedright
\textbf{20 = 100\% (A+)}
\end{minipage} \\
\midrule\noalign{}
\endhead
\bottomrule\noalign{}
\endlastfoot
\end{longtable}

\begin{center}\rule{0.5\linewidth}{0.5pt}\end{center}

\hypertarget{assignment-company-website-analysis}{%
\section*{Assignment: Company Website Analysis}\label{assignment-company-website-analysis}}
\addcontentsline{toc}{section}{Assignment: Company Website Analysis}

\begin{assessment}
Investigate the Human Resources or Faculty Development portion of a
company's website, a higher education institution or adult learning
facility, preferably one with which you are familiar. Focus on the
faculty or employee development part of the website. In this assignment,
you will apply the theory of teaching in/for/with depth by analyzing the
learning culture of an organization.

In a 4-5 page APA formatted paper, analyze the website by responding to
the following questions in your report:

\begin{enumerate}
\def\labelenumi{\arabic{enumi}.}
\tightlist
\item
  What can you infer about the company's learning culture?\\
\item
  From what is visible on the public website, would you say it is an
  authentic learning community? Why or why not? Discuss whether the
  website reflects aspects of one or more of the learning community
  models explored in previous lessons.\\
\item
  Do you see evidence that interconnectedness and integrity are valued?
  Explain.\\
\item
  What traits and skills seem to be valued in employees?\\
\item
  How does the company develop skills in its employees (e.g., workshops,
  seminars, mentoring)? Are the methods based on the principles of
  andragogy? (see Smith YouTube video). What specific adult learning
  strategies do you see reflected in the development/training
  opportunities for employees?
\end{enumerate}

Your paper should be 4-5 pages and should incorporate references to at
least five scholarly sources you have studied in this course, or other
scholarly sources you have identified.

The paper should include:

\begin{enumerate}
\def\labelenumi{\arabic{enumi}.}
\tightlist
\item
  Introduction\\
\item
  Analysis (responding to the prompts)\\
\item
  Conclusion\\
\item
  Reference List
\end{enumerate}
\end{assessment}

\hypertarget{company-website-analysis-rubric}{%
\subsection*{Company Website Analysis Rubric}\label{company-website-analysis-rubric}}
\addcontentsline{toc}{subsection}{Company Website Analysis Rubric}

See the following rubric that explains how your assignment will be evaluated. Also available as a \href{assets/assessment/Company-Website-Analysis-RUBRIC.pdf}{pdf}

\hypertarget{apa-formatting}{%
\subsubsection*{APA Formatting}\label{apa-formatting}}
\addcontentsline{toc}{subsubsection}{APA Formatting}

\textbf{Unsatisfactory:} Paper does not model language and conventions used in scholarly literature.
Writing is not well-organized. Several errors in grammar or composition. Sources
are not cited. APA citations are not appropriately formatted.

\textbf{Developing:} Paper partially models language and conventions used in scholarly literature.
Writing is somewhat well organized and includes some errors in grammar or
composition. Not all sources cited. APA citations are generally formatted
correctly, with several errors.

\textbf{Proficient:} \emph{Paper consistently models language and conventions used in scholarly
literature. Writing is well-organized and includes few (if any) errors in
grammar or composition. All resources are appropriately cited (including in-text
citations and bibliography information). Few (if any) errors in APA citations.}

\textbf{Exemplary:} Paper is an exemplar of language and conventions used in scholarly literature.
Writing is well-organized and free of errors in grammar or composition. All
resources are appropriately cited. No errors in APA format.

\hypertarget{developing-a-cohesive-and-logical-academic-argument-1}{%
\subsubsection*{DEVELOPING a COHESIVE and LOGICAL ACADEMIC ARGUMENT}\label{developing-a-cohesive-and-logical-academic-argument-1}}
\addcontentsline{toc}{subsubsection}{DEVELOPING a COHESIVE and LOGICAL ACADEMIC ARGUMENT}

\textbf{Unsatisfactory:} Does not make a focused, cohesive, or logical academic argument. Paper is
confusing, and is missing an introduction, body, or conclusion. Transitions
between sections and ideas are missing.

\textbf{Developing:} Makes an academic argument that is only partially focused, cohesive and logical.
Paper is generally organized, but is missing an introduction, body, or
conclusion. Transitions between sections and ideas are unclear.

\textbf{Proficient:} \emph{Makes a focused, cohesive, logical academic argument. Paper is effectively
organized and includes an introduction, body, and conclusion. Transitions
between sections and ideas are clear.}

\textbf{Exemplary:} Makes a focused, cohesive, logical and compelling academic argument. Paper is
effectively organized and includes an introduction, body, and conclusion.
Transitions between sections and ideas are clear and build on each other.

\hypertarget{analysis-of-learning-culture}{%
\subsubsection*{ANALYSIS of LEARNING CULTURE}\label{analysis-of-learning-culture}}
\addcontentsline{toc}{subsubsection}{ANALYSIS of LEARNING CULTURE}

\textbf{Unsatisfactory:} Does not include an analysis of the company learning culture, and no evaluation
of the authenticity of the learning community.

\textbf{Developing:} Includes a partial analysis of the company learning culture, including a limited
evaluation of the authenticity of the learning community.

\textbf{Proficient:} \emph{Includes a detailed analysis of the company learning culture, including an
evaluation of the authenticity of the learning community.}

\textbf{Exemplary:} Includes a detailed analysis of the company learning culture, including an
evaluation of the authenticity of the learning community. Includes a thoughtful
analysis, integrating scholarly literature to support analysis and furthering
scholarly thinking related to teacher identity.

\hypertarget{evaluation-of-interconnectedness-and-integrity}{%
\subsubsection*{EVALUATION of INTERCONNECTEDNESS and INTEGRITY}\label{evaluation-of-interconnectedness-and-integrity}}
\addcontentsline{toc}{subsubsection}{EVALUATION of INTERCONNECTEDNESS and INTEGRITY}

\textbf{Unsatisfactory:} Does not include an evaluation of evidence of interconnectedness and integrity
on the company website. Does not integrate scholarly sources in the evaluation.

\textbf{Developing:} Includes a partial evaluation of evidence of interconnectedness and integrity on
the company website. Evaluation includes only limited reference to scholarly
sources.

\textbf{Proficient:} \emph{Includes a detailed evaluation of evidence of interconnectedness and integrity
on the company website. Evaluation integrates scholarly sources.}

\textbf{Exemplary:} Includes a detailed evaluation of evidence of interconnectedness and integrity
on the company website. Includes recommendations for ways in which to integrate
interconnectedness and integrity into employee development.

\hypertarget{analysis-of-adult-learning-strategies}{%
\subsubsection*{ANALYSIS of ADULT LEARNING STRATEGIES}\label{analysis-of-adult-learning-strategies}}
\addcontentsline{toc}{subsubsection}{ANALYSIS of ADULT LEARNING STRATEGIES}

\textbf{Unsatisfactory:} Does not include a detailed analysis of valued skills and evidence of adult
learning theory in employee development. Does not integrate scholarly sources.

\textbf{Developing:} Includes a partial analysis of valued skills and evidence of adult learning
theory in employee development. Analysis integrates few, if any, scholarly
sources.

\textbf{Proficient:} \emph{Includes a detailed analysis of valued skills and evidence of adult learning
theory in employee development. Analysis integrates scholarly sources.}

\textbf{Exemplary:} Includes a detailed analysis of valued skills and evidence of adult learning
theory in employee development. Includes recommendations for ways in which to
integrate adult learning theory into employee development.

\hypertarget{scholarly-integration-1}{%
\subsubsection*{SCHOLARLY INTEGRATION}\label{scholarly-integration-1}}
\addcontentsline{toc}{subsubsection}{SCHOLARLY INTEGRATION}

\textbf{Unsatisfactory:} Does not integrate scholarly references to support claims and assertions made in
the paper.

\textbf{Developing:} Integrates scholarly references to support some of the claims and assertions
made in the paper.

\textbf{Proficient:} \emph{Integrates scholarly references to support claims and assertions made in the
paper.}

\textbf{Exemplary:} Integrates scholarly references to support claims and assertions made in the
paper, effectively synthesizing different perspectives and research results from
scholarly sources.

\begin{longtable}[]{@{}
  >{\raggedright\arraybackslash}p{(\columnwidth - 8\tabcolsep) * \real{0.2000}}
  >{\raggedright\arraybackslash}p{(\columnwidth - 8\tabcolsep) * \real{0.2000}}
  >{\raggedright\arraybackslash}p{(\columnwidth - 8\tabcolsep) * \real{0.2000}}
  >{\raggedright\arraybackslash}p{(\columnwidth - 8\tabcolsep) * \real{0.2000}}
  >{\raggedright\arraybackslash}p{(\columnwidth - 8\tabcolsep) * \real{0.2000}}@{}}
\toprule\noalign{}
\begin{minipage}[b]{\linewidth}\raggedright
\textbf{TOTAL}
\end{minipage} & \begin{minipage}[b]{\linewidth}\raggedright
\textbf{0 = 0\% (F)}
\end{minipage} & \begin{minipage}[b]{\linewidth}\raggedright
\textbf{10 = 50\% (C)}
\end{minipage} & \begin{minipage}[b]{\linewidth}\raggedright
\textbf{15 = 75 (B)}
\end{minipage} & \begin{minipage}[b]{\linewidth}\raggedright
\textbf{20 = 100\% (A+)}
\end{minipage} \\
\midrule\noalign{}
\endhead
\bottomrule\noalign{}
\endlastfoot
\end{longtable}

\begin{center}\rule{0.5\linewidth}{0.5pt}\end{center}

\hypertarget{assignment-platform-paper}{%
\section*{Assignment: Platform Paper}\label{assignment-platform-paper}}
\addcontentsline{toc}{section}{Assignment: Platform Paper}

\begin{assessment}
For this assignment, you will write a contextualized Platform Paper in
which you discuss your ideal learning community and your role as
teacher/leader of that learning community. Select a context for your
paper (i.e.~facilitating in a FAR Centre in a specific country, teaching
adult learners, facilitating employee development workshops, etc.). Your
paper should be written and referenced in APA format and include
references to a minimum of 10 scholarly sources (this can include
literature you read in this course). You will write a draft of the
Platform Paper in Unit 8 and post for Peer Review. In Unit 9, you will
provide feedback to another learner on their paper. You will make
revisions based on the Peer Review and, in Unit 10, you will submit the
final Platform Paper. Peer reviewers will be assigned in advance.

{Paper Outline}

This paper will be 12-15 pages long, and should include:

\begin{enumerate}
\def\labelenumi{\arabic{enumi}.}
\tightlist
\item
  Introduction (1-2 pages)\\
\item
  Section 1: Ideal Learning Environment (5-7 pages)\\
\item
  Section 2: Your Role as Teacher and Leader (5-7 pages)\\
\item
  Conclusion (1-2 pages)
\end{enumerate}

{Paper Guidelines}

\begin{itemize}
\tightlist
\item
  \textbf{Introduction}: Introduce the two sections in your paper,
  providing a brief description of the key points you will make in each
  section.\\
\item
  \textbf{Section 1}: In section one, you will describe your ideal
  education learning environment. This section should demonstrate your
  learning about authentic learning communities, incorporating scholarly
  sources and your own analysis to depict your ideal learning
  environment. Incorporate a discussion of the learning community
  environment, learning experiences, student learning outcomes, and
  personal beliefs about teaching and learning.\\
\item
  \textbf{Section 2}: In this section, describe your role as a teacher
  or leader within an authentic learning community. Incorporating
  scholarly literature, analyze your role as a facilitator/leader in
  planning learning experiences, facilitating student learning, and
  assessing student learning. Describe the actions, practices, and
  strategies you will engage in to achieve your vision of the learning
  community you described in section one.\\
\item
  \textbf{Conclusion}: Summarize the key points you made in each
  section.\\
\item
  \textbf{References}: Include a reference list with references to at
  least 10 scholarly sources.
\end{itemize}
\end{assessment}

\hypertarget{platform-paper-rubric}{%
\subsection*{Platform Paper Rubric}\label{platform-paper-rubric}}
\addcontentsline{toc}{subsection}{Platform Paper Rubric}

See the following rubric that explains how your assignment will be evaluated. Also available as a \href{assets/assessment/Platform-Paper-RUBRIC.pdf}{pdf}

\hypertarget{apawriting}{%
\subsubsection*{APA/WRITING}\label{apawriting}}
\addcontentsline{toc}{subsubsection}{APA/WRITING}

\textbf{Unsatisfactory:} Paper does not model language and conventions used in scholarly literature. Writing is not well-organized. Several errors in grammar or composition. Sources are not cited. APA citations are not appropriately formatted.

\textbf{Developing:} Paper partially models language and conventions used in scholarly literature. Writing is somewhat well organized and includes some errors in grammar or composition. Not all sources cited. APA citations are generally formatted correctly, with several errors.

\textbf{Proficient:} \emph{Paper consistently models language and conventions used in scholarly literature. Writing is well-organized and includes few (if any) errors in grammar or composition. All resources are appropriately cited (including in-text citations and bibliography information). Few (if any) errors in APA citations.}

\textbf{Exemplary:} Paper is an exemplar of language and conventions used in scholarly literature. Writing is well-organized and free of errors in grammar or composition. All resources are appropriately cited. No errors in APA format.

\hypertarget{developing-a-cohesive-and-logical-academic-argument-2}{%
\subsubsection*{DEVELOPING a COHESIVE and LOGICAL ACADEMIC ARGUMENT}\label{developing-a-cohesive-and-logical-academic-argument-2}}
\addcontentsline{toc}{subsubsection}{DEVELOPING a COHESIVE and LOGICAL ACADEMIC ARGUMENT}

\textbf{Unsatisfactory:} Does not make a focused, cohesive, or logical academic argument. Paper is confusing, and is missing an introduction, body, or conclusion. Transitions between sections and ideas are missing.

\textbf{Developing:} Makes an academic argument that is only partially focused, cohesive and logical. Paper is generally organized, but is missing an introduction, body, or conclusion. Transitions between sections and ideas are unclear.

\textbf{Proficient:} \emph{Makes a focused, cohesive, logical academic argument. Paper is effectively organized and includes an introduction, body, and conclusion. Transitions between sections and ideas are clear.}

\textbf{Exemplary:} Makes a focused, cohesive, logical and compelling academic argument. Paper is effectively organized and includes an introduction, body, and conclusion. Transitions between sections and ideas are clear, and build on each other.

\hypertarget{ideal-learning-environment}{%
\subsubsection*{IDEAL LEARNING ENVIRONMENT}\label{ideal-learning-environment}}
\addcontentsline{toc}{subsubsection}{IDEAL LEARNING ENVIRONMENT}

\textbf{Unsatisfactory:} Does not include a description of your ideal learning environment. Does not reference scholarly sources. Does note analyze key elements of an authentic learning community. Does not mention or describe the learning community environment, student learning outcomes, learning outcomes and personal beliefs about teaching and learning.

\textbf{Developing:} Includes a partial description of your ideal learning environment, referencing few scholarly sources and including a partial analysis of key elements of an authentic learning community. Mentions some elements, but does not fully describe the learning community environment, student learning outcomes, learning outcomes and personal beliefs about teaching and learning.

\textbf{Proficient:} \emph{Includes a detailed description of your ideal learning environment, referencing scholarly sources and analyzing key elements of an authentic learning community. Describes the learning community environment, student learning outcomes, learning outcomes and personal beliefs about teaching and learning.}

\textbf{Exemplary:} Includes a detailed description of your ideal learning environment, referencing scholarly sources and analyzing key elements of authentic learning communities. Provides a rationale for key elements of the learning community environment, student learning outcomes, learning outcomes and personal beliefs about teaching and learning. Advances scholarly thinking about authentic learning communities.

\hypertarget{your-role-as-teacher-and-leaders}{%
\subsubsection*{YOUR ROLE AS TEACHER AND LEADERS}\label{your-role-as-teacher-and-leaders}}
\addcontentsline{toc}{subsubsection}{YOUR ROLE AS TEACHER AND LEADERS}

\textbf{Unsatisfactory:} Does not include a description of your role as a teacher or leader within an authentic learning community, incorporating scholarly literature. Does not include an analysis of your role as a facilitator/leader in planning learning experiences, facilitating student learning, and assessing student learning. Does not include a description of the actions, practices, and strategies you will engage in to achieve your vision of the learning community you described in section one.

\textbf{Developing:} Includes a partial description of your role as a teacher or leader within an authentic learning community, incorporating scholarly literature. Describes but does not analyze your role as a facilitator/leader in planning learning experiences, facilitating student learning, and assessing student learning. Lists but does not describe the actions, practices, and strategies you will engage in to achieve your vision of the learning community you described in section one.

\textbf{Proficient:} \emph{Includes a detailed description of your role as a teacher or leader within an authentic learning community, incorporating scholarly literature. Includes a detailed analysis of your role as a facilitator/leader in planning learning experiences, facilitating student learning, and assessing student learning. Includes a detailed description of the actions, practices, and strategies you will engage in to achieve your vision of the learning community you described in section one.}

\textbf{Exemplary:} Includes a detailed analysis of your role as a teacher or leader within an authentic learning community, incorporating scholarly literature. Includes a detailed analysis of your role as a facilitator/leader in planning learning experiences, facilitating student learning, and assessing student learning. Includes a detailed description of the actions, practices, and strategies you will engage in to achieve your vision of the learning community you described in section one. Synthesizes scholarly thinking about the role of the teacher/leader.

\hypertarget{scholarly-integration-2}{%
\subsubsection*{SCHOLARLY INTEGRATION}\label{scholarly-integration-2}}
\addcontentsline{toc}{subsubsection}{SCHOLARLY INTEGRATION}

\textbf{Unsatisfactory:} Does not integrate many references to support the arguments made in the paper.

\textbf{Developing:} Integrates fewer than 10 scholarly sources to support arguments made in the paper.

\textbf{Proficient:} \emph{Integrates a minimum of 10 scholarly sources to support arguments made in each section of the paper.}

\textbf{Exemplary:} Integrates a minimum of 10 references to support the arguments made in each section, including several scholarly sources not included in course materials.

\begin{longtable}[]{@{}
  >{\raggedright\arraybackslash}p{(\columnwidth - 8\tabcolsep) * \real{0.2000}}
  >{\raggedright\arraybackslash}p{(\columnwidth - 8\tabcolsep) * \real{0.2000}}
  >{\raggedright\arraybackslash}p{(\columnwidth - 8\tabcolsep) * \real{0.2000}}
  >{\raggedright\arraybackslash}p{(\columnwidth - 8\tabcolsep) * \real{0.2000}}
  >{\raggedright\arraybackslash}p{(\columnwidth - 8\tabcolsep) * \real{0.2000}}@{}}
\toprule\noalign{}
\begin{minipage}[b]{\linewidth}\raggedright
\textbf{TOTAL}
\end{minipage} & \begin{minipage}[b]{\linewidth}\raggedright
\textbf{0 = 0\% (F)}
\end{minipage} & \begin{minipage}[b]{\linewidth}\raggedright
\textbf{10 = 50\% (C)}
\end{minipage} & \begin{minipage}[b]{\linewidth}\raggedright
\textbf{15 = 75 (B)}
\end{minipage} & \begin{minipage}[b]{\linewidth}\raggedright
\textbf{20 = 100\% (A+)}
\end{minipage} \\
\midrule\noalign{}
\endhead
\bottomrule\noalign{}
\endlastfoot
\end{longtable}

\hypertarget{references}{%
\chapter*{References}\label{references}}
\addcontentsline{toc}{chapter}{References}

The following are key references used in this course. \textbf{\emph{Check with your course syllabus for required readings.}}

  \bibliography{book.bib}

\end{document}
