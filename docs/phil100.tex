% Options for packages loaded elsewhere
\PassOptionsToPackage{unicode}{hyperref}
\PassOptionsToPackage{hyphens}{url}
%
\documentclass[
]{book}
\usepackage{amsmath,amssymb}
\usepackage{iftex}
\ifPDFTeX
  \usepackage[T1]{fontenc}
  \usepackage[utf8]{inputenc}
  \usepackage{textcomp} % provide euro and other symbols
\else % if luatex or xetex
  \usepackage{unicode-math} % this also loads fontspec
  \defaultfontfeatures{Scale=MatchLowercase}
  \defaultfontfeatures[\rmfamily]{Ligatures=TeX,Scale=1}
\fi
\usepackage{lmodern}
\ifPDFTeX\else
  % xetex/luatex font selection
\fi
% Use upquote if available, for straight quotes in verbatim environments
\IfFileExists{upquote.sty}{\usepackage{upquote}}{}
\IfFileExists{microtype.sty}{% use microtype if available
  \usepackage[]{microtype}
  \UseMicrotypeSet[protrusion]{basicmath} % disable protrusion for tt fonts
}{}
\makeatletter
\@ifundefined{KOMAClassName}{% if non-KOMA class
  \IfFileExists{parskip.sty}{%
    \usepackage{parskip}
  }{% else
    \setlength{\parindent}{0pt}
    \setlength{\parskip}{6pt plus 2pt minus 1pt}}
}{% if KOMA class
  \KOMAoptions{parskip=half}}
\makeatother
\usepackage{xcolor}
\usepackage{longtable,booktabs,array}
\usepackage{calc} % for calculating minipage widths
% Correct order of tables after \paragraph or \subparagraph
\usepackage{etoolbox}
\makeatletter
\patchcmd\longtable{\par}{\if@noskipsec\mbox{}\fi\par}{}{}
\makeatother
% Allow footnotes in longtable head/foot
\IfFileExists{footnotehyper.sty}{\usepackage{footnotehyper}}{\usepackage{footnote}}
\makesavenoteenv{longtable}
\usepackage{graphicx}
\makeatletter
\def\maxwidth{\ifdim\Gin@nat@width>\linewidth\linewidth\else\Gin@nat@width\fi}
\def\maxheight{\ifdim\Gin@nat@height>\textheight\textheight\else\Gin@nat@height\fi}
\makeatother
% Scale images if necessary, so that they will not overflow the page
% margins by default, and it is still possible to overwrite the defaults
% using explicit options in \includegraphics[width, height, ...]{}
\setkeys{Gin}{width=\maxwidth,height=\maxheight,keepaspectratio}
% Set default figure placement to htbp
\makeatletter
\def\fps@figure{htbp}
\makeatother
\setlength{\emergencystretch}{3em} % prevent overfull lines
\providecommand{\tightlist}{%
  \setlength{\itemsep}{0pt}\setlength{\parskip}{0pt}}
\setcounter{secnumdepth}{5}
\usepackage{booktabs}
\usepackage{amsthm}
\makeatletter
\def\thm@space@setup{%
  \thm@preskip=8pt plus 2pt minus 4pt
  \thm@postskip=\thm@preskip
}
\makeatother

\usepackage{tcolorbox}


\newtcolorbox{blackbox}{
  colback=black,
  coltext=white,
  colframe=black,
  boxsep=5pt,
  arc=4pt}
\newtcolorbox{bonus}{
  colback=blue!15,
  colframe=blue!15,
  coltext=black!80,
  boxsep=5pt,
  arc=4pt}
\newtcolorbox{reflect}{
  colback=green!5,
  colframe=green!5,
  coltext=black!80,
  boxsep=5pt,
  arc=4pt}
\newtcolorbox{assessment}{
  colback=blue!5,
  colframe=blue!5,
  coltext=black!80,
  boxsep=5pt,
  arc=4pt}
\newtcolorbox{progress}{
  colback=purple!10,
  colframe=purple!10,
  coltext=black!80,
  boxsep=5pt,
  arc=4pt}
\newtcolorbox{video}{
  colback=yellow!5,
  colframe=yellow!5,
  coltext=black!80,
  boxsep=5pt,
  arc=4pt}
\newtcolorbox{caution}{
  colback=red!5,
  colframe=red!5,
  coltext=black!80,
  boxsep=5pt,
  arc=4pt}
\newtcolorbox{feedback}{
  colback=black!5,
  colframe=black!5,
  coltext=black!80,
  boxsep=5pt,
  arc=4pt}
\ifLuaTeX
  \usepackage{selnolig}  % disable illegal ligatures
\fi
\usepackage[]{natbib}
\bibliographystyle{apalike}
\IfFileExists{bookmark.sty}{\usepackage{bookmark}}{\usepackage{hyperref}}
\IfFileExists{xurl.sty}{\usepackage{xurl}}{} % add URL line breaks if available
\urlstyle{same}
\hypersetup{
  pdftitle={Philosophy for Life},
  hidelinks,
  pdfcreator={LaTeX via pandoc}}

\title{Philosophy for Life}
\author{}
\date{\vspace{-2.5em}}

\begin{document}
\maketitle

{
\setcounter{tocdepth}{1}
\tableofcontents
}
\hypertarget{welcome}{%
\chapter*{Welcome}\label{welcome}}
\addcontentsline{toc}{chapter}{Welcome}

This is the course book for Philosophy for Life. This book is divided into thematic units of study to help you engage with the materials. The course resources and learning activities are designed not only to help prepare you for the course assessments, but also to give you opportunities to practice various skills.

\begin{quote}
Below you will find information about how to navigate this book. Please read the full course syllabus located on the Course Home page in Moodle. It includes key information about the course schedule, assignments, and policies.
\end{quote}

\hypertarget{course-notes}{%
\section*{Course Notes}\label{course-notes}}
\addcontentsline{toc}{section}{Course Notes}

You should be reading this information in the context of a Trinity Western University course offered via Moodle. If this is not the case, then this may be an unauthorized reproduction of the course. Please contact \href{mailto:elearning@twu.ca}{\nolinkurl{elearning@twu.ca}} if you have concerns.

These notes will be your guide through the learning activities and assessment strategies necessary for you to succeed in the course, so it is important for you to engage to the best of your ability and take advantage of the resources available to you through Trinity Western University.

\begin{quote}
Assessment tasks are managed in other sections of the Moodle course, so be sure to familiarize yourself with those requirements and resources.
\end{quote}

\hypertarget{how-to-navigate-this-book}{%
\section*{How To Navigate This Book}\label{how-to-navigate-this-book}}
\addcontentsline{toc}{section}{How To Navigate This Book}

To move quickly to different portions of the book, click on the appropriate chapter or section in the table of contents on the left. The buttons at the top of the page allow you to show/hide the table of contents, search the book, change font settings, download a pdf or ebook copy of this book, or get hints on various sections of the book.

\includegraphics{assets/course-intro/menu.png}

The faint left and right arrows at the sides of each page (or bottom of the page if it's narrow enough) allow you to step to the next/previous section. Here's what they look like:

\includegraphics{assets/course-intro/left_arrow.png} \includegraphics{assets/course-intro/right_arrow.png}

You can also download an offline copy of this book in various formats, such as pdf or an ebook. If you are having any accessibility or navigation issues with this book, please reach out to your instructor or our online team at \href{mailto:elearning@twu.ca}{\nolinkurl{elearning@twu.ca}}.

\hypertarget{course-units}{%
\subsection*{Course Units}\label{course-units}}
\addcontentsline{toc}{subsection}{Course Units}

This course is organized into thematic units. Each unit of the course will provide you with the following information:

\begin{itemize}
\tightlist
\item
  A general overview of the key concepts that will be addressed during the unit.\\
\item
  Specific learning outcomes and topics for the unit.\\
\item
  Learning activities to help you engage with the concepts. These often include key readings, videos, and reflective prompts.\\
\item
  The Assessment section provides details on assignments you will need to complete throughout the course to demonstrate your understanding of the course learning outcomes.
\end{itemize}

\begin{quote}
Note that assessments, including assignments and discussion posts will be submitted in Moodle. See the Assessment section(s) in Moodle for full assignment details.
\end{quote}

\hypertarget{course-activities}{%
\subsection*{Course Activities}\label{course-activities}}
\addcontentsline{toc}{subsection}{Course Activities}

Below is some key information on features you will see throughout the course.

\begin{reflect}
\textbf{\emph{Learning Activity}}

This box will prompt you to engage in course concepts, often by viewing resources and reflecting on your experience and/or learning. Most learning activities are ungraded and are designed to help prepare you for the assessment in this course.
\end{reflect}

\begin{assessment}
\textbf{\emph{Assessment}}

This box will signify an assignment or discussion post you will submit in Moodle. Note that these demonstrate your understanding of the course learning outcomes. Be sure to review the grading rubrics for each assignment.
\end{assessment}

\begin{progress}
\textbf{\emph{Checking Your Learning}}

This box is for checking your understanding, to make sure you are ready for what follows. Ways to check your learning might include self-check quizzes or questions for discussion. These activities are not graded but are critical for you to be able to begin to develop evaluative judgement in this domain of knowledge.
\end{progress}

\begin{caution}
\textbf{\emph{Note}}

This box signifies key notes. It may also warn you of possible problems or pitfalls you may encounter!
\end{caution}

If you have any questions, do not hesitate to ask. We are here to help and be your guide on this journey.

\hypertarget{wisdom}{%
\chapter{Wisdom}\label{wisdom}}

This video briefly introduces the four topics in the first unit: knowledge, humility, friendship, and rhetoric. While we will learn some basic ideas about the nature of wisdom in the first topic on knowledge, most of this unit is devoted to introducing some of the skills of living a life of wisdom and how we can begin to practice these skills.

\textbf{Wisdom Unit 1 Introduction} Video (6 min 10 sec)

\hypertarget{overview}{%
\section*{Overview}\label{overview}}
\addcontentsline{toc}{section}{Overview}

\hypertarget{topics}{%
\subsection*{Topics}\label{topics}}
\addcontentsline{toc}{subsection}{Topics}

This unit is divided into the following four topics:

\begin{enumerate}
\def\labelenumi{\arabic{enumi}.}
\tightlist
\item
  Wisdom and Knowledge
\item
  Wisdom and Humility
\item
  Wisdom and Friendship
\item
  Wisdom and Rhetoric
\end{enumerate}

\hypertarget{learning-outcomes}{%
\subsection*{Learning Outcomes}\label{learning-outcomes}}
\addcontentsline{toc}{subsection}{Learning Outcomes}

When you've completed this unit, you will have learned how to:

\begin{itemize}
\tightlist
\item
  Identify and understand the role wisdom plays in knowledge, humiloity, friendship and rhetoric
\item
  Practice and apply the skills of wisdom to knowledge, humility, friendship and rhetoric
\item
  Apply these skills to the types of actions a wise person may take
\item
  Enhance your skills of wisdom and reason to live life within the context of justice and faith
\item
  Develop the skills to become a well-rounded person and citizen and live a wise and just life
\end{itemize}

\hypertarget{wisdom-and-knowledge}{%
\section*{1.1 Wisdom and Knowledge}\label{wisdom-and-knowledge}}
\addcontentsline{toc}{section}{1.1 Wisdom and Knowledge}

Follow and complete the steps below to accomplish your learning for this topic.

\begin{enumerate}
\def\labelenumi{\arabic{enumi}.}
\tightlist
\item
  Read Topic Notes
\item
  Watch Topic Video
\item
  Read Topic Reading
\item
  Complete Topic Exercise
\item
  Complete Topic Questionnaire
\item
  Watch Optional Video
\end{enumerate}

\hypertarget{topic-notes}{%
\subsection*{Topic Notes}\label{topic-notes}}
\addcontentsline{toc}{subsection}{Topic Notes}

In this topic, we learn about the nature of wisdom and its relationship with knowledge. Wisdom is not just about knowledge. Wisdom is practical in that wisdom is required to live life well, to cope with and respond to the many challenges of life, and, if possible, avoid some of those challenges. While wisdom is practical, it does require at least some knowledge. For living life well, wisdom requires knowing the right beliefs and knowing how to apply these beliefs in the right way at the right time. How do we know which beliefs and actions are appropriate for living life well? Philosopher Robert Nozick argues that a wise person holds beliefs about many aspects of life, including beliefs about life's most important values, beliefs about achieving one's goals, and following the necessary steps to achieve these goals. Likewise, the right actions for living a fulfilling life are derived from applying the aforementioned beliefs. If a person knows about the most important values and goals in life, but never applies them to their behavior, then they are not considered wise. In other words, having knowledge does not entail wisdom. Nozick is also aware that applying the appropriate beliefs and actions is not sufficient to live life well. Sometimes bad things occur for no apparent reason, which may undermine our efforts. A wise person is aware of this problem, and while pursuing their goals, prepares for failure, learns how to avoid it, and how to respond to it if failure occurs.

\hypertarget{watch-and-reflect}{%
\subsection*{Watch and Reflect}\label{watch-and-reflect}}
\addcontentsline{toc}{subsection}{Watch and Reflect}

In this video, you will learn about wisdom and knowledge from Robert Nozick's piece, ``\emph{What is Wisdom and Why do Philosophers Love it So?}''

\textbf{Wisdom and Knowledge Unit 1 Topic 1} Video (11 min 15 sec)

\hypertarget{read-and-reflect}{%
\subsection*{Read and Reflect}\label{read-and-reflect}}
\addcontentsline{toc}{subsection}{Read and Reflect}

\begin{itemize}
\tightlist
\item
  Robert Nozick - \href{assets/u1/PHIL-100-Nozick-What-is-Wisdom.pdf}{\emph{What is Wisdom and Why Do Philosophers Love It So?}}
\end{itemize}

\hypertarget{required-nozick-reading-note-taking-exercise}{%
\subsection*{Required Nozick Reading Note-Taking Exercise}\label{required-nozick-reading-note-taking-exercise}}
\addcontentsline{toc}{subsection}{Required Nozick Reading Note-Taking Exercise}

\begin{reflect}
\begin{itemize}
\tightlist
\item
  Answering these questions will help you understand the Nozick reading and prepare you for your Unit 1 Reflection Assignment.
\item
  Click the Download button on the left after you've completed all the notes for this reading and save the answers to your computer.
\item
  You will be required to submit your downloaded notes to this reading as part of your Unit 1 Reflection Assignment.
\end{itemize}
\end{reflect}

\hypertarget{required-note-taking-questionnaire}{%
\subsection*{Required Note-Taking Questionnaire}\label{required-note-taking-questionnaire}}
\addcontentsline{toc}{subsection}{Required Note-Taking Questionnaire}

\begin{reflect}
\begin{itemize}
\tightlist
\item
  Answering these questions will help you reflect on the topic content and prepare you for your Unit 1 Reflection Assignment.
\item
  Click the Download button on the left after you've completed all the questions for this unit and save the answers to your computer.
\item
  You will be required to submit the downloaded answers to these questions as part of your Unit 1 Reflection Assignment.
\end{itemize}
\end{reflect}

\hypertarget{watch-and-reflect-1}{%
\subsection*{Watch and Reflect}\label{watch-and-reflect-1}}
\addcontentsline{toc}{subsection}{Watch and Reflect}

\begin{reflect}
In this 5 minute 55 second optional video, philosophers Valerie Tiberius and Philip Kitcher, as well as psychologist Lisa Feldman Barret discuss their definitions of wisdom.

\href{https://youtu.be/obqedyeUcwk}{Watch: \emph{What is Wisdom?}}
\end{reflect}

\hypertarget{wisdom-and-humility}{%
\section*{1.2 Wisdom and Humility}\label{wisdom-and-humility}}
\addcontentsline{toc}{section}{1.2 Wisdom and Humility}

Follow and complete the steps below to accomplish your learning for this topic.

\begin{enumerate}
\def\labelenumi{\arabic{enumi}.}
\tightlist
\item
  Read Topic Notes
\item
  Watch Topic Video
\item
  Read Topic Reading
\item
  Complete Topic Exercise
\item
  Complete Topic Questionnaire
\item
  Watch Optional Video
\end{enumerate}

\hypertarget{topic-notes-1}{%
\subsection*{Topic Notes}\label{topic-notes-1}}
\addcontentsline{toc}{subsection}{Topic Notes}

In this topic, we focus on wisdom and humility. A wise person understands their limitations about what they know and do not know. In other words, a wise person is aware of their ignorance. The word ``ignorance'' in this case does not have a negative connotation, but rather describes a lack of knowledge about a particular thing.

In ``the Apology,'' Plato introduces the character Socrates, who is on trial for his life. During his defense, Socrates describes the account of his friend, Chaerephon, seeking answers from the Oracle of Delphi, wondering if there is any person wiser than Socrates. In reply to Chaerephon, the oracle says that no one is wiser than Socrates. Socrates is puzzled by the revelation about his wisdom since he considers himself to be unwise. Socrates then asks questions of different people in the city, such as politicians, poets, and artisans, in an attempt to solve the mystery about the oracle's revelation. Socrates discovers that while these individuals claim to know a great deal about issues such as the nature of beauty and what is good, he observes that they know very little about such issues. These people think they possess knowledge, when in fact they do not, and that is why they lack wisdom. Their way of thinking contrasts with Socrates, who thinks he lacks knowledge about these issues; more specifically, Socrates is aware of his ignorance about the nature of beauty and what is good, while the others are unaware of their own ignorance about these same issues. Because of this, Socrates is the wisest person.

One application of being aware of ignorance for seeking wisdom is to practice a humble attitude towards the complex issues in life, such as how to interpret the Bible, evolution, the nature of sexuality, the implications of social media for our lives, and was Donald Trump an effective president? These types of questions represent complex ideas that even some of the most specialized scholars cannot answer. A wise person demonstrates their humility towards these complex issues by asking thoughtful questions about the opposing view, knowing when to withhold judgment and opinion, and with a willingness to change their mind.

\hypertarget{watch-and-reflect-2}{%
\subsection*{Watch and Reflect}\label{watch-and-reflect-2}}
\addcontentsline{toc}{subsection}{Watch and Reflect}

In this video, you will learn about wisdom and humility from Plato's \emph{Apology}.

\textbf{Wisdom and Humility Unit 1 Topic 2} Video (13 min 54 sec)

\hypertarget{read-and-reflect-1}{%
\subsection*{Read and Reflect}\label{read-and-reflect-1}}
\addcontentsline{toc}{subsection}{Read and Reflect}

\begin{itemize}
\tightlist
\item
  Plato - \emph{Apology} - \url{https://classics.mit.edu/Plato/apology.html}
\end{itemize}

\hypertarget{required-plato-reading-note-taking-exercise}{%
\subsection*{Required Plato Reading Note-Taking Exercise}\label{required-plato-reading-note-taking-exercise}}
\addcontentsline{toc}{subsection}{Required Plato Reading Note-Taking Exercise}

\begin{reflect}
\begin{itemize}
\tightlist
\item
  Answering these questions will help you understand the Plato reading and prepare you for your Unit 1 Reflection Assignment.
\item
  Click the Download button on the left after you've completed all the notes for this reading and save the answers to your computer.
\item
  You will be required to submit your downloaded notes to this reading as part of your Unit 1 Reflection Assignment.
\end{itemize}
\end{reflect}

\hypertarget{required-note-taking-questionnaire-1}{%
\subsection*{Required Note-Taking Questionnaire}\label{required-note-taking-questionnaire-1}}
\addcontentsline{toc}{subsection}{Required Note-Taking Questionnaire}

\begin{reflect}
\begin{itemize}
\tightlist
\item
  Answering these questions will help you reflect on the topic content and prepare you for your Unit 1 Reflection Assignment.
\item
  Click the Download button on the left after you've completed all the questions for this unit and save the answers to your computer.
\item
  You will be required to submit the downloaded answers to these questions as part of your Unit 1 Reflection Assignment.
\end{itemize}
\end{reflect}

\hypertarget{watch-and-reflect-3}{%
\subsection*{Watch and Reflect}\label{watch-and-reflect-3}}
\addcontentsline{toc}{subsection}{Watch and Reflect}

\begin{reflect}
In this 2 minute 59 second optional video with Dr.~Ian Church, \emph{What is Intellectual Humility}? A doxastic account. Here is the first of three parts. If you have the time, you should watch all three parts.

\href{https://www.youtube.com/watch?v=8CZIkGEJYRY}{Watch: \emph{What is Intellectual Humility?}}
\end{reflect}

\hypertarget{wisdom-and-friendship}{%
\section*{1.3 Wisdom and Friendship}\label{wisdom-and-friendship}}
\addcontentsline{toc}{section}{1.3 Wisdom and Friendship}

Follow and complete the steps below to accomplish your learning for this topic.

\begin{enumerate}
\def\labelenumi{\arabic{enumi}.}
\tightlist
\item
  Read Topic Notes
\item
  Watch Topic Video
\item
  Read Topic Reading
\item
  Complete Topic Exercise
\item
  Complete Topic Questionnaire
\item
  Watch Optional Video
\end{enumerate}

\hypertarget{topic-notes-2}{%
\subsection*{Topic Notes}\label{topic-notes-2}}
\addcontentsline{toc}{subsection}{Topic Notes}

This topic examines wisdom and friendship. A wise person is careful about how they view their friends. In doing so, it is best to learn about the various types of friendships. According to Aristotle, there are three types of friendships, friends of utility, pleasure, and virtue. Friends of utility are friends who are a means to an end, such as business partners, or some friends on a sports team. Friends of pleasure are friends with whom we share pleasurable activities. Friends of virtue are friends who exhibit good qualities and help us develop these qualities as well. Aristotle believes that the third type of friend is the best type of friend, and this type of friendship, while difficult to find and develop, is worth pursuing and fighting for. This is because friends of virtue care about us and our character development beyond the scope of utility and pleasure. They care about us even when things are difficult. But how do we find these friends and how can we be virtuous friends to other people? Thomas Aquinas provides some advice. Aquinas suggests that we should be careful about the bad qualities that make for a poor friend, such as envy and pride. These vices (or bad characteristics) undermine the pursuit of virtue and thus compromise any friendship that is built upon virtue. An envious person is someone who is upset about our own success. Think of a friend you have. Have they ever been upset or sad when good things happen to you? If the answer is yes, then this person may not be a virtuous friend. They may suffer from envy towards you. Or have you ever had a friend who jokes about you but can never receive the joke in return? This person may not be a virtuous friend because they suffer from arrogant pride. They think too highly of themselves. Furthermore, we should be cautious about whether we are exhibiting these defects as well. Do we treat our friends as a means to a utilitarian end? Do we make friends only for pleasure? Do we feel upset about our friends success? Do we exhibit a sense of arrogant pride about ourselves? This video introduces these ideas from Aristotle and Aquinas to help better equip us with the skills for developing healthy friendship for living well.

\hypertarget{watch-and-reflect-4}{%
\subsection*{Watch and Reflect}\label{watch-and-reflect-4}}
\addcontentsline{toc}{subsection}{Watch and Reflect}

In this video, you will learn about wisdom and friendship from Aristotle's \emph{Nicomachean Ethics}, Books 8, 9 and 10, and some additional points from Thomas Aquinas.

\textbf{Wisdom and Friendship Unit 1 Topic 3} Video (24 min 18 sec)

\hypertarget{read-and-reflect-2}{%
\subsection*{Read and Reflect}\label{read-and-reflect-2}}
\addcontentsline{toc}{subsection}{Read and Reflect}

\begin{itemize}
\tightlist
\item
  Aristotle - \href{assets/u1/PHIL-100-Aristotle-NE-VIII-IX-X.pdf}{\emph{Nicomachean Ethics}, Books VIII, IX and X}
\end{itemize}

\hypertarget{required-aristotle-reading-note-taking-exercise}{%
\subsection*{Required Aristotle Reading Note-Taking Exercise}\label{required-aristotle-reading-note-taking-exercise}}
\addcontentsline{toc}{subsection}{Required Aristotle Reading Note-Taking Exercise}

\begin{reflect}
\begin{itemize}
\tightlist
\item
  Answering these questions will help you understand the Aristotle reading and prepare you for your Unit 1 Reflection Assignment.
\item
  Click the Download button on the left after you've completed all the notes for this reading and save the answers to your computer.
\item
  You will be required to submit your downloaded notes to this reading as part of your Unit 1 Reflection Assignment.
\end{itemize}
\end{reflect}

\hypertarget{required-note-taking-questionnaire-2}{%
\subsection*{Required Note-Taking Questionnaire}\label{required-note-taking-questionnaire-2}}
\addcontentsline{toc}{subsection}{Required Note-Taking Questionnaire}

\begin{reflect}
\begin{itemize}
\tightlist
\item
  Answering these questions will help you reflect on the topic content and prepare you for your Unit 1 Reflection Assignment.
\item
  Click the Download button on the left after you've completed all the questions for this unit and save the answers to your computer.
\item
  You will be required to submit the downloaded answers to these questions as part of your Unit 1 Reflection Assignment.
\end{itemize}
\end{reflect}

\hypertarget{watch-and-reflect-5}{%
\subsection*{Watch and Reflect}\label{watch-and-reflect-5}}
\addcontentsline{toc}{subsection}{Watch and Reflect}

\begin{reflect}
In this 7 minute 21 second optional video, you will learn further details about Aristotle's view of friendship.

\href{https://www.youtube.com/watch?v=F18kSA8OxqY}{Watch: \emph{Aristotle's Timeless Advice on What Real Friendship Is and Why It Matters}}
\end{reflect}

\hypertarget{wisdom-and-rhetoric}{%
\section*{1.4 Wisdom and Rhetoric}\label{wisdom-and-rhetoric}}
\addcontentsline{toc}{section}{1.4 Wisdom and Rhetoric}

Follow and complete the steps below to accomplish your learning for this topic.

\begin{enumerate}
\def\labelenumi{\arabic{enumi}.}
\tightlist
\item
  Read Topic Notes
\item
  Watch Topic Video
\item
  Read Topic Reading
\item
  Complete Topic Exercise
\item
  Complete Topic Questionnaire
\item
  Watch Optional Video
\end{enumerate}

\hypertarget{topic-notes-3}{%
\subsection*{Topic Notes}\label{topic-notes-3}}
\addcontentsline{toc}{subsection}{Topic Notes}

This topic examines wisdom and rhetoric. Arguments with other people, such as family, friends, or colleagues, are part of everyday life. A wise person knows how to argue without fighting. According to Jay Heinrichs, a necessary skill for arguing productively is knowing the appropriate goal you wish to bring about. Are you trying to influence someone's decision? Is that feasible? Are you hoping to feel like a winner? Is that an appropriate goal? It may be your goal to change the mood of your opponent, or simply to inform your opponent of a different point of view. Would you like to end the argument quickly? After the argument, would you like to maintain the relationship with the person? Should you pursue these goals? How will you go about accomplishing your goal? In what way are you prepared to compromise? Are you prepared to experience defeat from your opponent's point of view in order to achieve your goal? What is the purpose of the argument and how should it be achieved?

\hypertarget{watch-and-reflect-6}{%
\subsection*{Watch and Reflect}\label{watch-and-reflect-6}}
\addcontentsline{toc}{subsection}{Watch and Reflect}

In this video, you will learn about wisdom and rhetoric from Jay Heinrich's \emph{Thank You for Arguing}, Chapter 2 ``Set Your Goals''.

\textbf{Wisdom and Rhetoric Unit 1 Topic 4} Video (12 min 19 sec)

\hypertarget{read-and-reflect-3}{%
\subsection*{Read and Reflect}\label{read-and-reflect-3}}
\addcontentsline{toc}{subsection}{Read and Reflect}

\begin{itemize}
\tightlist
\item
  Jay Heinrichs - \href{assets/u1/PHIL-100-Heinrichs-Thank-You-for-Arguing.pdf}{\emph{Thank You for Arguing}}
\end{itemize}

\hypertarget{required-heinrichs-reading-note-taking-exercise}{%
\subsection*{Required Heinrichs Reading Note-Taking Exercise}\label{required-heinrichs-reading-note-taking-exercise}}
\addcontentsline{toc}{subsection}{Required Heinrichs Reading Note-Taking Exercise}

\begin{reflect}
\begin{itemize}
\tightlist
\item
  Answering these questions will help you understand the Heinrichs reading and prepare you for your Unit 1 Reflection Assignment.
\item
  Click the Download button on the left after you've completed all the notes for this reading and save the answers to your computer.
\item
  You will be required to submit your downloaded notes to this reading as part of your Unit 1 Reflection Assignment.
\end{itemize}
\end{reflect}

\hypertarget{required-note-taking-questionnaire-3}{%
\subsection*{Required Note-Taking Questionnaire}\label{required-note-taking-questionnaire-3}}
\addcontentsline{toc}{subsection}{Required Note-Taking Questionnaire}

\begin{reflect}
\begin{itemize}
\tightlist
\item
  Answering these questions will help you reflect on the topic content and prepare you for your Unit 1 Reflection Assignment.
\item
  Click the Download button on the left after you've completed all the questions for this unit and save the answers to your computer.
\item
  You will be required to submit the downloaded answers to these questions as part of your Unit 1 Reflection Assignment.
\end{itemize}
\end{reflect}

\hypertarget{watch-and-reflect-7}{%
\subsection*{Watch and Reflect}\label{watch-and-reflect-7}}
\addcontentsline{toc}{subsection}{Watch and Reflect}

\begin{reflect}
In this 4 minute 29 second optional video. Here, Camille A. Langston, a specialist in rhetoric and nineteenth-century women, explains the basics of deliberative rhetoric and shares tips for appealing to the ethos, logos, and pathos of your audience.

\href{https://www.youtube.com/watch?v=3klMM9BkW5o}{Watch: \emph{How to Use Rhetoric to Get What You Want}}
\end{reflect}

\hypertarget{faith}{%
\chapter{Faith}\label{faith}}

This video briefly introduces the four topics in the second unit: definitions, hope, action, and rationality. We will learn about the nature of faith, both in terms of the attitudes of faith, such as beliefs and trust, and also the actions of faith, focusing briefly on ethics. Finally, we will learn about the rationality of faith and whether faith aligns with evidence.

\textbf{Faith Unit 2 Introduction} Video (7 min 34 sec)

\hypertarget{overview-1}{%
\section*{Overview}\label{overview-1}}
\addcontentsline{toc}{section}{Overview}

\hypertarget{topics-1}{%
\subsection*{Topics}\label{topics-1}}
\addcontentsline{toc}{subsection}{Topics}

This unit is divided into the following four topics:

\begin{enumerate}
\def\labelenumi{\arabic{enumi}.}
\tightlist
\item
  Faith and Definitions
\item
  Faith and Hope
\item
  Faith and Action
\item
  Faith and Rationality
\end{enumerate}

\hypertarget{learning-outcomes-1}{%
\subsection*{Learning Outcomes}\label{learning-outcomes-1}}
\addcontentsline{toc}{subsection}{Learning Outcomes}

When you've completed this unit, you will have learned how to:

\begin{itemize}
\tightlist
\item
  Identify and understand the nature of faith, including definitions, ethical implications, and its relationship to rationality
\item
  Develop a greater understanding of the ethical behavior required by faith
\item
  Identify briefly how Christianity applies to living life within the context of justice and faith
\item
  Enhance a greater understanding about faith and rationality
\item
  Navigate some of the objections to the nature of faith, particularly the beliefs about faith
\end{itemize}

\hypertarget{faith-and-definitions}{%
\section*{2.1 Faith and Definitions}\label{faith-and-definitions}}
\addcontentsline{toc}{section}{2.1 Faith and Definitions}

Follow and complete the steps below to accomplish your learning for this topic.

\begin{enumerate}
\def\labelenumi{\arabic{enumi}.}
\tightlist
\item
  Read Topic Notes
\item
  Watch Topic Video
\item
  Read Topic Reading
\item
  Complete Topic Exercise
\item
  Complete Topic Questionnaire
\item
  Watch Optional Video
\end{enumerate}

\hypertarget{topic-notes-4}{%
\subsection*{Topic Notes}\label{topic-notes-4}}
\addcontentsline{toc}{subsection}{Topic Notes}

In this topic, we learn about several definitions of the term faith. One distinction to make about faith is between faith as an attitude and faith as an action. Faith as an attitude means that faith is something in our heads. Faith is a mental and cognitive experience, such as a belief, trust, and hope. These ideas are a common way to think about faith. On the other hand, faith as an action refers to an ethical way of thinking about the term. The Greek word for faith is ``pistis'' and this can refer ethical behavior towards a person or ideal. With regard to faith as an attitude, we will learn about the difference between doxastic faith and non-doxastic faith. The term ``doxastic'' refers to belief, and the term ``Non-doxastic'' refers to faith without belief.

\hypertarget{watch-and-reflect-8}{%
\subsection*{Watch and Reflect}\label{watch-and-reflect-8}}
\addcontentsline{toc}{subsection}{Watch and Reflect}

In this video, you will learn about the various definitions of faith from Elizabeth Jackson's article, ``Faith: Contemporary Perspectives''. The main focus is on learning about various distinctions with the concept of faith, such as attitude faith and action faith, and doxastic and non-doxastic view of faith.

\textbf{Faith and Definitions Unit 2 Topic 1} Video (8 min 01 sec)

Read and Reflect

\begin{itemize}
\tightlist
\item
  Jackson - \emph{Types of Faith} - \url{https://iep.utm.edu/faith-contemporary-perspectives/\#SSH1ai\%7Btarget=\%22_blank}``\}
\end{itemize}

\hypertarget{required-jackson-reading-note-taking-exercise}{%
\subsection*{Required Jackson Reading Note-Taking Exercise}\label{required-jackson-reading-note-taking-exercise}}
\addcontentsline{toc}{subsection}{Required Jackson Reading Note-Taking Exercise}

\begin{reflect}
\begin{itemize}
\tightlist
\item
  Answering these questions will help you understand the Jackson reading and prepare you for your Unit 2 Reflection Assignment.
\item
  Click the Download button on the left after you've completed all the notes for this reading and save the answers to your computer.
\item
  You will be required to submit your downloaded notes to this reading as part of your Unit 2 Reflection Assignment.
\end{itemize}
\end{reflect}

\hypertarget{required-note-taking-questionnaire-4}{%
\subsection*{Required Note-Taking Questionnaire}\label{required-note-taking-questionnaire-4}}
\addcontentsline{toc}{subsection}{Required Note-Taking Questionnaire}

\begin{reflect}
\begin{itemize}
\tightlist
\item
  Answering these questions will help you reflect on the topic content and prepare you for your Unit 2 Reflection Assignment.
\item
  Click the Download button on the left after you've completed all the questions for this unit and save the answers to your computer.
\item
  You will be required to submit the downloaded answers to these questions as part of your Unit 2 Reflection Assignment.
\end{itemize}
\end{reflect}

\hypertarget{watch-and-reflect-9}{%
\subsection*{Watch and Reflect}\label{watch-and-reflect-9}}
\addcontentsline{toc}{subsection}{Watch and Reflect}

\begin{reflect}
In this 10 minute 06 second optional video, you will learn some further distinctions between the terms faith and belief, especially in religious contexts.

\href{https://www.youtube.com/watch?v=o8QKkHWUSu0}{Watch: \emph{Belief vs Faith (Philosophical Distinction)}}
\end{reflect}

\hypertarget{faith-and-hope}{%
\section*{2.2 Faith and Hope}\label{faith-and-hope}}
\addcontentsline{toc}{section}{2.2 Faith and Hope}

Follow and complete the steps below to accomplish your learning for this topic.

\begin{enumerate}
\def\labelenumi{\arabic{enumi}.}
\tightlist
\item
  Read Topic Notes
\item
  Watch Topic Video
\item
  Read Topic Reading
\item
  Complete Topic Exercise
\item
  Complete Topic Questionnaire
\item
  Watch Optional Video
\end{enumerate}

\hypertarget{topic-notes-5}{%
\subsection*{Topic Notes}\label{topic-notes-5}}
\addcontentsline{toc}{subsection}{Topic Notes}

In the previous topic, we learned about various definitions of faith, mainly faith as an attitude. The first two ways of thinking about faith as an attitude are belief and trust. These attitudes of faith are called doxastic because they generally focus on beliefs -- mental and cognitive experiences in our heads. In this topic, we focus on a non-doxastic attitude of faith called hope. Hope is non-doxastic because hope does not necessarily include belief when considering faith. Louis Pojman is going to argue that faith as hope is the more appropriate way to think about faith, especially Christian faith and the doctrine of salvation. Salvation is the idea that the human race is in danger and God is attempting to save them. According to some traditions, if humans believe in God, then God will save them. But for Pojman, humans can't control what they believe, including their doubts about God. So how can salvation be about something one cannot control? Rather, faith as hope in God and his saving plan seems to be the more accurate way to think about faith and salvation.

\hypertarget{watch-and-reflect-10}{%
\subsection*{Watch and Reflect}\label{watch-and-reflect-10}}
\addcontentsline{toc}{subsection}{Watch and Reflect}

In this video, you will learn about faith as hope in the midst of doubt from Louis Pojman's article, ``Faith, Hope, Doubt''. According to Pojman, faith is a matter of hope and not belief. Thus, faith is non-doxastic.

\textbf{Faith and Hope Unit 2 Topic 2} Video (10 min 26 sec)

\hypertarget{read-and-reflect-4}{%
\subsection*{Read and Reflect}\label{read-and-reflect-4}}
\addcontentsline{toc}{subsection}{Read and Reflect}

\begin{itemize}
\tightlist
\item
  Pojman - \href{assets/u2/PHIL-100-Pojman-Faith-Hope-and-Doubt.pdf}{\emph{Faith, Hope and Doubt}}
\end{itemize}

\hypertarget{required-pojman-reading-note-taking-exercise}{%
\subsection*{Required Pojman Reading Note-Taking Exercise}\label{required-pojman-reading-note-taking-exercise}}
\addcontentsline{toc}{subsection}{Required Pojman Reading Note-Taking Exercise}

\begin{reflect}
\begin{itemize}
\tightlist
\item
  Answering these questions will help you understand the Pojman reading and prepare you for your Unit 2 Reflection Assignment.
\item
  Click the Download button on the left after you've completed all the notes for this reading and save the answers to your computer.
\item
  You will be required to submit your downloaded notes to this reading as part of your Unit 2 Reflection Assignment.
\end{itemize}
\end{reflect}

\hypertarget{required-note-taking-questionnaire-5}{%
\subsection*{Required Note-Taking Questionnaire}\label{required-note-taking-questionnaire-5}}
\addcontentsline{toc}{subsection}{Required Note-Taking Questionnaire}

\begin{reflect}
\begin{itemize}
\tightlist
\item
  Answering these questions will help you reflect on the topic content and prepare you for your Unit 2 Reflection Assignment.
\item
  Click the Download button on the left after you've completed all the questions for this unit and save the answers to your computer.
\item
  You will be required to submit the downloaded answers to these questions as part of your Unit 2 Reflection Assignment.
\end{itemize}
\end{reflect}

\hypertarget{watch-and-reflect-11}{%
\subsection*{Watch and Reflect}\label{watch-and-reflect-11}}
\addcontentsline{toc}{subsection}{Watch and Reflect}

\begin{reflect}
In this 6 minute 17 second optional video, you will learn about faith and doubt by Richard Swinburne.

\href{https://www.youtube.com/watch?v=exsmSlxnbHQ}{Watch: \emph{Swinburne: On Doubt and Faith}}
\end{reflect}

\hypertarget{faith-and-action}{%
\section*{2.3 Faith and Action}\label{faith-and-action}}
\addcontentsline{toc}{section}{2.3 Faith and Action}

Follow and complete the steps below to accomplish your learning for this topic.

\begin{enumerate}
\def\labelenumi{\arabic{enumi}.}
\tightlist
\item
  Read Topic Notes
\item
  Watch Topic Video
\item
  Read Topic Reading
\item
  Complete Topic Exercise
\item
  Complete Topic Questionnaire
\item
  Watch Optional Video
\end{enumerate}

\hypertarget{topic-notes-6}{%
\subsection*{Topic Notes}\label{topic-notes-6}}
\addcontentsline{toc}{subsection}{Topic Notes}

In the previous two topics, we discussed faith as an attitude, such as a belief, a trust, or a hope. These conceptions of faith exist in one's head -- as a mental and cognitive experience. Faith as action focuses more on the behavior of a person and not as much on the mental experience. While it's true that our beliefs and desires and hopes may be causally relevant for our actions, this needn't always be the case. Thus, faith as action focuses more on ethical behavior as the mode of faith. In this topic, we briefly examine a controversy within Christian theology about the definition of faith. Traditionally, the view of faith was attitudinal, such as belief and trust. More recently, the suggestion is that Christian faith is about Christian ethics and good works and less so about having the attitudinal beliefs and trust about Christianity.

\hypertarget{watch-and-reflect-12}{%
\subsection*{Watch and Reflect}\label{watch-and-reflect-12}}
\addcontentsline{toc}{subsection}{Watch and Reflect}

In this video, you will learn about faith and action from Boyd and Thorsen's chapter, ``Ethics in the Christian Scriptures.'' Unlike faith as belief, trust, and hope, faith as action focuses on behavior and ethics, rather than on mental content. You will also be introduced to the debate in Christian theology about faith and works.

\textbf{Faith and Action Unit 2 Topic 3} Video (8 min 54 sec)

\hypertarget{read-and-reflect-5}{%
\subsection*{Read and Reflect}\label{read-and-reflect-5}}
\addcontentsline{toc}{subsection}{Read and Reflect}

\begin{itemize}
\tightlist
\item
  Boyd and Thorsen - \href{assets/u2/PHIL-100-Boyd-and-Thorsen-Ethics-in-the-Christian-Scriptures.pdf}{\emph{Ethics in the Christian Scriptures}}
\end{itemize}

\hypertarget{required-boyd-thorsen-reading-note-taking-exercise}{%
\subsection*{Required Boyd \& Thorsen Reading Note-Taking Exercise}\label{required-boyd-thorsen-reading-note-taking-exercise}}
\addcontentsline{toc}{subsection}{Required Boyd \& Thorsen Reading Note-Taking Exercise}

\begin{reflect}
\begin{itemize}
\tightlist
\item
  Answering these questions will help you understand the Boyd \& Thorsen reading and prepare you for your Unit 2 Reflection Assignment.
\item
  Click the Download button on the left after you've completed all the notes for this reading and save the answers to your computer.
\item
  You will be required to submit your downloaded notes to this reading as part of your Unit 2 Reflection Assignment.
\end{itemize}
\end{reflect}

\hypertarget{required-note-taking-questionnaire-6}{%
\subsection*{Required Note-Taking Questionnaire}\label{required-note-taking-questionnaire-6}}
\addcontentsline{toc}{subsection}{Required Note-Taking Questionnaire}

\begin{reflect}
\begin{itemize}
\tightlist
\item
  Answering these questions will help you reflect on the topic content and prepare you for your Unit 2 Reflection Assignment.
\item
  Click the Download button on the left after you've completed all the questions for this unit and save the answers to your computer.
\item
  You will be required to submit the downloaded answers to these questions as part of your Unit 2 Reflection Assignment.
\end{itemize}
\end{reflect}

\hypertarget{watch-and-reflect-13}{%
\subsection*{Watch and Reflect}\label{watch-and-reflect-13}}
\addcontentsline{toc}{subsection}{Watch and Reflect}

\begin{reflect}
In this 9 minute 04 second optional video, Professor Peter Singer discusses the role of religion in ethics. Questions discussed include: Is something good because a divine being approves of it, or does the divine being approve of it because it is good? How do we know what is good without religion? How do we reconcile ethics from different religions coexisting in the same society?

\href{https://www.youtube.com/watch?v=w9QtjQ5Ow7Y}{Watch: \emph{Religion and Ethics}}
\end{reflect}

\hypertarget{faith-and-rationality}{%
\section*{2.4 Faith and Rationality}\label{faith-and-rationality}}
\addcontentsline{toc}{section}{2.4 Faith and Rationality}

Follow and complete the steps below to accomplish your learning for this topic.

\begin{enumerate}
\def\labelenumi{\arabic{enumi}.}
\tightlist
\item
  Read Topic Notes
\item
  Watch Topic Video
\item
  Read Topic Reading
\item
  Complete Topic Exercise
\item
  Complete Topic Questionnaire
\item
  Watch Optional Video
\end{enumerate}

\hypertarget{topic-notes-7}{%
\subsection*{Topic Notes}\label{topic-notes-7}}
\addcontentsline{toc}{subsection}{Topic Notes}

In the previous topics, we briefly discussed some definitions of faith. A primary distinction is between faith as attitudes and faith as actions. While these categories may be interwoven, they can also remain distinct. Regardless, the critic of faith will challenge any of these ideas about faith by questioning the rationality of faith, either faith as belief and trust, or faith as action. In this topic, we explore this challenge. The term rationality in this case refers to beliefs and actions aligning with evidence. If beliefs and actions do not align with the evidence, then, according to the evidentialist, such beliefs and actions are irrational. So, according to the objection, for us to exhibit rationality, we must align our beliefs and actions according to the proper evidence. Is this a good objection to the rationality of faith as belief and action? We explore this issue by first reading William Clifford's famous piece on the ethics of belief. According to Clifford, it is wrong to believe anything not proportional to the evidence. We then consider some response to this evidential objection by noting (i) the possible incoherence of the objection; (ii) the fact that many of our rational beliefs are not proportional to the evidence; (iii) that in some cases of faith, there exists sufficient evidence to meet the rational standard.

\hypertarget{watch-and-reflect-14}{%
\subsection*{Watch and Reflect}\label{watch-and-reflect-14}}
\addcontentsline{toc}{subsection}{Watch and Reflect}

In this video, you will learn about faith and rationality from W.K. Clifford's article, ``The Ethics of Belief.'' You will be introduced to the evidential objection to religious belief, which states that religious belief is rational only if there is evidence for religious belief, and some brief responses and replies to this evidential objection.

\textbf{Faith and Rationality Unit 2 Topic 4} Video (19 min 01 sec)

\hypertarget{read-and-reflect-6}{%
\subsection*{Read and Reflect}\label{read-and-reflect-6}}
\addcontentsline{toc}{subsection}{Read and Reflect}

\begin{itemize}
\tightlist
\item
  WK Clifford - \href{assets/u2/PHIL-100-Clifford-Ethics-of-Belief.pdf}{\emph{Ethics of Belief}}
\end{itemize}

\hypertarget{required-clifford-reading-note-taking-exercise}{%
\subsection*{Required Clifford Reading Note-Taking Exercise}\label{required-clifford-reading-note-taking-exercise}}
\addcontentsline{toc}{subsection}{Required Clifford Reading Note-Taking Exercise}

\begin{reflect}
\begin{itemize}
\tightlist
\item
  Answering these questions will help you understand the Clifford reading and prepare you for your Unit 2 Reflection Assignment.
\item
  Click the Download button on the left after you've completed all the notes for this reading and save the answers to your computer.
\item
  You will be required to submit your downloaded notes to this reading as part of your Unit 2 Reflection Assignment.
\end{itemize}
\end{reflect}

\hypertarget{required-note-taking-questionnaire-7}{%
\subsection*{Required Note-Taking Questionnaire}\label{required-note-taking-questionnaire-7}}
\addcontentsline{toc}{subsection}{Required Note-Taking Questionnaire}

\begin{reflect}
\begin{itemize}
\tightlist
\item
  Answering these questions will help you reflect on the topic content and prepare you for your Unit 2 Reflection Assignment.
\item
  Click the Download button on the left after you've completed all the questions for this unit and save the answers to your computer.
\item
  You will be required to submit the downloaded answers to these questions as part of your Unit 2 Reflection Assignment.
\end{itemize}
\end{reflect}

\hypertarget{watch-and-reflect-15}{%
\subsection*{Watch and Reflect}\label{watch-and-reflect-15}}
\addcontentsline{toc}{subsection}{Watch and Reflect}

\begin{reflect}
In this 8 minute 38 second optional video, you will learn about further details about the relationship between faith and reason.

\href{https://www.youtube.com/watch?v=MTPHXNMi9tA}{Watch: \emph{Religion: Reason and Faith}}
\end{reflect}

\hypertarget{reason}{%
\chapter{Reason}\label{reason}}

This final unit is about arguments. This unit is valuable because everything we've discussed so far, both about wisdom and faith, is built upon the foundation of arguments. We give arguments for our views about wisdom. We give arguments about issues of faith and rationality. This last unit provides the opportunity for you to learn a basic skill of how to identify an argument, whether the argument is located in a book, and more often on YouTube, a podcast, and in the news. Note that we do not cover how to evaluate arguments in this unit. Evaluating arguments is a skill taught in other Philosophy courses.

In topic 1, we learn about the parts of the argument, such as the proper terms used to designate these different parts. If we wish to learn how to identify arguments, we must know what terms to look for. In topic 2, we learn about how to label and structure arguments. Often people do not organize their argument clearly and so we must learn how to reconstruct the argument so that we know the conclusion and the reasons for supporting the conclusion. In topic 3, we tackle slightly more complicated arguments and learn how to interpret what the author intended to say. As practiced in topic 2, people often organize their arguments (or, occasionally, non-arguments) in a confusing way. Moreover, we will learn about tricky terms often used in arguments, such as the terms ``if'' and ``then''. In the final topic, we practice identifying difficult arguments, which often include long texts, run on sentences, and additional irrelevant content to the argument being put forward. Our task in the final topic is to use our skills practiced in topics 1-3 to identify more difficult arguments.

\textbf{Reason Unit 3 Introduction} Video (0 min 0 sec)

\hypertarget{overview-2}{%
\section*{Overview}\label{overview-2}}
\addcontentsline{toc}{section}{Overview}

\hypertarget{topics-2}{%
\subsection*{Topics}\label{topics-2}}
\addcontentsline{toc}{subsection}{Topics}

This unit is divided into the following four topics:

\begin{enumerate}
\def\labelenumi{\arabic{enumi}.}
\tightlist
\item
  Parts and Keywords of an Argument
\item
  Labelling and Restructuring
\item
  Interpreting Arguments and Conditionals (If\ldots Then)
\item
  Identifying and Practicing Difficult Arguments
\end{enumerate}

\hypertarget{learning-outcomes-2}{%
\subsection*{Learning Outcomes}\label{learning-outcomes-2}}
\addcontentsline{toc}{subsection}{Learning Outcomes}

When you've completed this unit, you will have learned how to:

\begin{itemize}
\tightlist
\item
  Identify the parts of an argument and how to label and structure these parts
\item
  Interpret arguments with ambiguous terminology to be made more precise
\item
  Develop a greater understanding of how arguments work in contemporary culture, including issues of justice and faith
\item
  Practice the skills necessary for navigating arguments in literature, news, and podcasts
\item
  Appreciate the value and skill of knowing how to handle the basic features of an argument
\end{itemize}

\hypertarget{parts-and-keywords-of-an-argument}{%
\section*{3.1 Parts and Keywords of an Argument}\label{parts-and-keywords-of-an-argument}}
\addcontentsline{toc}{section}{3.1 Parts and Keywords of an Argument}

Follow and complete the steps below to accomplish your learning for this topic.

\begin{enumerate}
\def\labelenumi{\arabic{enumi}.}
\tightlist
\item
  Read Topic Notes
\item
  Watch Topic Video
\item
  Complete Logic Exercise 1
\item
  Complete Logic Exercise 2
\item
  Complete Topic Questionnaire
\item
  Watch Optional Video
\end{enumerate}

\hypertarget{topic-notes-8}{%
\subsection*{Topic Notes}\label{topic-notes-8}}
\addcontentsline{toc}{subsection}{Topic Notes}

In this topic, we learn about the parts of the argument and the words often used to designate these parts. Arguments are made up of statements. Statements are claims that are either true or false. Sometimes statements are whole sentences, while other times statements are phrases. For example, the question, ``Is there a God?'' is not a statement because the question does not state something that is either true or false. The answer to this question, ``yes there is a God,'' or ``no, there is no God'' are statements because they express claims that are either true or false. Other parts of the argument are the premises and conclusion. The premises are the statements that support the conclusion. The conclusion is the statement that is being argued for by the premises. The inference is the move from the premises to the conclusion. Imagine someone says they believe that God exists. This is their conclusion. You ask them, ``why think that God exists?'' Their response is going to be a reason (i.e.~a premise) or a set of reasons (i.e.~a set of premises) that are supposed to support their conclusion. The inference is the nature of the support from premise to conclusion.

Some of the primary keywords of an argument are terms that describe premises and conclusions. For example, the word ``therefore'' often picks out a conclusion. The statement that follows from the term ``therefore'' may act as a conclusion. Terms that describe premise can be ``all'' ``every'' ``some''. As we shall see, these types of words help us identify the premises of an argument.

\hypertarget{watch-and-reflect-16}{%
\subsection*{Watch and Reflect}\label{watch-and-reflect-16}}
\addcontentsline{toc}{subsection}{Watch and Reflect}

In this video, you will learn about wisdom and knowledge from Robert Nozick's piece, ``\emph{What is Wisdom and Why do Philosophers Love it So?}''

\textbf{Parts and Keywords of an Argument Unit 3 Topic 1} Video (0 min 0 sec)

\hypertarget{required-parts-of-an-argument-logic-exercise}{%
\subsection*{Required Parts of an Argument Logic Exercise}\label{required-parts-of-an-argument-logic-exercise}}
\addcontentsline{toc}{subsection}{Required Parts of an Argument Logic Exercise}

\begin{reflect}
\end{reflect}

\hypertarget{required-keywords-of-an-argument-logic-exercise}{%
\subsection*{Required Keywords of an Argument Logic Exercise}\label{required-keywords-of-an-argument-logic-exercise}}
\addcontentsline{toc}{subsection}{Required Keywords of an Argument Logic Exercise}

\begin{reflect}
\end{reflect}

\hypertarget{watch-and-reflect-17}{%
\subsection*{Watch and Reflect}\label{watch-and-reflect-17}}
\addcontentsline{toc}{subsection}{Watch and Reflect}

\begin{reflect}
In this 5 minute 24 second optional video, you will learn about the basic structure of an argument.

\href{https://www.youtube.com/watch?v=wbRxR53F3rI}{Watch: \emph{Critical Thinking \#3: Types of Arguments}}
\end{reflect}

\hypertarget{labelling-and-restructuring}{%
\section*{3.2 Labelling and Restructuring}\label{labelling-and-restructuring}}
\addcontentsline{toc}{section}{3.2 Labelling and Restructuring}

Follow and complete the steps below to accomplish your learning for this topic.

\begin{enumerate}
\def\labelenumi{\arabic{enumi}.}
\tightlist
\item
  Read Topic Notes
\item
  Watch Topic Video
\item
  Complete Logic Exercise 1
\item
  Complete Logic Exercise 2
\item
  Complete Topic Questionnaire
\item
  Watch Optional Video
\end{enumerate}

\hypertarget{topic-notes-9}{%
\subsection*{Topic Notes}\label{topic-notes-9}}
\addcontentsline{toc}{subsection}{Topic Notes}

In the previous topic, we learned about the parts and keywords of an argument. In this topic, we learn how to label these parts of the argument and then restructure the argument vertically, with premises and conclusion, so that we may see the argument clearly. The steps for applying these tools follow this pattern: (i) focus on statements and ignore non-statements. Recall from the previous topic that arguments are made up of statements only and never non-statements. Thus, only focus on statements and cross out any non-statements. (ii) Locate the premises and conclusion amongst the set of statements using keywords. As discussed, premises and conclusions often have terms that designate them, such as ``therefore''.

Once the argument has been labelled, the next step is to restructure the argument vertically with the premises on top and the conclusion beneath. This way we can see what the author's argument is exactly. That is, we can see the conclusion -- the main point of their position -- and the reasons (i.e.~premises) for thinking the conclusion is true. Once these statements are ordered correctly, anyone can see the argument. The reader needn't labour over the text and the many confusing non-statements to locate the point of the passage.

\textbf{Labelling and Restructuring Unit 3 Topic 2} Video (0 min 0 sec)

\hypertarget{required-labelling-an-argument-exercise}{%
\subsection*{Required Labelling an Argument Exercise}\label{required-labelling-an-argument-exercise}}
\addcontentsline{toc}{subsection}{Required Labelling an Argument Exercise}

\begin{reflect}
topic-4/chromeless:true/hidepagetitle:true'' allowfullscreen=``allowfullscreen'' width=``100\%'' height=``12800''\textgreater{}
\end{reflect}

\hypertarget{required-restructuring-an-argument-exercise}{%
\subsection*{Required Restructuring an Argument Exercise}\label{required-restructuring-an-argument-exercise}}
\addcontentsline{toc}{subsection}{Required Restructuring an Argument Exercise}

\begin{reflect}
\end{reflect}

\hypertarget{watch-and-reflect-18}{%
\subsection*{Watch and Reflect}\label{watch-and-reflect-18}}
\addcontentsline{toc}{subsection}{Watch and Reflect}

\begin{reflect}
In this 1 minute 22 second optional video, you will learn about the basic structure of an argument, with brief attention to what makes a bad argument.

\href{https://www.youtube.com/watch?v=Z7f_uuy1JcM}{Watch: \emph{Ninety Second Philosophy: Arguments}}
\end{reflect}

\hypertarget{interpreting-arguments-and-conditionals-ifthen}{%
\section*{3.3 Interpreting Arguments and Conditionals (If\ldots Then)}\label{interpreting-arguments-and-conditionals-ifthen}}
\addcontentsline{toc}{section}{3.3 Interpreting Arguments and Conditionals (If\ldots Then)}

Follow and complete the steps below to accomplish your learning for this topic.

\begin{enumerate}
\def\labelenumi{\arabic{enumi}.}
\tightlist
\item
  Read Topic Notes
\item
  Watch Topic Video
\item
  Complete Logic Exercise 1
\item
  Complete Logic Exercise 2
\item
  Complete Topic Questionnaire
\item
  Watch Optional Video
\end{enumerate}

\hypertarget{topic-notes-10}{%
\subsection*{Topic Notes}\label{topic-notes-10}}
\addcontentsline{toc}{subsection}{Topic Notes}

In the first two topics, we learned some foundational skills for identifying arguments, such as the parts and keywords of an argument, and the methods for labeling and restructuring arguments. The challenge, however, is that ordinary discourse -- both written and verbal -- often makes the argument difficult to identify. We often must interpret what we think the author intended to say about their view and then structure the argument according to this intention. In this topic, we practice the skill of interpretating arguments. One strategy for interpreting an argument is the ``Why/Because'' strategy. If it's difficult to identify the premises and conclusion, then look for statements that may follow with a hypothetical ``why''. Such statements may be the conclusion(s). The reason is that concluding statements have by definition reasons for them. And the term ``why'' may pick out these reasons. The term ``because'' often precedes a reason (i.e.~a premise) and so looking for statements that may be preceded with ``because'' may help with identifying the premises.

A second skill of interpretation is learning to navigate conditional statements. A conditional is a ``if \ldots{} then'' statement. For example, ``If it's raining outside, then we can't go to the park.'' Sentences with the words ``if'' and ``then'' may be interpreted in different ways depending on the order of these words. In this topic we practice some of these interpretive skills.

\textbf{Interpreting Arguments and Conditionals (If\ldots Then) Unit 3 Topic 3} Video (0 min 0 sec)

\hypertarget{required-interpreting-an-argument-exercise}{%
\subsection*{Required Interpreting an Argument Exercise}\label{required-interpreting-an-argument-exercise}}
\addcontentsline{toc}{subsection}{Required Interpreting an Argument Exercise}

\begin{reflect}
\end{reflect}

\hypertarget{required-arguments-with-ifthen-exercise}{%
\subsection*{Required Arguments with ``If\ldots Then'' Exercise}\label{required-arguments-with-ifthen-exercise}}
\addcontentsline{toc}{subsection}{Required Arguments with ``If\ldots Then'' Exercise}

\begin{reflect}
\end{reflect}

\hypertarget{watch-and-reflect-19}{%
\subsection*{Watch and Reflect}\label{watch-and-reflect-19}}
\addcontentsline{toc}{subsection}{Watch and Reflect}

\begin{reflect}
In this 6 minute optional video, you will be introduced to some further techniques for identifying arguments.

\href{https://www.youtube.com/watch?v=lYiEj5z8le8}{Watch: \emph{Lesson 2. Identifying Arguments}}
\end{reflect}

\hypertarget{identifying-and-practicing-difficult-arguments}{%
\section*{3.4 Identifying and Practicing Difficult Arguments}\label{identifying-and-practicing-difficult-arguments}}
\addcontentsline{toc}{section}{3.4 Identifying and Practicing Difficult Arguments}

Follow and complete the steps below to accomplish your learning for this topic.

\begin{enumerate}
\def\labelenumi{\arabic{enumi}.}
\tightlist
\item
  Read Topic Notes
\item
  Watch Topic Video
\item
  Complete Logic Exercise 1
\item
  Complete Logic Exercise 2
\item
  Complete Topic Questionnaire
\item
  Watch Optional Video
\end{enumerate}

\hypertarget{topic-notes-11}{%
\subsection*{Topic Notes}\label{topic-notes-11}}
\addcontentsline{toc}{subsection}{Topic Notes}

In the previous topics, we have attempted to build our foundation for identifying arguments. The real challenge is when we're confronted with long speeches, texts, reports, stories, and accusations whereby the argument may often seem unclear. Our task is to use the tools we've learned to identify the argument(s) in more difficult contexts. In this topic, we practice these skills by following these steps: (i) ignore non-statements and label the premises and conclusion; (ii) restructure the argument vertically with premises on top and the conclusion beneath. Notice that the difficult step is (i). We must sift through the content to locate keywords, interpret ambiguous statements, possibly use the ``Why/Because'' strategy in cases of doubt, decipher unusual ``if/then'' statements, and so on. Sometimes there is no argument at all; other times there may exist more than one argument, such that there is a main argument, and a secondary argument. The point of this entire exercise is to learn the skill that so many of us fail to properly master: to learn what the opposing person is trying to say. What is their position? What is their argument if there is one? Only after we've successfully identified the argument can we then properly raise objections and evaluate their argument.

\textbf{Identifying and Practicing Difficult Arguments Unit 3 Topic 4} Video (0 min 0 sec)

\hypertarget{required-identifying-difficult-arguments-exercise}{%
\subsection*{Required Identifying Difficult Arguments Exercise}\label{required-identifying-difficult-arguments-exercise}}
\addcontentsline{toc}{subsection}{Required Identifying Difficult Arguments Exercise}

\begin{reflect}
\end{reflect}

\hypertarget{required-practice-exercises}{%
\subsection*{Required Practice Exercises}\label{required-practice-exercises}}
\addcontentsline{toc}{subsection}{Required Practice Exercises}

\begin{reflect}
\end{reflect}

\hypertarget{watch-and-reflect-20}{%
\subsection*{Watch and Reflect}\label{watch-and-reflect-20}}
\addcontentsline{toc}{subsection}{Watch and Reflect}

\begin{reflect}
In this 5 minute 34 second optional video, you will be introduced to strategies for locating terms that describe parts of the argument in difficult passages, such as terms that pick out conclusions and premises.

\href{https://www.youtube.com/watch?v=07mehbgE5jc}{Watch: \emph{Identifying Premises and Conclusions}}
\end{reflect}

\hypertarget{assessment}{%
\chapter*{Assessment}\label{assessment}}
\addcontentsline{toc}{chapter}{Assessment}

The following assignments are opportunities for learners to demonstrate their understanding of the course outcomes. Please confirm assignment details with your instructor, referring to the course syllabus.

Note that Assignment dropboxes are located in Moodle. Also refer to the Course Schedule in Moodle for the specific due dates.

\hypertarget{assignment}{%
\section*{Assignment:}\label{assignment}}
\addcontentsline{toc}{section}{Assignment:}

\hypertarget{grading-criteria}{%
\subsection*{Grading Criteria}\label{grading-criteria}}
\addcontentsline{toc}{subsection}{Grading Criteria}

See the following rubric that explains how your assignment will be evaluated. Also available as a \href{assets/assessment/Identity-as-a-Teacher-RUBRIC.pdf}{pdf}

\textbf{APA/WRITING}

\textbf{Unsatisfactory:} Paper does not model language and conventions used in scholarly literature. Writing is not well-organized. Several errors in grammar or composition. Sources are not cited. APA citations are not appropriately formatted.

\textbf{Developing:} Paper partially models language and conventions used in scholarly literature. Writing is somewhat well organized and includes some errors in grammar or composition. Not all sources cited. APA citations are generally formatted correctly, with several errors.

\textbf{Proficient:} \emph{Paper consistently models language and conventions used in scholarly literature. Writing is well-organized and includes few (if any) errors in grammar or composition. All resources are appropriately cited (including in-text citations and bibliography information). Few (if any) errors in APA citations.}

\textbf{Exemplary:} Paper is an exemplar of language and conventions used in scholarly literature. Writing is well-organized and free of errors in grammar or composition. All resources are appropriately cited. No errors in APA format.

\hypertarget{statement-of-teaching-identity}{%
\subsubsection*{STATEMENT OF TEACHING IDENTITY}\label{statement-of-teaching-identity}}
\addcontentsline{toc}{subsubsection}{STATEMENT OF TEACHING IDENTITY}

\textbf{Unsatisfactory:} Does not provide a statement about identity as a teacher/facilitator

\textbf{Developing:} Provides an unclear statement about identity as a teacher/facilitator.

\textbf{Proficient:} \emph{Provides a clear, concise, and powerful statement about identity as a teacher/facilitator.}

\textbf{Exemplary:} Provides a clear, concise, and powerful statement about identity as a teacher/facilitator. Statement incorporates theory or research from course materials.

\hypertarget{developing-a-cohesive-and-logical-academic-argument}{%
\subsubsection*{DEVELOPING A COHESIVE AND LOGICAL ACADEMIC ARGUMENT}\label{developing-a-cohesive-and-logical-academic-argument}}
\addcontentsline{toc}{subsubsection}{DEVELOPING A COHESIVE AND LOGICAL ACADEMIC ARGUMENT}

\textbf{Unsatisfactory:} Does not make a focused, cohesive, or logical academic argument. Paper is confusing, and is missing an introduction, body, or conclusion. Transitions between sections and ideas are missing.

\textbf{Developing:} Makes an academic argument that is only partially focused, cohesive and logical. Paper is generally organized, but is missing an introduction, body, or conclusion. Transitions between sections and ideas are unclear.

\textbf{Proficient:} \emph{Makes a focused, cohesive, logical academic argument. Paper is effectively organized and includes an introduction, body, and conclusion. Transitions between sections and ideas are clear.}

\textbf{Exemplary:} Makes a focused, cohesive, logical and compelling academic argument. Paper is effectively organized and includes an introduction, body, and conclusion. Transitions between sections and ideas are clear, and build on each other.

\hypertarget{analysis-of-identity-as-a-teacher}{%
\subsubsection*{ANALYSIS OF IDENTITY AS A TEACHER}\label{analysis-of-identity-as-a-teacher}}
\addcontentsline{toc}{subsubsection}{ANALYSIS OF IDENTITY AS A TEACHER}

\textbf{Unsatisfactory:} Does not include three important aspects of identity as a teacher/facilitator. Does not include an analysis.

\textbf{Developing:} Lists but does not discuss three important aspects of identity as a teacher/facilitator. Includes a partial analysis.

\textbf{Proficient:} \emph{Includes a detailed discussion of three important aspects of identity as a teacher/facilitator. Includes thoughtful analysis of each of the three elements.}

\textbf{Exemplary:} Includes a detailed discussion of three important aspects of identity as a teacher/facilitator. Includes a thoughtful analysis, integrating scholarly literature to support analysis and furthering scholarly thinking related to teacher identity.

\hypertarget{scholarly-integration}{%
\subsubsection*{SCHOLARLY INTEGRATION}\label{scholarly-integration}}
\addcontentsline{toc}{subsubsection}{SCHOLARLY INTEGRATION}

\textbf{Unsatisfactory:} Does not integrate references to support claims and assertions made in the paper.

\textbf{Developing:} Integrates references to support some of the claims and assertions made in the paper.

\textbf{Proficient:} \emph{Integrates references to support claims and assertions made in the paper.}

\textbf{Exemplary:} Integrates references to support claims and assertions made in the paper, effectively synthesizing different perspectives and research results from scholarly sources.

\begin{longtable}[]{@{}
  >{\raggedright\arraybackslash}p{(\columnwidth - 8\tabcolsep) * \real{0.2000}}
  >{\raggedright\arraybackslash}p{(\columnwidth - 8\tabcolsep) * \real{0.2000}}
  >{\raggedright\arraybackslash}p{(\columnwidth - 8\tabcolsep) * \real{0.2000}}
  >{\raggedright\arraybackslash}p{(\columnwidth - 8\tabcolsep) * \real{0.2000}}
  >{\raggedright\arraybackslash}p{(\columnwidth - 8\tabcolsep) * \real{0.2000}}@{}}
\toprule\noalign{}
\begin{minipage}[b]{\linewidth}\raggedright
\textbf{TOTAL}
\end{minipage} & \begin{minipage}[b]{\linewidth}\raggedright
\textbf{0 = 0\% (F)}
\end{minipage} & \begin{minipage}[b]{\linewidth}\raggedright
\textbf{10 = 50\% (C)}
\end{minipage} & \begin{minipage}[b]{\linewidth}\raggedright
\textbf{15 = 75 (B)}
\end{minipage} & \begin{minipage}[b]{\linewidth}\raggedright
\textbf{20 = 100\% (A+)}
\end{minipage} \\
\midrule\noalign{}
\endhead
\bottomrule\noalign{}
\endlastfoot
\end{longtable}

\begin{center}\rule{0.5\linewidth}{0.5pt}\end{center}

\hypertarget{assignment-company-website-analysis}{%
\section*{Assignment: Company Website Analysis}\label{assignment-company-website-analysis}}
\addcontentsline{toc}{section}{Assignment: Company Website Analysis}

\begin{assessment}
Investigate the Human Resources or Faculty Development portion of a
company's website, a higher education institution or adult learning
facility, preferably one with which you are familiar. Focus on the
faculty or employee development part of the website. In this assignment,
you will apply the theory of teaching in/for/with depth by analyzing the
learning culture of an organization.

In a 4-5 page APA formatted paper, analyze the website by responding to
the following questions in your report:

\begin{enumerate}
\def\labelenumi{\arabic{enumi}.}
\tightlist
\item
  What can you infer about the company's learning culture?\\
\item
  From what is visible on the public website, would you say it is an
  authentic learning community? Why or why not? Discuss whether the
  website reflects aspects of one or more of the learning community
  models explored in previous lessons.\\
\item
  Do you see evidence that interconnectedness and integrity are valued?
  Explain.\\
\item
  What traits and skills seem to be valued in employees?\\
\item
  How does the company develop skills in its employees (e.g., workshops,
  seminars, mentoring)? Are the methods based on the principles of
  andragogy? (see Smith YouTube video). What specific adult learning
  strategies do you see reflected in the development/training
  opportunities for employees?
\end{enumerate}

Your paper should be 4-5 pages and should incorporate references to at
least five scholarly sources you have studied in this course, or other
scholarly sources you have identified.

The paper should include:

\begin{enumerate}
\def\labelenumi{\arabic{enumi}.}
\tightlist
\item
  Introduction\\
\item
  Analysis (responding to the prompts)\\
\item
  Conclusion\\
\item
  Reference List
\end{enumerate}
\end{assessment}

\hypertarget{company-website-analysis-rubric}{%
\subsection*{Company Website Analysis Rubric}\label{company-website-analysis-rubric}}
\addcontentsline{toc}{subsection}{Company Website Analysis Rubric}

See the following rubric that explains how your assignment will be evaluated. Also available as a \href{assets/assessment/Company-Website-Analysis-RUBRIC.pdf}{pdf}

\hypertarget{apa-formatting}{%
\subsubsection*{APA Formatting}\label{apa-formatting}}
\addcontentsline{toc}{subsubsection}{APA Formatting}

\textbf{Unsatisfactory:} Paper does not model language and conventions used in scholarly literature.
Writing is not well-organized. Several errors in grammar or composition. Sources
are not cited. APA citations are not appropriately formatted.

\textbf{Developing:} Paper partially models language and conventions used in scholarly literature.
Writing is somewhat well organized and includes some errors in grammar or
composition. Not all sources cited. APA citations are generally formatted
correctly, with several errors.

\textbf{Proficient:} \emph{Paper consistently models language and conventions used in scholarly
literature. Writing is well-organized and includes few (if any) errors in
grammar or composition. All resources are appropriately cited (including in-text
citations and bibliography information). Few (if any) errors in APA citations.}

\textbf{Exemplary:} Paper is an exemplar of language and conventions used in scholarly literature.
Writing is well-organized and free of errors in grammar or composition. All
resources are appropriately cited. No errors in APA format.

\hypertarget{developing-a-cohesive-and-logical-academic-argument-1}{%
\subsubsection*{DEVELOPING a COHESIVE and LOGICAL ACADEMIC ARGUMENT}\label{developing-a-cohesive-and-logical-academic-argument-1}}
\addcontentsline{toc}{subsubsection}{DEVELOPING a COHESIVE and LOGICAL ACADEMIC ARGUMENT}

\textbf{Unsatisfactory:} Does not make a focused, cohesive, or logical academic argument. Paper is
confusing, and is missing an introduction, body, or conclusion. Transitions
between sections and ideas are missing.

\textbf{Developing:} Makes an academic argument that is only partially focused, cohesive and logical.
Paper is generally organized, but is missing an introduction, body, or
conclusion. Transitions between sections and ideas are unclear.

\textbf{Proficient:} \emph{Makes a focused, cohesive, logical academic argument. Paper is effectively
organized and includes an introduction, body, and conclusion. Transitions
between sections and ideas are clear.}

\textbf{Exemplary:} Makes a focused, cohesive, logical and compelling academic argument. Paper is
effectively organized and includes an introduction, body, and conclusion.
Transitions between sections and ideas are clear and build on each other.

\hypertarget{analysis-of-learning-culture}{%
\subsubsection*{ANALYSIS of LEARNING CULTURE}\label{analysis-of-learning-culture}}
\addcontentsline{toc}{subsubsection}{ANALYSIS of LEARNING CULTURE}

\textbf{Unsatisfactory:} Does not include an analysis of the company learning culture, and no evaluation
of the authenticity of the learning community.

\textbf{Developing:} Includes a partial analysis of the company learning culture, including a limited
evaluation of the authenticity of the learning community.

\textbf{Proficient:} \emph{Includes a detailed analysis of the company learning culture, including an
evaluation of the authenticity of the learning community.}

\textbf{Exemplary:} Includes a detailed analysis of the company learning culture, including an
evaluation of the authenticity of the learning community. Includes a thoughtful
analysis, integrating scholarly literature to support analysis and furthering
scholarly thinking related to teacher identity.

\hypertarget{evaluation-of-interconnectedness-and-integrity}{%
\subsubsection*{EVALUATION of INTERCONNECTEDNESS and INTEGRITY}\label{evaluation-of-interconnectedness-and-integrity}}
\addcontentsline{toc}{subsubsection}{EVALUATION of INTERCONNECTEDNESS and INTEGRITY}

\textbf{Unsatisfactory:} Does not include an evaluation of evidence of interconnectedness and integrity
on the company website. Does not integrate scholarly sources in the evaluation.

\textbf{Developing:} Includes a partial evaluation of evidence of interconnectedness and integrity on
the company website. Evaluation includes only limited reference to scholarly
sources.

\textbf{Proficient:} \emph{Includes a detailed evaluation of evidence of interconnectedness and integrity
on the company website. Evaluation integrates scholarly sources.}

\textbf{Exemplary:} Includes a detailed evaluation of evidence of interconnectedness and integrity
on the company website. Includes recommendations for ways in which to integrate
interconnectedness and integrity into employee development.

\hypertarget{analysis-of-adult-learning-strategies}{%
\subsubsection*{ANALYSIS of ADULT LEARNING STRATEGIES}\label{analysis-of-adult-learning-strategies}}
\addcontentsline{toc}{subsubsection}{ANALYSIS of ADULT LEARNING STRATEGIES}

\textbf{Unsatisfactory:} Does not include a detailed analysis of valued skills and evidence of adult
learning theory in employee development. Does not integrate scholarly sources.

\textbf{Developing:} Includes a partial analysis of valued skills and evidence of adult learning
theory in employee development. Analysis integrates few, if any, scholarly
sources.

\textbf{Proficient:} \emph{Includes a detailed analysis of valued skills and evidence of adult learning
theory in employee development. Analysis integrates scholarly sources.}

\textbf{Exemplary:} Includes a detailed analysis of valued skills and evidence of adult learning
theory in employee development. Includes recommendations for ways in which to
integrate adult learning theory into employee development.

\hypertarget{scholarly-integration-1}{%
\subsubsection*{SCHOLARLY INTEGRATION}\label{scholarly-integration-1}}
\addcontentsline{toc}{subsubsection}{SCHOLARLY INTEGRATION}

\textbf{Unsatisfactory:} Does not integrate scholarly references to support claims and assertions made in
the paper.

\textbf{Developing:} Integrates scholarly references to support some of the claims and assertions
made in the paper.

\textbf{Proficient:} \emph{Integrates scholarly references to support claims and assertions made in the
paper.}

\textbf{Exemplary:} Integrates scholarly references to support claims and assertions made in the
paper, effectively synthesizing different perspectives and research results from
scholarly sources.

\begin{longtable}[]{@{}
  >{\raggedright\arraybackslash}p{(\columnwidth - 8\tabcolsep) * \real{0.2000}}
  >{\raggedright\arraybackslash}p{(\columnwidth - 8\tabcolsep) * \real{0.2000}}
  >{\raggedright\arraybackslash}p{(\columnwidth - 8\tabcolsep) * \real{0.2000}}
  >{\raggedright\arraybackslash}p{(\columnwidth - 8\tabcolsep) * \real{0.2000}}
  >{\raggedright\arraybackslash}p{(\columnwidth - 8\tabcolsep) * \real{0.2000}}@{}}
\toprule\noalign{}
\begin{minipage}[b]{\linewidth}\raggedright
\textbf{TOTAL}
\end{minipage} & \begin{minipage}[b]{\linewidth}\raggedright
\textbf{0 = 0\% (F)}
\end{minipage} & \begin{minipage}[b]{\linewidth}\raggedright
\textbf{10 = 50\% (C)}
\end{minipage} & \begin{minipage}[b]{\linewidth}\raggedright
\textbf{15 = 75 (B)}
\end{minipage} & \begin{minipage}[b]{\linewidth}\raggedright
\textbf{20 = 100\% (A+)}
\end{minipage} \\
\midrule\noalign{}
\endhead
\bottomrule\noalign{}
\endlastfoot
\end{longtable}

\begin{center}\rule{0.5\linewidth}{0.5pt}\end{center}

\hypertarget{assignment-platform-paper}{%
\section*{Assignment: Platform Paper}\label{assignment-platform-paper}}
\addcontentsline{toc}{section}{Assignment: Platform Paper}

\begin{assessment}
For this assignment, you will write a contextualized Platform Paper in
which you discuss your ideal learning community and your role as
teacher/leader of that learning community. Select a context for your
paper (i.e.~facilitating in a FAR Centre in a specific country, teaching
adult learners, facilitating employee development workshops, etc.). Your
paper should be written and referenced in APA format and include
references to a minimum of 10 scholarly sources (this can include
literature you read in this course). You will write a draft of the
Platform Paper in Unit 8 and post for Peer Review. In Unit 9, you will
provide feedback to another learner on their paper. You will make
revisions based on the Peer Review and, in Unit 10, you will submit the
final Platform Paper. Peer reviewers will be assigned in advance.

{Paper Outline}

This paper will be 12-15 pages long, and should include:

\begin{enumerate}
\def\labelenumi{\arabic{enumi}.}
\tightlist
\item
  Introduction (1-2 pages)\\
\item
  Section 1: Ideal Learning Environment (5-7 pages)\\
\item
  Section 2: Your Role as Teacher and Leader (5-7 pages)\\
\item
  Conclusion (1-2 pages)
\end{enumerate}

{Paper Guidelines}

\begin{itemize}
\tightlist
\item
  \textbf{Introduction}: Introduce the two sections in your paper,
  providing a brief description of the key points you will make in each
  section.\\
\item
  \textbf{Section 1}: In section one, you will describe your ideal
  education learning environment. This section should demonstrate your
  learning about authentic learning communities, incorporating scholarly
  sources and your own analysis to depict your ideal learning
  environment. Incorporate a discussion of the learning community
  environment, learning experiences, student learning outcomes, and
  personal beliefs about teaching and learning.\\
\item
  \textbf{Section 2}: In this section, describe your role as a teacher
  or leader within an authentic learning community. Incorporating
  scholarly literature, analyze your role as a facilitator/leader in
  planning learning experiences, facilitating student learning, and
  assessing student learning. Describe the actions, practices, and
  strategies you will engage in to achieve your vision of the learning
  community you described in section one.\\
\item
  \textbf{Conclusion}: Summarize the key points you made in each
  section.\\
\item
  \textbf{References}: Include a reference list with references to at
  least 10 scholarly sources.
\end{itemize}
\end{assessment}

\hypertarget{platform-paper-rubric}{%
\subsection*{Platform Paper Rubric}\label{platform-paper-rubric}}
\addcontentsline{toc}{subsection}{Platform Paper Rubric}

See the following rubric that explains how your assignment will be evaluated. Also available as a \href{assets/assessment/Platform-Paper-RUBRIC.pdf}{pdf}

\hypertarget{apawriting}{%
\subsubsection*{APA/WRITING}\label{apawriting}}
\addcontentsline{toc}{subsubsection}{APA/WRITING}

\textbf{Unsatisfactory:} Paper does not model language and conventions used in scholarly literature. Writing is not well-organized. Several errors in grammar or composition. Sources are not cited. APA citations are not appropriately formatted.

\textbf{Developing:} Paper partially models language and conventions used in scholarly literature. Writing is somewhat well organized and includes some errors in grammar or composition. Not all sources cited. APA citations are generally formatted correctly, with several errors.

\textbf{Proficient:} \emph{Paper consistently models language and conventions used in scholarly literature. Writing is well-organized and includes few (if any) errors in grammar or composition. All resources are appropriately cited (including in-text citations and bibliography information). Few (if any) errors in APA citations.}

\textbf{Exemplary:} Paper is an exemplar of language and conventions used in scholarly literature. Writing is well-organized and free of errors in grammar or composition. All resources are appropriately cited. No errors in APA format.

\hypertarget{developing-a-cohesive-and-logical-academic-argument-2}{%
\subsubsection*{DEVELOPING a COHESIVE and LOGICAL ACADEMIC ARGUMENT}\label{developing-a-cohesive-and-logical-academic-argument-2}}
\addcontentsline{toc}{subsubsection}{DEVELOPING a COHESIVE and LOGICAL ACADEMIC ARGUMENT}

\textbf{Unsatisfactory:} Does not make a focused, cohesive, or logical academic argument. Paper is confusing, and is missing an introduction, body, or conclusion. Transitions between sections and ideas are missing.

\textbf{Developing:} Makes an academic argument that is only partially focused, cohesive and logical. Paper is generally organized, but is missing an introduction, body, or conclusion. Transitions between sections and ideas are unclear.

\textbf{Proficient:} \emph{Makes a focused, cohesive, logical academic argument. Paper is effectively organized and includes an introduction, body, and conclusion. Transitions between sections and ideas are clear.}

\textbf{Exemplary:} Makes a focused, cohesive, logical and compelling academic argument. Paper is effectively organized and includes an introduction, body, and conclusion. Transitions between sections and ideas are clear, and build on each other.

\hypertarget{ideal-learning-environment}{%
\subsubsection*{IDEAL LEARNING ENVIRONMENT}\label{ideal-learning-environment}}
\addcontentsline{toc}{subsubsection}{IDEAL LEARNING ENVIRONMENT}

\textbf{Unsatisfactory:} Does not include a description of your ideal learning environment. Does not reference scholarly sources. Does note analyze key elements of an authentic learning community. Does not mention or describe the learning community environment, student learning outcomes, learning outcomes and personal beliefs about teaching and learning.

\textbf{Developing:} Includes a partial description of your ideal learning environment, referencing few scholarly sources and including a partial analysis of key elements of an authentic learning community. Mentions some elements, but does not fully describe the learning community environment, student learning outcomes, learning outcomes and personal beliefs about teaching and learning.

\textbf{Proficient:} \emph{Includes a detailed description of your ideal learning environment, referencing scholarly sources and analyzing key elements of an authentic learning community. Describes the learning community environment, student learning outcomes, learning outcomes and personal beliefs about teaching and learning.}

\textbf{Exemplary:} Includes a detailed description of your ideal learning environment, referencing scholarly sources and analyzing key elements of authentic learning communities. Provides a rationale for key elements of the learning community environment, student learning outcomes, learning outcomes and personal beliefs about teaching and learning. Advances scholarly thinking about authentic learning communities.

\hypertarget{your-role-as-teacher-and-leaders}{%
\subsubsection*{YOUR ROLE AS TEACHER AND LEADERS}\label{your-role-as-teacher-and-leaders}}
\addcontentsline{toc}{subsubsection}{YOUR ROLE AS TEACHER AND LEADERS}

\textbf{Unsatisfactory:} Does not include a description of your role as a teacher or leader within an authentic learning community, incorporating scholarly literature. Does not include an analysis of your role as a facilitator/leader in planning learning experiences, facilitating student learning, and assessing student learning. Does not include a description of the actions, practices, and strategies you will engage in to achieve your vision of the learning community you described in section one.

\textbf{Developing:} Includes a partial description of your role as a teacher or leader within an authentic learning community, incorporating scholarly literature. Describes but does not analyze your role as a facilitator/leader in planning learning experiences, facilitating student learning, and assessing student learning. Lists but does not describe the actions, practices, and strategies you will engage in to achieve your vision of the learning community you described in section one.

\textbf{Proficient:} \emph{Includes a detailed description of your role as a teacher or leader within an authentic learning community, incorporating scholarly literature. Includes a detailed analysis of your role as a facilitator/leader in planning learning experiences, facilitating student learning, and assessing student learning. Includes a detailed description of the actions, practices, and strategies you will engage in to achieve your vision of the learning community you described in section one.}

\textbf{Exemplary:} Includes a detailed analysis of your role as a teacher or leader within an authentic learning community, incorporating scholarly literature. Includes a detailed analysis of your role as a facilitator/leader in planning learning experiences, facilitating student learning, and assessing student learning. Includes a detailed description of the actions, practices, and strategies you will engage in to achieve your vision of the learning community you described in section one. Synthesizes scholarly thinking about the role of the teacher/leader.

\hypertarget{scholarly-integration-2}{%
\subsubsection*{SCHOLARLY INTEGRATION}\label{scholarly-integration-2}}
\addcontentsline{toc}{subsubsection}{SCHOLARLY INTEGRATION}

\textbf{Unsatisfactory:} Does not integrate many references to support the arguments made in the paper.

\textbf{Developing:} Integrates fewer than 10 scholarly sources to support arguments made in the paper.

\textbf{Proficient:} \emph{Integrates a minimum of 10 scholarly sources to support arguments made in each section of the paper.}

\textbf{Exemplary:} Integrates a minimum of 10 references to support the arguments made in each section, including several scholarly sources not included in course materials.

\begin{longtable}[]{@{}
  >{\raggedright\arraybackslash}p{(\columnwidth - 8\tabcolsep) * \real{0.2000}}
  >{\raggedright\arraybackslash}p{(\columnwidth - 8\tabcolsep) * \real{0.2000}}
  >{\raggedright\arraybackslash}p{(\columnwidth - 8\tabcolsep) * \real{0.2000}}
  >{\raggedright\arraybackslash}p{(\columnwidth - 8\tabcolsep) * \real{0.2000}}
  >{\raggedright\arraybackslash}p{(\columnwidth - 8\tabcolsep) * \real{0.2000}}@{}}
\toprule\noalign{}
\begin{minipage}[b]{\linewidth}\raggedright
\textbf{TOTAL}
\end{minipage} & \begin{minipage}[b]{\linewidth}\raggedright
\textbf{0 = 0\% (F)}
\end{minipage} & \begin{minipage}[b]{\linewidth}\raggedright
\textbf{10 = 50\% (C)}
\end{minipage} & \begin{minipage}[b]{\linewidth}\raggedright
\textbf{15 = 75 (B)}
\end{minipage} & \begin{minipage}[b]{\linewidth}\raggedright
\textbf{20 = 100\% (A+)}
\end{minipage} \\
\midrule\noalign{}
\endhead
\bottomrule\noalign{}
\endlastfoot
\end{longtable}

\hypertarget{references}{%
\chapter*{References}\label{references}}
\addcontentsline{toc}{chapter}{References}

The following are key references used in this course. \textbf{\emph{Check with your course syllabus for required readings.}}

  \bibliography{book.bib}

\end{document}
