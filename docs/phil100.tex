% Options for packages loaded elsewhere
\PassOptionsToPackage{unicode}{hyperref}
\PassOptionsToPackage{hyphens}{url}
%
\documentclass[
]{book}
\usepackage{amsmath,amssymb}
\usepackage{iftex}
\ifPDFTeX
  \usepackage[T1]{fontenc}
  \usepackage[utf8]{inputenc}
  \usepackage{textcomp} % provide euro and other symbols
\else % if luatex or xetex
  \usepackage{unicode-math} % this also loads fontspec
  \defaultfontfeatures{Scale=MatchLowercase}
  \defaultfontfeatures[\rmfamily]{Ligatures=TeX,Scale=1}
\fi
\usepackage{lmodern}
\ifPDFTeX\else
  % xetex/luatex font selection
\fi
% Use upquote if available, for straight quotes in verbatim environments
\IfFileExists{upquote.sty}{\usepackage{upquote}}{}
\IfFileExists{microtype.sty}{% use microtype if available
  \usepackage[]{microtype}
  \UseMicrotypeSet[protrusion]{basicmath} % disable protrusion for tt fonts
}{}
\makeatletter
\@ifundefined{KOMAClassName}{% if non-KOMA class
  \IfFileExists{parskip.sty}{%
    \usepackage{parskip}
  }{% else
    \setlength{\parindent}{0pt}
    \setlength{\parskip}{6pt plus 2pt minus 1pt}}
}{% if KOMA class
  \KOMAoptions{parskip=half}}
\makeatother
\usepackage{xcolor}
\usepackage{longtable,booktabs,array}
\usepackage{calc} % for calculating minipage widths
% Correct order of tables after \paragraph or \subparagraph
\usepackage{etoolbox}
\makeatletter
\patchcmd\longtable{\par}{\if@noskipsec\mbox{}\fi\par}{}{}
\makeatother
% Allow footnotes in longtable head/foot
\IfFileExists{footnotehyper.sty}{\usepackage{footnotehyper}}{\usepackage{footnote}}
\makesavenoteenv{longtable}
\usepackage{graphicx}
\makeatletter
\def\maxwidth{\ifdim\Gin@nat@width>\linewidth\linewidth\else\Gin@nat@width\fi}
\def\maxheight{\ifdim\Gin@nat@height>\textheight\textheight\else\Gin@nat@height\fi}
\makeatother
% Scale images if necessary, so that they will not overflow the page
% margins by default, and it is still possible to overwrite the defaults
% using explicit options in \includegraphics[width, height, ...]{}
\setkeys{Gin}{width=\maxwidth,height=\maxheight,keepaspectratio}
% Set default figure placement to htbp
\makeatletter
\def\fps@figure{htbp}
\makeatother
\setlength{\emergencystretch}{3em} % prevent overfull lines
\providecommand{\tightlist}{%
  \setlength{\itemsep}{0pt}\setlength{\parskip}{0pt}}
\setcounter{secnumdepth}{5}
\usepackage{booktabs}
\usepackage{amsthm}
\makeatletter
\def\thm@space@setup{%
  \thm@preskip=8pt plus 2pt minus 4pt
  \thm@postskip=\thm@preskip
}
\makeatother

\usepackage{tcolorbox}


\newtcolorbox{blackbox}{
  colback=black,
  coltext=white,
  colframe=black,
  boxsep=5pt,
  arc=4pt}
\newtcolorbox{bonus}{
  colback=blue!15,
  colframe=blue!15,
  coltext=black!80,
  boxsep=5pt,
  arc=4pt}
\newtcolorbox{reflect}{
  colback=green!5,
  colframe=green!5,
  coltext=black!80,
  boxsep=5pt,
  arc=4pt}
\newtcolorbox{assessment}{
  colback=blue!5,
  colframe=blue!5,
  coltext=black!80,
  boxsep=5pt,
  arc=4pt}
\newtcolorbox{progress}{
  colback=purple!10,
  colframe=purple!10,
  coltext=black!80,
  boxsep=5pt,
  arc=4pt}
\newtcolorbox{video}{
  colback=yellow!5,
  colframe=yellow!5,
  coltext=black!80,
  boxsep=5pt,
  arc=4pt}
\newtcolorbox{caution}{
  colback=red!5,
  colframe=red!5,
  coltext=black!80,
  boxsep=5pt,
  arc=4pt}
\newtcolorbox{feedback}{
  colback=black!5,
  colframe=black!5,
  coltext=black!80,
  boxsep=5pt,
  arc=4pt}
\ifLuaTeX
  \usepackage{selnolig}  % disable illegal ligatures
\fi
\usepackage[]{natbib}
\bibliographystyle{apalike}
\IfFileExists{bookmark.sty}{\usepackage{bookmark}}{\usepackage{hyperref}}
\IfFileExists{xurl.sty}{\usepackage{xurl}}{} % add URL line breaks if available
\urlstyle{same}
\hypersetup{
  pdftitle={Philosophy for Life},
  hidelinks,
  pdfcreator={LaTeX via pandoc}}

\title{Philosophy for Life}
\author{}
\date{\vspace{-2.5em}}

\begin{document}
\maketitle

{
\setcounter{tocdepth}{1}
\tableofcontents
}
\hypertarget{welcome}{%
\chapter*{Welcome}\label{welcome}}
\addcontentsline{toc}{chapter}{Welcome}

This is the course book for Philosophy for Life. This book is divided into thematic units of study to help you engage with the materials. The course resources and learning activities are designed not only to help prepare you for the course assessments, but also to give you opportunities to practice various skills.

\begin{quote}
Below you will find information about how to navigate this book. Please read the full course syllabus located on the Course Home page in Moodle. It includes key information about the course schedule, assignments, and policies.
\end{quote}

\hypertarget{course-notes}{%
\section*{Course Notes}\label{course-notes}}
\addcontentsline{toc}{section}{Course Notes}

You should be reading this information in the context of a Trinity Western University course offered via Moodle. If this is not the case, then this may be an unauthorized reproduction of the course. Please contact \href{mailto:elearning@twu.ca}{\nolinkurl{elearning@twu.ca}} if you have concerns.

These notes will be your guide through the learning activities and assessment strategies necessary for you to succeed in the course, so it is important for you to engage to the best of your ability and take advantage of the resources available to you through Trinity Western University.

\begin{quote}
Assessment tasks are managed in other sections of the Moodle course, so be sure to familiarize yourself with those requirements and resources.
\end{quote}

\hypertarget{how-to-navigate-this-book}{%
\section*{How To Navigate This Book}\label{how-to-navigate-this-book}}
\addcontentsline{toc}{section}{How To Navigate This Book}

To move quickly to different portions of the book, click on the appropriate chapter or section in the table of contents on the left. The buttons at the top of the page allow you to show/hide the table of contents, search the book, change font settings, download a pdf or ebook copy of this book, or get hints on various sections of the book.

\includegraphics{assets/course-intro/menu.png}

The faint left and right arrows at the sides of each page (or bottom of the page if it's narrow enough) allow you to step to the next/previous section. Here's what they look like:

\includegraphics{assets/course-intro/left_arrow.png} \includegraphics{assets/course-intro/right_arrow.png}

You can also download an offline copy of this book in various formats, such as pdf or an ebook. If you are having any accessibility or navigation issues with this book, please reach out to your instructor or our online team at \href{mailto:elearning@twu.ca}{\nolinkurl{elearning@twu.ca}}.

\hypertarget{course-units}{%
\subsection*{Course Units}\label{course-units}}
\addcontentsline{toc}{subsection}{Course Units}

This course is organized into thematic units. Each unit of the course will provide you with the following information:

\begin{itemize}
\tightlist
\item
  A general overview of the key concepts that will be addressed during the unit.\\
\item
  Specific learning outcomes and topics for the unit.\\
\item
  Learning activities to help you engage with the concepts. These often include key readings, videos, and reflective prompts.\\
\item
  The Assessment section provides details on assignments you will need to complete throughout the course to demonstrate your understanding of the course learning outcomes.
\end{itemize}

\begin{quote}
Note that assessments, including assignments and discussion posts will be submitted in Moodle. See the Assessment section(s) in Moodle for full assignment details.
\end{quote}

\hypertarget{course-activities}{%
\subsection*{Course Activities}\label{course-activities}}
\addcontentsline{toc}{subsection}{Course Activities}

Below is some key information on features you will see throughout the course.

\begin{reflect}
\textbf{\emph{Learning Activity}}

This box will prompt you to engage in course concepts, often by viewing resources and reflecting on your experience and/or learning. Most learning activities are ungraded and are designed to help prepare you for the assessment in this course.
\end{reflect}

\begin{assessment}
\textbf{\emph{Assessment}}

This box will signify an assignment or discussion post you will submit in Moodle. Note that these demonstrate your understanding of the course learning outcomes. Be sure to review the grading rubrics for each assignment.
\end{assessment}

\begin{progress}
\textbf{\emph{Checking Your Learning}}

This box is for checking your understanding, to make sure you are ready for what follows. Ways to check your learning might include self-check quizzes or questions for discussion. These activities are not graded but are critical for you to be able to begin to develop evaluative judgement in this domain of knowledge.
\end{progress}

\begin{caution}
\textbf{\emph{Note}}

This box signifies key notes. It may also warn you of possible problems or pitfalls you may encounter!
\end{caution}

If you have any questions, do not hesitate to ask. We are here to help and be your guide on this journey.

\hypertarget{wisdom}{%
\chapter{Wisdom}\label{wisdom}}

In this video, we provide a concise overview of the initial unit's four key themes: knowledge, humility, friendship, and rhetoric. While the first topic examines some fundamental concepts regarding the essence of wisdom, the majority of this unit is dedicated to unveiling practical skills essential for leading a wise and fulfilling life. We explore how to cultivate and apply these skills, emphasizing a proactive approach to embodying the principles of wisdom in our daily experiences.

\textbf{Wisdom Unit 1 Introduction} Video (6 min 10 sec)

\hypertarget{overview}{%
\section*{Overview}\label{overview}}
\addcontentsline{toc}{section}{Overview}

\hypertarget{topics}{%
\subsection*{Topics}\label{topics}}
\addcontentsline{toc}{subsection}{Topics}

This unit is divided into the following four topics:

\begin{enumerate}
\def\labelenumi{\arabic{enumi}.}
\tightlist
\item
  Wisdom and Knowledge
\item
  Wisdom and Humility
\item
  Wisdom and Friendship
\item
  Wisdom and Rhetoric
\end{enumerate}

\hypertarget{learning-outcomes}{%
\subsection*{Learning Outcomes}\label{learning-outcomes}}
\addcontentsline{toc}{subsection}{Learning Outcomes}

When you've completed this unit, you will have learned how to:

\begin{itemize}
\tightlist
\item
  Identify and understand the role wisdom plays in knowledge, humiloity, friendship and rhetoric
\item
  Practice and apply the skills of wisdom to knowledge, humility, friendship and rhetoric
\item
  Apply these skills to the types of actions a wise person may take
\item
  Enhance your skills of wisdom and reason to live life within the context of justice and faith
\item
  Develop the skills to become a well-rounded person and citizen and live a wise and just life
\end{itemize}

\hypertarget{wisdom-and-knowledge}{%
\section*{1.1 Wisdom and Knowledge}\label{wisdom-and-knowledge}}
\addcontentsline{toc}{section}{1.1 Wisdom and Knowledge}

Follow and complete the steps below to accomplish your learning for this topic.

\begin{enumerate}
\def\labelenumi{\arabic{enumi}.}
\tightlist
\item
  Read Topic Notes
\item
  Watch Topic Video
\item
  Read Topic Reading
\item
  Complete Note-Taking Exercise
\item
  Complete Topic Questionnaire
\item
  Watch Optional Video
\end{enumerate}

\hypertarget{topic-notes}{%
\subsection*{Topic Notes}\label{topic-notes}}
\addcontentsline{toc}{subsection}{Topic Notes}

In this topic, we delve into the essence of wisdom and its intricate relationship with knowledge. Wisdom transcends mere information; it is a practical virtue essential for navigating life's complexities, confronting challenges, and ideally, circumventing adversity. While wisdom is rooted in practicality, it necessitates a foundation of knowledge. To lead a well-lived life, wisdom calls for an understanding of the right beliefs and the skill to apply them judiciously and timely.

Philosopher Robert Nozick contends that a wise individual possesses beliefs spanning various facets of life, encompassing crucial values, strategies for goal achievement, and the requisite steps to realize these objectives. Furthermore, wise actions, integral to a fulfilling life, stem from the application of these foundational beliefs. Mere possession of knowledge does not equate to wisdom; the discerning application of beliefs in one's behavior is paramount.

Nozick acknowledges that, despite meticulous planning and application of appropriate beliefs and actions, unforeseen challenges may arise, disrupting our endeavors. A wise individual anticipates this inherent uncertainty, preparing for potential setbacks, acquiring the skills to avert failure, and developing resilience to respond effectively when faced with adversity. In essence, wisdom involves not only knowing what is right but also navigating the unpredictable nature of life with foresight, adaptability, and a resilient spirit.

\hypertarget{topic-video}{%
\subsection*{Topic Video}\label{topic-video}}
\addcontentsline{toc}{subsection}{Topic Video}

In this video, you will learn about wisdom and knowledge from Robert Nozick's piece, ``\emph{What is Wisdom and Why do Philosophers Love it So?}''

\textbf{Wisdom and Knowledge Unit 1 Topic 1} Video (11 min 15 sec)

\hypertarget{topic-reading}{%
\subsection*{Topic Reading}\label{topic-reading}}
\addcontentsline{toc}{subsection}{Topic Reading}

\begin{itemize}
\tightlist
\item
  Robert Nozick - \href{assets/u1/PHIL-100-Nozick-What-is-Wisdom.pdf}{\emph{What is Wisdom and Why Do Philosophers Love It So?}}
\end{itemize}

\hypertarget{note-taking-exercise}{%
\subsection*{Note-Taking Exercise}\label{note-taking-exercise}}
\addcontentsline{toc}{subsection}{Note-Taking Exercise}

\textbf{These note-taking exercises are designed to help you navigate challenging texts. While reading through the entire text without interruption can provide a grasp of the author's overarching message, it might be overwhelming in some cases, making comprehension difficult. In such instances, students may be tempted to abandon the text and seek online summaries, which may not accurately reflect the author's intent.}

\textbf{If you find the readings challenging, one effective strategy is to break down the text, either by section or even paragraph. Take point-form notes on the key ideas within each chunk of the text. As you progress and continue taking notes, you may find it helpful to revisit earlier sections to revise or add additional insights.}

\textbf{By the end of this exercise, you should have detailed notes covering the entire text, enhancing your understanding of the author's intended message. This approach proves particularly beneficial in courses where content recall is crucial for exams or when crafting a research essay, helping you better recall various research points you've gathered.}

\begin{reflect}
Click the Download button on the left after you've completed all the notes for this reading and save the answers to your computer as you will be required to submit your downloaded notes to this reading as part of your Unit 1 Reflection Assignment.
\end{reflect}

\hypertarget{topic-questionnaire}{%
\subsection*{Topic Questionnaire}\label{topic-questionnaire}}
\addcontentsline{toc}{subsection}{Topic Questionnaire}

\begin{reflect}
\begin{itemize}
\tightlist
\item
  Answering these questions will help you reflect on the topic content and prepare you for your Unit 1 Reflection Assignment.
\item
  Click the Download button on the left after you've completed all the questions for this unit and save the answers to your computer as you will be required to submit the downloaded answers to these questions as part of your Unit 1 Reflection Assignment.
\end{itemize}
\end{reflect}

\hypertarget{optional-video}{%
\subsection*{Optional Video}\label{optional-video}}
\addcontentsline{toc}{subsection}{Optional Video}

\begin{reflect}
In this 5 minute 55 second optional video, philosophers Valerie Tiberius and Philip Kitcher, as well as psychologist Lisa Feldman Barret discuss their definitions of wisdom.

\href{https://youtu.be/obqedyeUcwk}{Watch: \emph{What is Wisdom?}}
\end{reflect}

\hypertarget{wisdom-and-humility}{%
\section*{1.2 Wisdom and Humility}\label{wisdom-and-humility}}
\addcontentsline{toc}{section}{1.2 Wisdom and Humility}

Follow and complete the steps below to accomplish your learning for this topic.

\begin{enumerate}
\def\labelenumi{\arabic{enumi}.}
\tightlist
\item
  Read Topic Notes
\item
  Watch Topic Video
\item
  Read Topic Reading
\item
  Complete Note-Taking Exercise
\item
  Complete Topic Questionnaire
\item
  Watch Optional Video
\end{enumerate}

\hypertarget{topic-notes-1}{%
\subsection*{Topic Notes}\label{topic-notes-1}}
\addcontentsline{toc}{subsection}{Topic Notes}

Within this topic, our attention shifts to the intertwining realms of wisdom and humility. A wise individual possesses a keen awareness of the boundaries of their knowledge, acknowledging both what they comprehend and what lies beyond their understanding. Here, ``ignorance'' is not disparaging but denotes a specific lack of knowledge in a given domain.

The dialogue unfolds in Plato's ``the Apology,'' where Socrates, perplexed by the Oracle of Delphi's proclamation that no one is wiser than he, embarks on a quest to unravel the meaning of this paradox. Through dialogues with politicians, poets, and artisans, Socrates discovers that those who claim profound knowledge often lack true understanding. The pivotal distinction lies in Socrates' awareness of his ignorance, particularly regarding the nature of beauty and what is good, in contrast to others who falsely believe they possess knowledge. This self-awareness becomes the hallmark of Socrates' wisdom.

Applying the principle of acknowledging ignorance to the pursuit of wisdom involves cultivating a humble stance towards life's intricate questions. Whether grappling with the interpretation of the Bible, contemplating evolutionary theories, exploring the nature of sexuality, evaluating the impact of social media, or assessing the effectiveness of political leadership, a wise person approaches these complex issues with humility. This involves posing thoughtful questions, recognizing the limits of one's own understanding, refraining from hasty judgments, and maintaining a receptivity to evolving perspectives. In essence, humility becomes a guiding virtue in the pursuit of wisdom, fostering a mindset that values curiosity, open-mindedness, and the continual quest for understanding.

\hypertarget{topic-video-1}{%
\subsection*{Topic Video}\label{topic-video-1}}
\addcontentsline{toc}{subsection}{Topic Video}

In this video, you will learn about wisdom and humility from Plato's \emph{Apology}.

\textbf{Wisdom and Humility Unit 1 Topic 2} Video (13 min 54 sec)

\hypertarget{topic-reading-1}{%
\subsection*{Topic Reading}\label{topic-reading-1}}
\addcontentsline{toc}{subsection}{Topic Reading}

\begin{itemize}
\tightlist
\item
  Plato - \emph{Apology} - \url{https://classics.mit.edu/Plato/apology.html}
\end{itemize}

\hypertarget{note-taking-exercise-1}{%
\subsection*{Note-Taking Exercise}\label{note-taking-exercise-1}}
\addcontentsline{toc}{subsection}{Note-Taking Exercise}

\begin{reflect}
Click the Download button on the left after you've completed all the notes for this reading and save the answers to your computer as you will be required to submit your downloaded notes to this reading as part of your Unit 1 Reflection Assignment.
\end{reflect}

\hypertarget{topic-questionnaire-1}{%
\subsection*{Topic Questionnaire}\label{topic-questionnaire-1}}
\addcontentsline{toc}{subsection}{Topic Questionnaire}

\begin{reflect}
\begin{itemize}
\tightlist
\item
  Answering these questions will help you reflect on the topic content and prepare you for your Unit 1 Reflection Assignment.
\item
  Click the Download button on the left after you've completed all the questions for this unit and save the answers to your computer as you will be required to submit the downloaded answers to these questions as part of your Unit 1 Reflection Assignment.
\end{itemize}
\end{reflect}

\hypertarget{optional-video-1}{%
\subsection*{Optional Video}\label{optional-video-1}}
\addcontentsline{toc}{subsection}{Optional Video}

\begin{reflect}
In this 2 minute 59 second optional video with Dr.~Ian Church, \emph{What is Intellectual Humility}? A doxastic account. Here is the first of three parts. If you have the time, you should watch all three parts.

\href{https://www.youtube.com/watch?v=8CZIkGEJYRY}{Watch: \emph{What is Intellectual Humility?}}
\end{reflect}

\hypertarget{wisdom-and-friendship}{%
\section*{1.3 Wisdom and Friendship}\label{wisdom-and-friendship}}
\addcontentsline{toc}{section}{1.3 Wisdom and Friendship}

Follow and complete the steps below to accomplish your learning for this topic.

\begin{enumerate}
\def\labelenumi{\arabic{enumi}.}
\tightlist
\item
  Read Topic Notes
\item
  Watch Topic Video
\item
  Read Topic Reading
\item
  Complete Note-Taking Exercise
\item
  Complete Topic Questionnaire
\item
  Watch Optional Video
\end{enumerate}

\hypertarget{topic-notes-2}{%
\subsection*{Topic Notes}\label{topic-notes-2}}
\addcontentsline{toc}{subsection}{Topic Notes}

The third topic explores the intricate relationship between wisdom and friendship, emphasizing the discerning approach a wise individual takes in understanding the various dimensions of companionship. To navigate this terrain effectively, it is imperative to comprehend the diverse types of friendships, as expounded by Aristotle.

Aristotle categorizes friendships into three distinct types: those based on utility, pleasure, and virtue. Friends of utility serve as instrumental connections, such as business partners or teammates in sports. Friends of pleasure share enjoyable activities, while friends of virtue embody individuals with admirable qualities, actively contributing to our moral development. Aristotle contends that friendships rooted in virtue are the most profound, albeit challenging to cultivate. Such friendships extend beyond mere utility or pleasure; they transcend difficulties, demonstrating a genuine concern for one another's character development.

Thomas Aquinas further enriches this discourse by cautioning against qualities that undermine the pursuit of virtue in friendships. Envy, manifesting as displeasure at a friend's success, and arrogant pride, reflected in an inability to handle jests or an inflated sense of self-worth, are identified as detrimental vices. Aquinas urges us to be vigilant in recognizing and avoiding these pitfalls, both in our friends and within ourselves.

Crucially, this exploration prompts introspection. Are our friendships solely utilitarian or pleasure-seeking? Do we harbor envy towards our friends' accomplishments? Do we exhibit arrogant pride? The video introduces these insights from Aristotle and Aquinas, providing valuable guidance to nurture robust and virtuous friendships conducive to a life well-lived.

\hypertarget{topic-video-2}{%
\subsection*{Topic Video}\label{topic-video-2}}
\addcontentsline{toc}{subsection}{Topic Video}

In this video, you will learn about wisdom and friendship from Aristotle's \emph{Nicomachean Ethics}, Books 8, 9 and 10, and some additional points from Thomas Aquinas.

\textbf{Wisdom and Friendship Unit 1 Topic 3} Video (24 min 18 sec)

\hypertarget{topic-reading-2}{%
\subsection*{Topic Reading}\label{topic-reading-2}}
\addcontentsline{toc}{subsection}{Topic Reading}

\begin{itemize}
\tightlist
\item
  Aristotle - \href{assets/u1/PHIL-100-Aristotle-NE-VIII-IX-X.pdf}{\emph{Nicomachean Ethics}, Books VIII, IX and X}
\end{itemize}

\hypertarget{note-taking-exercise-2}{%
\subsection*{Note-Taking Exercise}\label{note-taking-exercise-2}}
\addcontentsline{toc}{subsection}{Note-Taking Exercise}

\begin{reflect}
Click the Download button on the left after you've completed all the notes for this reading and save the answers to your computer as you will be required to submit your downloaded notes to this reading as part of your Unit 1 Reflection Assignment.
\end{reflect}

\hypertarget{topic-questionnaire-2}{%
\subsection*{Topic Questionnaire}\label{topic-questionnaire-2}}
\addcontentsline{toc}{subsection}{Topic Questionnaire}

\begin{reflect}
\begin{itemize}
\tightlist
\item
  Answering these questions will help you reflect on the topic content and prepare you for your Unit 1 Reflection Assignment.
\item
  Click the Download button on the left after you've completed all the questions for this unit and save the answers to your computer as you will be required to submit the downloaded answers to these questions as part of your Unit 1 Reflection Assignment.
\end{itemize}
\end{reflect}

\hypertarget{optional-video-2}{%
\subsection*{Optional Video}\label{optional-video-2}}
\addcontentsline{toc}{subsection}{Optional Video}

\begin{reflect}
In this 7 minute 21 second optional video, you will learn further details about Aristotle's view of friendship.

\href{https://www.youtube.com/watch?v=F18kSA8OxqY}{Watch: \emph{Aristotle's Timeless Advice on What Real Friendship Is and Why It Matters}}
\end{reflect}

\hypertarget{wisdom-and-rhetoric}{%
\section*{1.4 Wisdom and Rhetoric}\label{wisdom-and-rhetoric}}
\addcontentsline{toc}{section}{1.4 Wisdom and Rhetoric}

Follow and complete the steps below to accomplish your learning for this topic.

\begin{enumerate}
\def\labelenumi{\arabic{enumi}.}
\tightlist
\item
  Read Topic Notes
\item
  Watch Topic Video
\item
  Read Topic Reading
\item
  Complete Note-Taking Exercise
\item
  Complete Topic Questionnaire
\item
  Watch Optional Video
\end{enumerate}

\hypertarget{topic-notes-3}{%
\subsection*{Topic Notes}\label{topic-notes-3}}
\addcontentsline{toc}{subsection}{Topic Notes}

The final topic examines the intersection of wisdom and rhetoric, recognizing that engaging in arguments with others is an inherent aspect of daily life. A wise individual possesses the skill to navigate these discussions without descending into conflict. According to Jay Heinrichs, productive argumentation requires a clear understanding of the desired outcome.

The crucial skill in constructive argumentation involves defining the appropriate goal for the situation. Are you seeking to influence a decision, aiming for a sense of victory, or hoping to shift your opponent's mood or perspective? Perhaps your goal is to efficiently conclude the argument while maintaining a positive relationship with the other person. Each goal requires a tailored approach, prompting consideration of the feasibility of your objectives and the methods to achieve them.

Moreover, one ought to consider reflecting on the post-argument scenario. Do you wish to uphold the relationship after the disagreement? What compromises are you willing to make, and are you prepared to acknowledge defeat from your opponent's perspective if it serves your overarching goal? Heinrichs encourages a thoughtful examination of the argument's purpose, emphasizing the strategic choices and compromises necessary to align with your desired outcome.

In essence, this exploration aims to equip individuals with the skills and insights to engage in persuasive and constructive discussions, fostering an environment where arguments serve as tools for understanding and collaboration rather than sources of conflict.

\hypertarget{topic-video-3}{%
\subsection*{Topic Video}\label{topic-video-3}}
\addcontentsline{toc}{subsection}{Topic Video}

In this video, you will learn about wisdom and rhetoric from Jay Heinrich's \emph{Thank You for Arguing}, Chapter 2 ``Set Your Goals''.

\textbf{Wisdom and Rhetoric Unit 1 Topic 4} Video (12 min 19 sec)

\hypertarget{topic-reading-3}{%
\subsection*{Topic Reading}\label{topic-reading-3}}
\addcontentsline{toc}{subsection}{Topic Reading}

\begin{itemize}
\tightlist
\item
  Jay Heinrichs - \href{assets/u1/PHIL-100-Heinrichs-Thank-You-for-Arguing.pdf}{\emph{Thank You for Arguing}}
\end{itemize}

\hypertarget{note-taking-exercise-3}{%
\subsection*{Note-Taking Exercise}\label{note-taking-exercise-3}}
\addcontentsline{toc}{subsection}{Note-Taking Exercise}

\begin{reflect}
Click the Download button on the left after you've completed all the notes for this reading and save the answers to your computer as you will be required to submit your downloaded notes to this reading as part of your Unit 1 Reflection Assignment.
\end{reflect}

\hypertarget{topic-questionnaire-3}{%
\subsection*{Topic Questionnaire}\label{topic-questionnaire-3}}
\addcontentsline{toc}{subsection}{Topic Questionnaire}

\begin{reflect}
\begin{itemize}
\tightlist
\item
  Answering these questions will help you reflect on the topic content and prepare you for your Unit 1 Reflection Assignment.
\item
  Click the Download button on the left after you've completed all the questions for this unit and save the answers to your computer as you will be required to submit the downloaded answers to these questions as part of your Unit 1 Reflection Assignment.
\end{itemize}
\end{reflect}

\hypertarget{optional-video-3}{%
\subsection*{Optional Video}\label{optional-video-3}}
\addcontentsline{toc}{subsection}{Optional Video}

\begin{reflect}
In this 4 minute 29 second optional video. Here, Camille A. Langston, a specialist in rhetoric and nineteenth-century women, explains the basics of deliberative rhetoric and shares tips for appealing to the ethos, logos, and pathos of your audience.

\href{https://www.youtube.com/watch?v=3klMM9BkW5o}{Watch: \emph{How to Use Rhetoric to Get What You Want}}
\end{reflect}

\hypertarget{faith}{%
\chapter{Faith}\label{faith}}

In this video, we provide a succinct overview of the four key themes in the second unit: definitions, hope, action, and rationality. Our exploration will center around the intricate nature of faith, encompassing not only the attitudes associated with faith, such as beliefs and trust, but also delving into the actions inspired by faith, with a brief emphasis on ethical considerations. Lastly, our journey will lead us to scrutinize the rationality of faith, contemplating its alignment with evidence and the broader spectrum of reasoned thought. Join us as we navigate these nuanced dimensions of faith, seeking a deeper understanding of its essence and implications.

\textbf{Faith Unit 2 Introduction} Video (7 min 34 sec)

\hypertarget{overview-1}{%
\section*{Overview}\label{overview-1}}
\addcontentsline{toc}{section}{Overview}

\hypertarget{topics-1}{%
\subsection*{Topics}\label{topics-1}}
\addcontentsline{toc}{subsection}{Topics}

This unit is divided into the following four topics:

\begin{enumerate}
\def\labelenumi{\arabic{enumi}.}
\tightlist
\item
  Faith and Definitions
\item
  Faith and Hope
\item
  Faith and Action
\item
  Faith and Rationality
\end{enumerate}

\hypertarget{learning-outcomes-1}{%
\subsection*{Learning Outcomes}\label{learning-outcomes-1}}
\addcontentsline{toc}{subsection}{Learning Outcomes}

When you've completed this unit, you will have learned how to:

\begin{itemize}
\tightlist
\item
  Identify and understand the nature of faith, including definitions, ethical implications, and its relationship to rationality
\item
  Develop a greater understanding of the ethical behavior required by faith
\item
  Identify briefly how Christianity applies to living life within the context of justice and faith
\item
  Enhance a greater understanding about faith and rationality
\item
  Navigate some of the objections to the nature of faith, particularly the beliefs about faith
\end{itemize}

\hypertarget{faith-and-definitions}{%
\section*{2.1 Faith and Definitions}\label{faith-and-definitions}}
\addcontentsline{toc}{section}{2.1 Faith and Definitions}

Follow and complete the steps below to accomplish your learning for this topic.

\begin{enumerate}
\def\labelenumi{\arabic{enumi}.}
\tightlist
\item
  Read Topic Notes
\item
  Watch Topic Video
\item
  Read Topic Reading
\item
  Complete Note-Taking Exercise
\item
  Complete Topic Questionnaire
\item
  Watch Optional Video
\end{enumerate}

\hypertarget{topic-notes-4}{%
\subsection*{Topic Notes}\label{topic-notes-4}}
\addcontentsline{toc}{subsection}{Topic Notes}

The first topic includes considering the multifaceted definitions of the term ``faith.'' A pivotal distinction surfaces when considering faith both as an attitude and as an action. Faith as an attitude encapsulates the cognitive realm, residing in our thoughts and beliefs, encompassing notions of trust and hope. This conceptualization aligns with common perceptions of faith as a mental and cognitive experience.

In contrast, faith as an action introduces an ethical dimension to the term. Rooted in the Greek word ``pistis,'' faith, in this context, pertains to ethical behavior directed towards a person or an ideal. This perspective invites us to view faith not merely as a mental state but as a manifestation of ethical considerations in our actions and conduct.

Further nuances emerge within faith as an attitude, where we explore the differentiation between doxastic faith and non-doxastic faith. The term ``doxastic'' centers on belief, indicating a cognitive dimension to faith. Conversely, ``non-doxastic'' faith extends beyond mere belief, encompassing faith that operates independently of explicit cognitive assent.

This exploration aims to unravel the intricate layers of faith, encouraging a comprehensive understanding that spans both the cognitive and ethical dimensions of this complex and significant concept.

\hypertarget{topic-video-4}{%
\subsection*{Topic Video}\label{topic-video-4}}
\addcontentsline{toc}{subsection}{Topic Video}

In this video, you will learn about the various definitions of faith from Elizabeth Jackson's article, ``Faith: Contemporary Perspectives''. The main focus is on learning about various distinctions with the concept of faith, such as attitude faith and action faith, and doxastic and non-doxastic view of faith.

\textbf{Faith and Definitions Unit 2 Topic 1} Video (8 min 01 sec)

\hypertarget{topic-reading-4}{%
\subsection*{Topic Reading}\label{topic-reading-4}}
\addcontentsline{toc}{subsection}{Topic Reading}

\begin{itemize}
\tightlist
\item
  Jackson - \emph{Types of Faith} - \url{https://iep.utm.edu/faith-contemporary-perspectives/\#SSH1ai\%7Btarget=\%22_blank}``\}
\end{itemize}

\hypertarget{note-taking-exercise-4}{%
\subsection*{Note-Taking Exercise}\label{note-taking-exercise-4}}
\addcontentsline{toc}{subsection}{Note-Taking Exercise}

\textbf{These note-taking exercises are designed to help you navigate challenging texts. While reading through the entire text without interruption can provide a grasp of the author's overarching message, it might be overwhelming in some cases, making comprehension difficult. In such instances, students may be tempted to abandon the text and seek online summaries, which may not accurately reflect the author's intent.}

\textbf{If you find the readings challenging, one effective strategy is to break down the text, either by section or even paragraph. Take point-form notes on the key ideas within each chunk of the text. As you progress and continue taking notes, you may find it helpful to revisit earlier sections to revise or add additional insights.}

\textbf{By the end of this exercise, you should have detailed notes covering the entire text, enhancing your understanding of the author's intended message. This approach proves particularly beneficial in courses where content recall is crucial for exams or when crafting a research essay, helping you better recall various research points you've gathered.}

\begin{reflect}
Click the Download button on the left after you've completed all the notes for this reading and save the answers to your computer as you will be required to submit your downloaded notes to this reading as part of your Unit 1 Reflection Assignment.
\end{reflect}

\hypertarget{topic-questionnaire-4}{%
\subsection*{Topic Questionnaire}\label{topic-questionnaire-4}}
\addcontentsline{toc}{subsection}{Topic Questionnaire}

\begin{reflect}
\begin{itemize}
\tightlist
\item
  Answering these questions will help you reflect on the topic content and prepare you for your Unit 2 Reflection Assignment.
\item
  Click the Download button on the left after you've completed all the questions for this unit and save the answers to your computer as you will be required to submit the downloaded answers to these questions as part of your Unit 2 Reflection Assignment.
\end{itemize}
\end{reflect}

\hypertarget{optional-video-4}{%
\subsection*{Optional Video}\label{optional-video-4}}
\addcontentsline{toc}{subsection}{Optional Video}

\begin{reflect}
In this 10 minute 06 second optional video, you will learn some further distinctions between the terms faith and belief, especially in religious contexts.

\href{https://www.youtube.com/watch?v=o8QKkHWUSu0}{Watch: \emph{Belief vs Faith (Philosophical Distinction)}}
\end{reflect}

\hypertarget{faith-and-hope}{%
\section*{2.2 Faith and Hope}\label{faith-and-hope}}
\addcontentsline{toc}{section}{2.2 Faith and Hope}

Follow and complete the steps below to accomplish your learning for this topic.

\begin{enumerate}
\def\labelenumi{\arabic{enumi}.}
\tightlist
\item
  Read Topic Notes
\item
  Watch Topic Video
\item
  Read Topic Reading
\item
  Complete Note-Taking Exercise
\item
  Complete Topic Questionnaire
\item
  Watch Optional Video
\end{enumerate}

\hypertarget{topic-notes-5}{%
\subsection*{Topic Notes}\label{topic-notes-5}}
\addcontentsline{toc}{subsection}{Topic Notes}

Building on our exploration of faith as an attitude, particularly focusing on belief and trust in the previous topic, we now inquire into a non-doxastic facet of faith: hope. Unlike belief and trust, hope doesn't necessarily hinge on cognitive assent, marking it as a non-doxastic attitude within the realm of faith. Louis Pojman contends that understanding faith as hope is particularly relevant, especially in the context of Christian faith and the doctrine of salvation.

The concept of salvation suggests that humanity is in jeopardy, and God is endeavoring to rescue them. Traditional perspectives posit that belief in God is a prerequisite for salvation. However, Pojman challenges this notion, recognizing that humans don't have absolute control over their beliefs, including doubts about God. In this context, the idea that salvation is contingent on something beyond one's control raises questions. Pojman argues that faith as hope in God and his redemptive plan provides a more accurate and inclusive perspective on faith and salvation.

By emphasizing hope as a non-doxastic attitude within faith, this discussion broadens our understanding of faith beyond mere cognitive elements, offering a nuanced perspective that aligns with the complexities of belief, doubt, and the pursuit of salvation.

\hypertarget{topic-video-5}{%
\subsection*{Topic Video}\label{topic-video-5}}
\addcontentsline{toc}{subsection}{Topic Video}

In this video, you will learn about faith as hope in the midst of doubt from Louis Pojman's article, ``Faith, Hope, Doubt''. According to Pojman, faith is a matter of hope and not belief. Thus, faith is non-doxastic.

\textbf{Faith and Hope Unit 2 Topic 2} Video (10 min 26 sec)

\hypertarget{topic-reading-5}{%
\subsection*{Topic Reading}\label{topic-reading-5}}
\addcontentsline{toc}{subsection}{Topic Reading}

\begin{itemize}
\tightlist
\item
  Pojman - \href{assets/u2/PHIL-100-Pojman-Faith-Hope-and-Doubt.pdf}{\emph{Faith, Hope and Doubt}}
\end{itemize}

\hypertarget{note-taking-exercise-5}{%
\subsection*{Note-Taking Exercise}\label{note-taking-exercise-5}}
\addcontentsline{toc}{subsection}{Note-Taking Exercise}

\begin{reflect}
Click the Download button on the left after you've completed all the notes for this reading and save the answers to your computer as you will be required to submit your downloaded notes to this reading as part of your Unit 1 Reflection Assignment.
\end{reflect}

\hypertarget{topic-questionnaire-5}{%
\subsection*{Topic Questionnaire}\label{topic-questionnaire-5}}
\addcontentsline{toc}{subsection}{Topic Questionnaire}

\begin{reflect}
\begin{itemize}
\tightlist
\item
  Answering these questions will help you reflect on the topic content and prepare you for your Unit 2 Reflection Assignment.
\item
  Click the Download button on the left after you've completed all the questions for this unit and save the answers to your computer as you will be required to submit the downloaded answers to these questions as part of your Unit 2 Reflection Assignment.
\end{itemize}
\end{reflect}

\hypertarget{optional-video-5}{%
\subsection*{Optional Video}\label{optional-video-5}}
\addcontentsline{toc}{subsection}{Optional Video}

\begin{reflect}
In this 6 minute 17 second optional video, you will learn about faith and doubt by Richard Swinburne.

\href{https://www.youtube.com/watch?v=exsmSlxnbHQ}{Watch: \emph{Swinburne: On Doubt and Faith}}
\end{reflect}

\hypertarget{faith-and-action}{%
\section*{2.3 Faith and Action}\label{faith-and-action}}
\addcontentsline{toc}{section}{2.3 Faith and Action}

Follow and complete the steps below to accomplish your learning for this topic.

\begin{enumerate}
\def\labelenumi{\arabic{enumi}.}
\tightlist
\item
  Read Topic Notes
\item
  Watch Topic Video
\item
  Read Topic Reading
\item
  Complete Note-Taking Exercise
\item
  Complete Topic Questionnaire
\item
  Watch Optional Video
\end{enumerate}

\hypertarget{topic-notes-6}{%
\subsection*{Topic Notes}\label{topic-notes-6}}
\addcontentsline{toc}{subsection}{Topic Notes}

In our recent discussions, we've examined faith as an attitude, encompassing beliefs, trust, and hope---mental and cognitive experiences that reside within an individual. Shifting the focus, we now explore faith as action, centering on behavioral aspects rather than solely on mental experiences. While beliefs, desires, and hopes may influence actions, this connection is not absolute. Consequently, faith as action directs attention to ethical behavior as the primary mode of expressing faith.

In this third topic, we briefly navigate a theological controversy within Christianity concerning the definition of faith. Traditionally, faith was construed as an attitudinal concept, emphasizing beliefs and trust. However, contemporary perspectives challenge this tradition, proposing that Christian faith is more intricately linked to Christian ethics and good works, de-emphasizing the importance of specific attitudinal beliefs and trust in Christianity.

This exploration invites us to reconsider the nature of faith within the Christian context, prompting reflection on whether faith should be predominantly characterized by mental attitudes or by the ethical conduct and good works that spring forth from one's beliefs.

\hypertarget{topic-video-6}{%
\subsection*{Topic Video}\label{topic-video-6}}
\addcontentsline{toc}{subsection}{Topic Video}

In this video, you will learn about faith and action from Boyd and Thorsen's chapter, ``Ethics in the Christian Scriptures.'' Unlike faith as belief, trust, and hope, faith as action focuses on behavior and ethics, rather than on mental content. You will also be introduced to the debate in Christian theology about faith and works.

\textbf{Faith and Action Unit 2 Topic 3} Video (8 min 54 sec)

\hypertarget{topic-reading-6}{%
\subsection*{Topic Reading}\label{topic-reading-6}}
\addcontentsline{toc}{subsection}{Topic Reading}

\begin{itemize}
\tightlist
\item
  Boyd and Thorsen - \href{assets/u2/PHIL-100-Boyd-and-Thorsen-Ethics-in-the-Christian-Scriptures.pdf}{\emph{Ethics in the Christian Scriptures}}
\end{itemize}

\hypertarget{note-taking-exercise-6}{%
\subsection*{Note-Taking Exercise}\label{note-taking-exercise-6}}
\addcontentsline{toc}{subsection}{Note-Taking Exercise}

\begin{reflect}
Click the Download button on the left after you've completed all the notes for this reading and save the answers to your computer as you will be required to submit your downloaded notes to this reading as part of your Unit 1 Reflection Assignment.
\end{reflect}

\hypertarget{topic-questionnaire-6}{%
\subsection*{Topic Questionnaire}\label{topic-questionnaire-6}}
\addcontentsline{toc}{subsection}{Topic Questionnaire}

\begin{reflect}
\begin{itemize}
\tightlist
\item
  Answering these questions will help you reflect on the topic content and prepare you for your Unit 2 Reflection Assignment.
\item
  Click the Download button on the left after you've completed all the questions for this unit and save the answers to your computer as you will be required to submit the downloaded answers to these questions as part of your Unit 2 Reflection Assignment.
\end{itemize}
\end{reflect}

\hypertarget{optional-video-6}{%
\subsection*{Optional Video}\label{optional-video-6}}
\addcontentsline{toc}{subsection}{Optional Video}

\begin{reflect}
In this 9 minute 04 second optional video, Professor Peter Singer discusses the role of religion in ethics. Questions discussed include: Is something good because a divine being approves of it, or does the divine being approve of it because it is good? How do we know what is good without religion? How do we reconcile ethics from different religions coexisting in the same society?

\href{https://www.youtube.com/watch?v=w9QtjQ5Ow7Y}{Watch: \emph{Religion and Ethics}}
\end{reflect}

\hypertarget{faith-and-rationality}{%
\section*{2.4 Faith and Rationality}\label{faith-and-rationality}}
\addcontentsline{toc}{section}{2.4 Faith and Rationality}

Follow and complete the steps below to accomplish your learning for this topic.

\begin{enumerate}
\def\labelenumi{\arabic{enumi}.}
\tightlist
\item
  Read Topic Notes
\item
  Watch Topic Video
\item
  Read Topic Reading
\item
  Complete Note-Taking Exercise
\item
  Complete Topic Questionnaire
\item
  Watch Optional Video
\end{enumerate}

\hypertarget{topic-notes-7}{%
\subsection*{Topic Notes}\label{topic-notes-7}}
\addcontentsline{toc}{subsection}{Topic Notes}

In our preceding discussions, we touched upon various definitions of faith, distinguishing between faith as attitudes and faith as actions. While these categories may overlap, they can also maintain distinct identities. However, a persistent challenge to these notions of faith arises from critics who question the rationality of faith, whether conceived as belief and trust or as action. This final topic examines this challenge.

The term ``rationality'' here pertains to the alignment of beliefs and actions with evidence. According to the evidentialist perspective, if beliefs and actions deviate from the evidence, they are deemed irrational. Hence, the objection posits that to exhibit rationality, one must align beliefs and actions according to the available evidence. The ensuing exploration assesses the strength of this objection, beginning with an examination of William Clifford's renowned work on the ethics of belief.

Clifford contends that it is ethically wrong to believe anything not proportional to the evidence. To evaluate this evidential objection, we consider several responses, including the potential incoherence of the objection, the observation that many of our rational beliefs are not strictly proportional to the evidence, and the recognition that, in certain cases of faith, there may be sufficient evidence to meet rational standards.

\hypertarget{topic-video-7}{%
\subsection*{Topic Video}\label{topic-video-7}}
\addcontentsline{toc}{subsection}{Topic Video}

In this video, you will learn about faith and rationality from W.K. Clifford's article, ``The Ethics of Belief.'' You will be introduced to the evidential objection to religious belief, which states that religious belief is rational only if there is evidence for religious belief, and some brief responses and replies to this evidential objection.

\textbf{Faith and Rationality Unit 2 Topic 4} Video (19 min 01 sec)

\hypertarget{topic-reading-7}{%
\subsection*{Topic Reading}\label{topic-reading-7}}
\addcontentsline{toc}{subsection}{Topic Reading}

\begin{itemize}
\tightlist
\item
  WK Clifford - \href{assets/u2/PHIL-100-Clifford-Ethics-of-Belief.pdf}{\emph{Ethics of Belief}}
\end{itemize}

\hypertarget{note-taking-exercise-7}{%
\subsection*{Note-Taking Exercise}\label{note-taking-exercise-7}}
\addcontentsline{toc}{subsection}{Note-Taking Exercise}

\begin{reflect}
Click the Download button on the left after you've completed all the notes for this reading and save the answers to your computer as you will be required to submit your downloaded notes to this reading as part of your Unit 1 Reflection Assignment.
\end{reflect}

\hypertarget{topic-questionnaire-7}{%
\subsection*{Topic Questionnaire}\label{topic-questionnaire-7}}
\addcontentsline{toc}{subsection}{Topic Questionnaire}

\begin{reflect}
\begin{itemize}
\tightlist
\item
  Answering these questions will help you reflect on the topic content and prepare you for your Unit 2 Reflection Assignment.
\item
  Click the Download button on the left after you've completed all the questions for this unit and save the answers to your computer as you will be required to submit the downloaded answers to these questions as part of your Unit 2 Reflection Assignment.
\end{itemize}
\end{reflect}

\hypertarget{optional-video-7}{%
\subsection*{Optional Video}\label{optional-video-7}}
\addcontentsline{toc}{subsection}{Optional Video}

\begin{reflect}
In this 8 minute 38 second optional video, you will learn about further details about the relationship between faith and reason.

\href{https://www.youtube.com/watch?v=MTPHXNMi9tA}{Watch: \emph{Religion: Reason and Faith}}
\end{reflect}

\hypertarget{reason}{%
\chapter{Reason}\label{reason}}

This final unit is about arguments. This unit is valuable because everything we've discussed so far, both about wisdom and faith, is built upon the foundation of arguments. We give arguments for our views about wisdom. We give arguments about issues of faith and rationality. This last unit provides the opportunity for you to learn a basic skill of how to identify an argument, whether the argument is located in a book, and more often on YouTube, a podcast, and in the news. Note that we do not cover how to evaluate arguments in this unit. Evaluating arguments is a skill taught in other philosophy courses.

In topic 1, we learn about the parts of the argument, such as the proper terms used to designate these different parts. If we wish to learn how to identify arguments, we must know what terms to look for. In topic 2, we learn about how to label and structure arguments. Often people do not organize their argument clearly and so we must learn how to reconstruct the argument so that we know the conclusion and the reasons for supporting the conclusion. In topic 3, we tackle slightly more complicated arguments and learn how to interpret what the author intended to say. As practiced in topic 2, people often organize their arguments (or, occasionally, non-arguments) in a confusing way. Moreover, we will learn about tricky terms often used in arguments, such as the terms ``if'' and ``then''. In the final topic, we practice identifying difficult arguments, which often include long texts, run on sentences, and additional irrelevant content to the argument being put forward. Our task in the final topic is to use our skills practiced in topics 1-3 to identify more difficult arguments.

\textbf{Reason Unit 3 Introduction} Video (5 min 41 sec)

\hypertarget{overview-2}{%
\section*{Overview}\label{overview-2}}
\addcontentsline{toc}{section}{Overview}

\hypertarget{topics-2}{%
\subsection*{Topics}\label{topics-2}}
\addcontentsline{toc}{subsection}{Topics}

This unit is divided into the following four topics:

\begin{enumerate}
\def\labelenumi{\arabic{enumi}.}
\tightlist
\item
  Parts and Keywords of an Argument
\item
  Labelling and Restructuring
\item
  Interpreting Arguments and Conditionals (If\ldots Then)
\item
  Identifying and Practicing Difficult Arguments
\end{enumerate}

\hypertarget{learning-outcomes-2}{%
\subsection*{Learning Outcomes}\label{learning-outcomes-2}}
\addcontentsline{toc}{subsection}{Learning Outcomes}

When you've completed this unit, you will have learned how to:

\begin{itemize}
\tightlist
\item
  Identify the parts of an argument and how to label and structure these parts
\item
  Interpret arguments with ambiguous terminology to be made more precise
\item
  Develop a greater understanding of how arguments work in contemporary culture, including issues of justice and faith
\item
  Practice the skills necessary for navigating arguments in literature, news, and podcasts
\item
  Appreciate the value and skill of knowing how to handle the basic features of an argument
\end{itemize}

\hypertarget{parts-and-keywords-of-an-argument}{%
\section*{3.1 Parts and Keywords of an Argument}\label{parts-and-keywords-of-an-argument}}
\addcontentsline{toc}{section}{3.1 Parts and Keywords of an Argument}

Follow and complete the steps below to accomplish your learning for this topic.

\begin{enumerate}
\def\labelenumi{\arabic{enumi}.}
\tightlist
\item
  Read Topic Notes
\item
  Watch Topic Video
\item
  Complete Logic Exercise 1
\item
  Complete Logic Exercise 2
\item
  Watch Optional Video
\end{enumerate}

\hypertarget{topic-notes-8}{%
\subsection*{Topic Notes}\label{topic-notes-8}}
\addcontentsline{toc}{subsection}{Topic Notes}

In this topic, we learn about the parts of the argument and the words often used to designate these parts. Arguments are made up of statements. Statements are claims that are either true or false. Sometimes statements are whole sentences, while other times statements are phrases. For example, the question, ``Is there a God?'' is not a statement because the question does not state something that is either true or false. The answer to this question, ``yes there is a God,'' or ``no, there is no God'' are statements because they express claims that are either true or false. Other parts of the argument are the premises and conclusion. The premises are the statements that support the conclusion. The conclusion is the statement that is being argued for by the premises. The inference is the move from the premises to the conclusion. Imagine someone says they believe that God exists. This is their conclusion. You ask them, ``why think that God exists?'' Their response is going to be a reason (i.e.~a premise) or a set of reasons (i.e.~a set of premises) that are supposed to support their conclusion. The inference is the nature of the support from premise to conclusion.

Some of the primary keywords of an argument are terms that describe premises and conclusions. For example, the word ``therefore'' often picks out a conclusion. The statement that follows from the term ``therefore'' may act as a conclusion. Terms that describe premise can be ``all'' ``every'' ``some''. As we shall see, these types of words help us identify the premises of an argument.

\hypertarget{topic-video-8}{%
\subsection*{Topic Video}\label{topic-video-8}}
\addcontentsline{toc}{subsection}{Topic Video}

In this video, you will learn about the parts and keywords of an argument. The parts of an argument include statements and never include non-statements. A statement is a claim that is either true or false. Statements in an argument can be premises and/or conclusions. There are often keywords that designate premises and conclusions, such as ``all'' and ``therefore'', respectively.

\textbf{Parts and Keywords of an Argument Unit 3 Topic 1} Video (6 min 46 sec)

\hypertarget{logic-exercise-1}{%
\subsection*{Logic Exercise 1}\label{logic-exercise-1}}
\addcontentsline{toc}{subsection}{Logic Exercise 1}

\begin{reflect}
In this activity, you will learn about the parts of an argument, including statements and non-statements, premises, and conclusions.

An argument is a set of statements organized in a specific way, for example:

\begin{enumerate}
\def\labelenumi{\arabic{enumi}.}
\tightlist
\item
  All humans are mortal. (Premise)
\item
  Lucy is human. (Premise)
\item
  Therefore, Lucy is mortal. (Conclusion)
\end{enumerate}

The statements above are organized as an argument because one of the statements, the conclusion, is supported by the other set of statements, the premises. By reading these premises and the conclusion carefully, one can see that the premises support the truth of the conclusion. \textbf{For our purposes, the order of the premises does not matter}. For example, the following argument is the same as the argument above:

\begin{enumerate}
\def\labelenumi{\arabic{enumi}.}
\tightlist
\item
  Lucy is human.
\item
  All humans are mortal.
\item
  Therefore, Lucy is mortal.
\end{enumerate}

The space between the premises and the conclusion is called the \textbf{inference}. The term ``inference'' refers to the relationship between premises and conclusion, and that relationship changes depending on the type of argument in use. Put in a different way, the inference refers to how one moves from premise to conclusion. Sometimes that inference is one of certainty, in that the truth of the premises guarantees the truth of the conclusion; other times that inference is one of probability, in that the truth of the premises raises the probable truth of the conclusion. \textbf{(You are not required to know about the different types of inferences in this class)}.

The \textbf{premises} are the ``reasons'' for why one should think the \textbf{conclusion} is true, and therefore the conclusion is being defended with those premises. The premises and conclusion will always be described with \textbf{statements}, and statements are claims that are \textbf{either true or false}. Non-statements, however, do not express claims that are either true or false, and thus non-statements are never premises or conclusions. Knowing the difference between statements and non-statements is important because that knowledge will help with identifying the parts of the argument. For example:

\begin{enumerate}
\def\labelenumi{\arabic{enumi}.}
\tightlist
\item
  Apples are red.
\item
  Tony Stark is Iron Man.
\item
  Was that a magic trick?
\item
  Hey, partner!
\end{enumerate}

The first two examples are statements because they express claims that are either true or false. The third example is a question and does not express a claim that is either true or false. Therefore, the third example is a non-statement. (Of course, if one answered the question by saying, ``that is a magic trick,'' then that answer to the question is a statement that is either true or false). The fourth example is an exclamatory greeting and does not express a claim that is either true or false. Therefore, the fourth example is a non-statement. Remember, then, that arguments are made up of statements and never made up of non-statements.

Note that a \textbf{statement} is different from a \textbf{sentence}. A sentence has a subject and a predicate. The subject is what or whom the sentence is about, and the predicate is a description of the subject and contains the verb. Sometimes a whole sentence is one statement, for example:

\begin{quote}
``The cat is in the hat.''
\end{quote}

That entire sentence is a statement because the sentence is either true or false. Other times a whole sentence includes multiple statements, for example:

\begin{quote}
``If the cat is in the hat, then the fish is in the bowl.''
\end{quote}

In that example, the first part, ``the cat is in the hat,'' is the first statement, and the second part, ``the fish is in the bowl,'' is the second statement. So even though that example is one sentence, the entire sentence includes multiple statements that are either true or false.

More difficult sentences may include both statements and non-statements, for example:

\begin{quote}
``Stop, the train is approaching.''
\end{quote}

In that example, the first part, ``Stop,'' is a non-statement, and the second part, ``the train is approaching,'' is a statement. The reason is because the word ``Stop'' does not express a claim that is either true or false, and the phrase ``the train is approaching'' does express a claim that is either true or false. \textbf{When identifying the premises and conclusion of an argument, one should only focus on statements and always ignore non-statements.}

\hypertarget{identifying-statements-and-non-statement}{%
\subsection*{Identifying Statements and Non-statement}\label{identifying-statements-and-non-statement}}
\addcontentsline{toc}{subsection}{Identifying Statements and Non-statement}

\textbf{Which of the following are statements? Which of the following are non-statements?. Identify the statements and non-statements by dragging down the appropriate statement and non-statement to the corresponding box below.}
\end{reflect}

\hypertarget{logic-exercise-2}{%
\subsection*{Logic Exercise 2}\label{logic-exercise-2}}
\addcontentsline{toc}{subsection}{Logic Exercise 2}

\begin{reflect}
In this activity, you will learn about the keywords of an argument, including keywords that designate the premises and the conclusions of an argument.

Some arguments are difficult to locate because the premises and conclusion remain unclear. In those difficult cases, one strategy for identifying the argument is locating \textbf{keywords} indicating the premises and conclusion. Keywords that indicate premises include: \textbf{``All, every, most, some, none, for, because, If\ldots then.''} Keywords that indicate the conclusion are: \textbf{``therefore, so, thus, hence, consequently, it follows that\ldots{}''} Consider the following example:

\begin{enumerate}
\def\labelenumi{\arabic{enumi}.}
\tightlist
\item
  \textbf{All} humans are mortal. (Premise)
\item
  Lucy is a human. (Premise)
\item
  \textbf{Therefore}, Lucy is mortal. (Conclusion)
\end{enumerate}

In that example, the first premise follows the term ``All,'' and the conclusion follows the term ``Therefore.'' The second premise does not have a keyword, but, in this case, the reader knows that statement is not the conclusion -- because the conclusion was identified already with the keyword ``therefore'' -- and so the second statement must be a premise (this isn't always the case, but we needn't be concerned about that here). The argument is easy to identify because both the premise and conclusion have keywords.

\hypertarget{identifying-premises-and-conclusions}{%
\subsection*{Identifying Premises and Conclusions}\label{identifying-premises-and-conclusions}}
\addcontentsline{toc}{subsection}{Identifying Premises and Conclusions}

\textbf{Identify which of the following are premises and which are conclusions by clicking the correct answer.}

\textbf{Checklist}

\begin{itemize}
\tightlist
\item
  Focus on statements and ignore non-statements
\item
  Locate the premises and conclusion using keywords
\end{itemize}
\end{reflect}

\hypertarget{optional-video-8}{%
\subsection*{Optional Video}\label{optional-video-8}}
\addcontentsline{toc}{subsection}{Optional Video}

\begin{reflect}
In this 5 minute 24 second optional video, you will learn about the basic structure of an argument.

\href{https://www.youtube.com/watch?v=wbRxR53F3rI}{Watch: \emph{Critical Thinking \#3: Types of Arguments}}
\end{reflect}

\hypertarget{labelling-and-restructuring}{%
\section*{3.2 Labelling and Restructuring}\label{labelling-and-restructuring}}
\addcontentsline{toc}{section}{3.2 Labelling and Restructuring}

Follow and complete the steps below to accomplish your learning for this topic.

\begin{enumerate}
\def\labelenumi{\arabic{enumi}.}
\tightlist
\item
  Read Topic Notes
\item
  Watch Topic Video
\item
  Complete Logic Exercise 1
\item
  Complete Logic Exercise 2
\item
  Watch Optional Video
\end{enumerate}

\hypertarget{topic-notes-9}{%
\subsection*{Topic Notes}\label{topic-notes-9}}
\addcontentsline{toc}{subsection}{Topic Notes}

In the previous topic, we learned about the parts and keywords of an argument. In this topic, we learn how to label these parts of the argument and then restructure the argument vertically, with premises and conclusion, so that we may see the argument clearly. The steps for applying these tools follow this pattern: (i) focus on statements and ignore non-statements. Recall from the previous topic that arguments are made up of statements only and never non-statements. Thus, only focus on statements and cross out any non-statements. (ii) Locate the premises and conclusion amongst the set of statements using keywords. As discussed, premises and conclusions often have terms that designate them, such as ``therefore''.

Once the argument has been labelled, the next step is to restructure the argument vertically with the premises on top and the conclusion beneath. This way we can see what the author's argument is exactly. That is, we can see the conclusion -- the main point of their position -- and the reasons (i.e.~premises) for thinking the conclusion is true. Once these statements are ordered correctly, anyone can see the argument. The reader needn't labour over the text and the many confusing non-statements to locate the point of the passage.

\hypertarget{topic-video-9}{%
\subsection*{Topic Video}\label{topic-video-9}}
\addcontentsline{toc}{subsection}{Topic Video}

In this video, you will learn about labelling and restructuring arguments. Labelling an argument involves inserting the letter (p) next to the premises and the letter (c) next to the conclusion. You then restructure the argument vertically with the premises on top and the conclusion beneath.

\textbf{Labelling and Restructuring Unit 3 Topic 2} Video (8 min 56 sec)

\hypertarget{logic-exercise-1-1}{%
\subsection*{Logic Exercise 1}\label{logic-exercise-1-1}}
\addcontentsline{toc}{subsection}{Logic Exercise 1}

\begin{reflect}
In this activity, you will learn how to label the parts of the argument by adding the letters (p) and (c) in front of the premise(s) and conclusion(s). For example:

\begin{quote}
``All humans are mortal, and Lucy is a human. So, Lucy is mortal.''
\end{quote}

Unlike the examples of arguments in the topics 1 and 2, notice that this argument is not written vertically but horizontally. Most arguments written in ordinary discourse (e.g.~the news, Facebook, etc.) will be expressed horizontally. That horizontal format already makes the argument more difficult to locate. The first sentence includes two statements and both statements are different premises of the argument. The conclusion is the second sentence indicated by the keyword ``So.'' The objective is to label the premises and conclusion with symbols so that the reader can identify the argument more accurately. In doing so, begin by first labeling the premises and conclusion by writing the letter ``p'' in parentheses at the beginning of the premises, and the letter ``c'' in parentheses next to the conclusion. For example:

\begin{quote}
\textbf{(P)} All humans are mortal, and \textbf{(P)} Lucy is a human. \textbf{(C)} So, Lucy is mortal.
\end{quote}

In that example, ``P'' stands for premise and (C) means conclusion. \textbf{Note again: the order of the premises does not matter in this Unit. One can reverse the order of the premises and produce the same argument}. When reading more complex arguments, labeling the premises and conclusion becomes essential to be sure one correctly identifies the argument.

\hypertarget{label-the-premises-and-conclusions}{%
\subsection*{Label the Premises and Conclusions}\label{label-the-premises-and-conclusions}}
\addcontentsline{toc}{subsection}{Label the Premises and Conclusions}

\textbf{Identify which of the following are premises and which are conclusions. Label by filling the blanks using the appropriate letters.}

\textbf{Checklist}

\begin{itemize}
\tightlist
\item
  Focus on statements and ignore non-statements
\item
  Locate the premises and conclusion using keywords
\item
  Label the premises and conclusion by inserting the appropriate letter in the blank in front of the statements: P for premise, and C for Conclusion. Be sure to ignore non-statements by inserting an ``X'' in the blank in front of the non-statement.
\end{itemize}

\textbf{Example}

Argument:
\textgreater{} ``All conservatives are Christian. Charlie is conservative. Therefore, Charlie is Christian.''

Solution:
\textgreater{} (P) All conservatives are Christian. (P) Charlie is conservative. (C) Therefore, Charlie is Christian.
\end{reflect}

\hypertarget{logic-exercise-2-1}{%
\subsection*{Logic Exercise 2}\label{logic-exercise-2-1}}
\addcontentsline{toc}{subsection}{Logic Exercise 2}

\begin{reflect}
As illustrated in the previous lessons and activities, notice that writing an argument in vertical form, with premises on top and the conclusion on the bottom, is the most effective way to illustrate the final argument. For example:

\begin{enumerate}
\def\labelenumi{\arabic{enumi}.}
\tightlist
\item
  All humans are mortal. (P)
\item
  Lucy is a human. (P)
\item
  Therefore, Lucy is mortal. (C)
\end{enumerate}

That vertical format is valuable for the reader to know exactly what the argument is about. Unfortunately, arguments in ordinary discourse are not written in that vertical fashion - as illustrated in Topic 3, where arguments were written horizontally - and therefore one must restructure the horizontal argument format to a vertical argument format. In Topic 4, we practice all of the methods above with the aim of restructuring the argument vertically.

\hypertarget{label-and-structure-the-argument}{%
\subsection*{Label and Structure the Argument}\label{label-and-structure-the-argument}}
\addcontentsline{toc}{subsection}{Label and Structure the Argument}

\textbf{This activity has two steps. The first step is to click/drag the appropriate letter into the box to label the premises and conclusion. In the second step, click/drag the appropriate premise and conclusion to complete the vertical argument.}

\textbf{Checklist}

\begin{itemize}
\tightlist
\item
  Focus on statements and ignore non-statements
\item
  Locate the premises and conclusion using keywords
\item
  Label the premises and conclusion using letters in parentheses
\item
  Restructure the argument vertically, with premises on top and the conclusion on the bottom
\end{itemize}

For example:

\begin{quote}
``Lucy is a teacher. All teachers drink coffee. Therefore, Lucy drinks coffee.''
\end{quote}

Step 1: Label the argument

\begin{quote}
\begin{enumerate}
\def\labelenumi{(\Alph{enumi})}
\setcounter{enumi}{15}
\tightlist
\item
  Lucy is a teacher. (P) All teachers drink coffee. (C) Therefore, Lucy drinks coffee.
\end{enumerate}
\end{quote}

Step 2: Restructure the argument vertically

\begin{enumerate}
\def\labelenumi{\arabic{enumi}.}
\tightlist
\item
  Lucy is a teacher. (P)
\item
  All teachers drink coffee. (P)
\item
  Therefore, Lucy drinks coffee. (C)
\end{enumerate}

\hypertarget{practice-exercises}{%
\subsubsection*{Practice Exercises}\label{practice-exercises}}
\addcontentsline{toc}{subsubsection}{Practice Exercises}
\end{reflect}

\hypertarget{optional-video-9}{%
\subsection*{Optional Video}\label{optional-video-9}}
\addcontentsline{toc}{subsection}{Optional Video}

\begin{reflect}
In this 1 minute 22 second optional video, you will learn about the basic structure of an argument, with brief attention to what makes a bad argument.

\href{https://www.youtube.com/watch?v=Z7f_uuy1JcM}{Watch: \emph{Ninety Second Philosophy: Arguments}}
\end{reflect}

\hypertarget{interpreting-arguments-and-conditionals-ifthen}{%
\section*{3.3 Interpreting Arguments and Conditionals (If\ldots Then)}\label{interpreting-arguments-and-conditionals-ifthen}}
\addcontentsline{toc}{section}{3.3 Interpreting Arguments and Conditionals (If\ldots Then)}

Follow and complete the steps below to accomplish your learning for this topic.

\begin{enumerate}
\def\labelenumi{\arabic{enumi}.}
\tightlist
\item
  Read Topic Notes
\item
  Watch Topic Video
\item
  Complete Logic Exercise 1
\item
  Complete Logic Exercise 2
\item
  Watch Optional Video
\end{enumerate}

\hypertarget{topic-notes-10}{%
\subsection*{Topic Notes}\label{topic-notes-10}}
\addcontentsline{toc}{subsection}{Topic Notes}

In the first two topics, we learned some foundational skills for identifying arguments, such as the parts and keywords of an argument, and the methods for labeling and restructuring arguments. The challenge, however, is that ordinary discourse -- both written and verbal -- often makes the argument difficult to identify. We often must interpret what we think the author intended to say about their view and then structure the argument according to this intention. In this topic, we practice the skill of interpretating arguments. One strategy for interpreting an argument is the ``Why/Because'' strategy. If it's difficult to identify the premises and conclusion, then look for statements that may follow with a hypothetical ``why''. Such statements may be the conclusion(s). The reason is that concluding statements have by definition reasons for them. And the term ``why'' may pick out these reasons. The term ``because'' often precedes a reason (i.e.~a premise) and so looking for statements that may be preceded with ``because'' may help with identifying the premises.

A second skill of interpretation is learning to navigate conditional statements. A conditional is a ``if \ldots{} then'' statement. For example, ``If it's raining outside, then we can't go to the park.'' Sentences with the words ``if'' and ``then'' may be interpreted in different ways depending on the order of these words. In this topic we practice some of these interpretive skills.

\hypertarget{topic-video-10}{%
\subsection*{Topic Video}\label{topic-video-10}}
\addcontentsline{toc}{subsection}{Topic Video}

In this video, you will learn some strategies for interpreting arguments. Interpretation is important because we often we fail to clearly communicate the argument. One strategy for interpretation is called the ``why/because'' strategy. This strategy involves looking for statements that could be followed with a hypothetical ``why'', and statements that precede with a hypothetical ``because''. In addition, you will learn how to interpret conditional sentences with the term ``if'' and ``then''.

\textbf{Interpreting Arguments and Conditionals (If\ldots Then) Unit 3 Topic 3} Video (15 min 32 sec)

\hypertarget{logic-exercise-1-2}{%
\subsection*{Logic Exercise 1}\label{logic-exercise-1-2}}
\addcontentsline{toc}{subsection}{Logic Exercise 1}

\begin{reflect}
Another challenge with arguments written in ordinary discourse is not only that they're written horizontally but they lack keywords for identifying premises and conclusions. When there are no keywords for identifying premise and conclusion, the reader must interpret what the author intended to say and then determine which statements are the premises and conclusion. Consider two examples, the first simpler and the second more complex. The first example is the following:

\begin{quote}
``I think Lucy is mortal. She is human and humans are mortal.''
\end{quote}

This example does not include a keyword for the conclusion but rather uses the phrase ``I think.'' Furthermore, note that the conclusion comes first, ``I think Lucy is mortal.'' That informal way of writing the conclusion makes identification more challenging. Premise (1) also includes the pronoun ``she'', which in this case refers to ``Lucy.'' Note also that premises (1) and (2) are statements within the same sentence, respectively, and lack keywords for identification. In most situations, arguments resemble this example in that they are unclear, and one must interpret the argument according to what the author intended to say. If we apply the steps from the previous examples, first label the premises and conclusion with letters:

\begin{enumerate}
\def\labelenumi{(\Alph{enumi})}
\setcounter{enumi}{2}
\tightlist
\item
  I think that Lucy is mortal. (P) She is human, and (P) humans are mortal.
\end{enumerate}

Now that the argument is labeled, order the argument vertically with abbreviations in parentheses, and then \textbf{observe} the added keywords that provide a clear interpretation. For example:

\begin{enumerate}
\def\labelenumi{\arabic{enumi}.}
\tightlist
\item
  \textbf{Lucy} is human. (P)
\item
  \textbf{All} humans are mortal. (P)
\item
  \textbf{Therefore}, Lucy is mortal. (C)
\end{enumerate}

Now the final version of that argument is clearer. We ordered the premises and conclusion vertically, replaced the pronoun ``she'' with ``Lucy,'' added the keyword ``All'' to premise (2), on the assumption that the author intended to refer to ``all'' humans and not ``some'' humans, and added the conclusion keyword ``Therefore.''

The second more complex example is the following:
\textgreater{} ``Dogs must be man's best friend.'' Dogs are loyal and kind.''

In this example, there are no relevant keywords, and the conclusion may not be obvious. The task, then, is to carefully interpret the sentences and their relationship with each other to discover (1) does an argument even exist? and (2), if yes, then what are the premises and the conclusion?

Identifying the argument can be difficult. One technique for identifying obscure arguments like the one above is using the ``Why/Because'' strategy. A conclusion can always be followed by asking a hypothetical ``Why?'' and a premise can always be introduced with the term ``Because.'' Consider the following:

\begin{quote}
``Dogs must be man's best friend. (\textbf{Why})?\ldots(\textbf{Because}) Dogs are loyal and (\textbf{Because}) dogs are kind.''
\end{quote}

The first statement, ``Dogs must be man's best friend,'' can be followed by asking ``\textbf{Why} do you think that claim is true?'' The second and third statements can be introduced with the term ``because''. This process demonstrates that the first statement, followed by the term ``why,'' is the conclusion, and the second and third statements, preceded with the term ``because,'' are the premises. So, in difficult cases with a lack of keywords, locate the statements and note which statement best follows with a ``Why?'' (likely the conclusion), and the statements that best precede with a ``Because'' (likely the premises supporting that conclusion).

However, at times that strategy may create confusion because most statements could in principle be followed with a hypothetical ``Why?'' For example:

\begin{quote}
``Dogs must be man's best friend. Dogs are loyal \textbf{(Why?)} and dogs are kind.''
\end{quote}

In that example, the statement followed by the hypothetical ``why?'' is not the conclusion, but one could technically apply the strategy in that case, thereby creating confusion. How does one resolve this confusion? One strategy is to answer the ``Why'' question with the other statements involved, and test and compare each statement with another until the conclusion becomes more apparent. For example, suppose one wishes to test the statement, ``Dogs are loyal,'' to discover if that statement is the conclusion:

\begin{quote}
``Dogs are loyal.'' \textbf{Why?}
\end{quote}

Answer that question by using only the other two statements involved, for example:

\begin{enumerate}
\def\labelenumi{\arabic{enumi}.}
\tightlist
\item
  Dogs are loyal. \textbf{Why? Because} dogs must be man's best friend.
\item
  Dogs are loyal. \textbf{Why? Because} dogs are kind.
\end{enumerate}

Neither answer to the question seems accurate, because dog loyalty has to do with other characteristics absent from those statements. That process indicates that the statement, ``dogs are loyal,'' is likely not the conclusion. Repeat the process with the other two statements until a conclusion becomes evident. If there's no apparent conclusion, then that may show the argument is poorly constructed.

A final comment is the terms ``why'' and ``because'' may be used in sentences without invoking an argument. For example,

\begin{quote}
``I'm at the birthday party because I was invited.''
\end{quote}

Those statements do not (really) constitute an argument even though the term ``because'' is used and the term ``why'' is implied.

\hypertarget{label-and-structure-the-arguments-vertically}{%
\subsection*{Label and Structure the Arguments Vertically}\label{label-and-structure-the-arguments-vertically}}
\addcontentsline{toc}{subsection}{Label and Structure the Arguments Vertically}

\textbf{This activity has two steps. The first step is to click/drag the appropriate letter into the box to label the premises and conclusion. In the second step, click/drag the appropriate premise and conclusion to complete the vertical argument.}

\textbf{Checklist}

\begin{itemize}
\tightlist
\item
  Focus on statements and ignore non-statements.
\item
  Locate the premises and conclusion using keywords. \textbf{NOTE: use the ``Why/Because'' strategy for difficult cases.}
\item
  Label the premises and conclusion using letters in parentheses.
\item
  Restructure the argument vertically, with premises on top and the conclusion on the bottom.
\item
  \textbf{Note the keywords that are added to the premises and conclusions in step 2.}
\end{itemize}

Example:

\begin{quote}
``I think all children are artists. Marcie is a child. Marcie must be an artist.''
\end{quote}

\textbf{Step 1:} Label the arguments

\begin{quote}
\begin{enumerate}
\def\labelenumi{(\Alph{enumi})}
\setcounter{enumi}{15}
\tightlist
\item
  I think all children are artists. (P) Marcie is a child. (C) Marcie must be an artist.
\end{enumerate}
\end{quote}

\textbf{Step 2:} Restructure the argument vertically

\begin{enumerate}
\def\labelenumi{\arabic{enumi}.}
\tightlist
\item
  \textbf{All} children are artists. (P)
\item
  Marcie is a child. (P)
\item
  \textbf{Therefore}, Marcie is an artist. (C)
\end{enumerate}
\end{reflect}

\hypertarget{logic-exercise-2-2}{%
\subsection*{Logic Exercise 2}\label{logic-exercise-2-2}}
\addcontentsline{toc}{subsection}{Logic Exercise 2}

\begin{reflect}
Some arguments use sentences with the terms ``If\ldots Then.'' These types of sentences are called conditional sentences (or conditionals) and often appear in arguments. For example:

\begin{quote}
``\textbf{IF} Willy is a whale, \textbf{THEN} Willy is a mammal.''
\end{quote}

As discussed briefly in an earlier section, note that some sentences may include multiple statements and thus include multiple premises. However, note that conditional sentences, while they may include multiple statements, \textbf{always constitute one premise}. That is, both ``If\ldots Then'' statements in a conditional sentence are not different premises of the argument but are parts of the same premise.

For example:
1. \textbf{If Willy is a whale, then Willy is a mammal.} (P)
2. Willy is a whale. (P2)
3. Therefore, Willy is a mammal. (C1)

\textbf{The first premise includes both ``If\ldots Then'' statements}, and both statements in the conditional can be defined. The first statement following the term ``If'' is called the \textbf{antecedent}. That means the thing that comes before. The second statement following the term ``Then'' is called the \textbf{consequent}. That means the thing that comes after, or the thing that follows from the antecedent, for example:

\begin{quote}
``If Willy is a whale (antecedent), then Willy is a mammal (consequent).''
\end{quote}

Conditional sentences are tricky business. While we needn't be concerned with those difficulties in this Unit, three factors deserve attention for the purpose of identifying arguments: (i) notice that the terms ``if'\,' and ``then'' are keywords that may indicate a conditional premise; (ii), notice the conclusion in the argument above, ``Willy is a mammal,'' is smuggled into the second part (the consequent) of the first premise. When identifying a conditional sentence, the consequent of that conditional may be a clue for locating the conclusion (not always, but sometimes); and (iii), conditional sentences can be expressed in English in different ways. So one must learn to identify some of these expressions and practice how to convert them into standard ``If\ldots then'' form.

\textbf{a. Sentences with the term ``if'' in the middle}

\begin{quote}
``Willy is a mammal \textbf{IF} Willy is a whale.''
\end{quote}

\textbf{The term ``if'' introduces the antecedent, and the antecedent always comes first in the conditional sentence}. So, rewrite the sentence above to standard conditional form by moving the term ``if'' and the antecedent, ``Willy is a whale,'' to the beginning of the sentence, and add the term ``then'' to the beginning of the consequent, ``Willy is a mammal'':

\begin{quote}
``\textbf{If} Willy is a whale, \textbf{then} Willy is a mammal.''
\end{quote}

Now the argument can be structured vertically, for example:
1. If Willy is a whale, then Willy is a mammal. (P)
2. Willy is a whale. (P)
3. Therefore, Willy is a mammal. (C)

\textbf{b. Sentences with the phrase ``only if'' at the beginning}
\textgreater{} ``\textbf{Only if} Willy is a mammal, is Willy a whale.''

\textbf{The term ``only if'' introduces the consequent, and the consequent is always located at the end of the conditional sentence}. So, change the term ``only if'' to ``then'' and move its consequent, ``Willy is a mammal,'' to the end. Then move the last statement, ``Willy is a whale,'' to the beginning with the term ``If'' added in front because that statement is the antecedent.

\begin{quote}
``If Willy is a whale, then Willy is a mammal.''
\end{quote}

\textbf{c.~Sentences with the phrase ``only if'' in the middle:}
\textgreater{} ``Willy is a whale only if Willy is a mammal.''

Note, again, that the phrase ``\textbf{only if}'' introduces the consequent. Rewrite the sentence by converting ``only if'' to ``then,'' and add the term ``If'' to the beginning.

\begin{quote}
``If Willy is a whale, then Willy is a mammal.''
\end{quote}

\textbf{d.~Sentences with the term ``unless'' at the beginning}
\textgreater{} ``\textbf{Unless} you eat vegetables, there's no dessert.''

\textbf{The term ``unless'' introduces the antecedent with a negation} (i.e.~``not'') and is translated to ``\textbf{if not.}'' So, rewrite by replacing ``unless'' with ``if not'' and add the term ``then'' to the beginning of the consequent.

``\textbf{If not} you eat vegetables, \textbf{then} there's no dessert.''

Obviously, that sentence above, although logically correct, must be rewritten to make sense:
\textgreater{} ``\textbf{If} you do \textbf{not} eat vegetables, then there's no dessert.''

\textbf{e. Sentences with the term ``unless'' in the middle}
\textgreater{} ``There's no dessert UNLESS you eat vegetables.''

Rewrite by changing ``unless'' to ``if not,'' move ``if not'' and its statement, ``you eat vegetables,'' to the beginning to make the antecedent; add ``then'' to the second statement, ``there's no dessert,'' to make the consequent.

\begin{quote}
``\textbf{If} you do not eat vegetables, \textbf{then} there's no dessert.''
\end{quote}

\textbf{f.~Sentences with the phrase ``provided that''}
\textgreater{} ``You may eat dessert \textbf{provided that} you eat vegetables.''

\textbf{The phrase ``provided that'' introduces the antecedent}. Rewrite by changing ``provided that'' to ``if'' and move ``if'' and its statement, ``you eat vegetables,'' to the beginning to make the antecedent; add the term ``then'' to the beginning of the other statement, ``You may eat dessert,'' to make the consequent, for example:

\begin{quote}
``\textbf{If} you eat vegetables, \textbf{then} you may eat dessert.''
\end{quote}

Now you know how to identify different ways of expressing ``If\ldots then,'' and how to rewrite those expressions to simplify ordering the argument vertically.

\textbf{Conditional Sentences Cheat Sheet}

\begin{enumerate}
\def\labelenumi{\arabic{enumi}.}
\tightlist
\item
  Antecedent: the first statement in a conditional sentence introduced with the term ``If''
\item
  Consequent: the second statement in a conditional sentence introduced with the term ``then''
\item
  ``If'' = always introduces the antecedent, regardless of location
\item
  ``Only if'' = always introduces the consequent, regardless of location
\item
  ``Unless'' = always introduces the antecedent with a negation, regardless of location, translated as ``if not''
\item
  ``Provided that'' = always introduces the antecedent regardless of location
\end{enumerate}

\hypertarget{arguments-with-conditionals}{%
\subsection*{Arguments with Conditionals}\label{arguments-with-conditionals}}
\addcontentsline{toc}{subsection}{Arguments with Conditionals}

\textbf{This activity has two steps. The first step is to click/drag the appropriate letter into the box to label the premises and conclusion. In the second step, click/drag the appropriate premise and conclusion to complete the vertical argument.}

\textbf{Checklist}

\begin{itemize}
\tightlist
\item
  Focus on statements and ignore non-statements.
\item
  Locate the premises and conclusion using keywords. \textbf{NOTE: the terms ``if'' and ``then'' may be keywords for indicating a conditional sentence and a premise of an argument. NOTE: expressions such as ``only if'' and ``unless'' (in green) indicate conditionals. Note how those expressions (in green) will change to standard ``If/Then'' conditionals when ordering the argument vertically.}
\item
  Label the premises and conclusion using letters in parentheses.
\item
  Restructure the argument vertically, with premises on top and the conclusion beneath.
\item
  \textbf{Notice the changes to the wording in some of the premises and conclusions.}
\end{itemize}

Example 1:
\textgreater{} ``If that thing walks like a duck, then that thing is a duck. That thing walks like a duck. Therefore, that thing is a duck.

\textbf{Step 1:} Label the argument

\begin{quote}
\begin{enumerate}
\def\labelenumi{(\Alph{enumi})}
\setcounter{enumi}{15}
\tightlist
\item
  If that thing walks like a duck, then that thing is a duck. (P) That thing walks like a duck. (C) Therefore, that thing is a duck.
\end{enumerate}
\end{quote}

\textbf{Step 2:} Restructure the argument vertically
1. If that animal walks like a duck, then that animal is a duck.
2. That animal walks like a duck.
3. Therefore, that animal is a duck.

Example 2:
\textgreater{} ``I'll marry you only if you watch Downton Abbey. What!? You refuse to watch Downton Abbey. You're kidding me! Well, I refuse to marry you.''

\textbf{Step 1:} Label the argument (notice the change from ``only if'' to ``if then'')

\begin{quote}
\begin{enumerate}
\def\labelenumi{(\Alph{enumi})}
\setcounter{enumi}{15}
\tightlist
\item
  If I marry you, then you must watch Downton Abbey. (P) You refuse to watch Downton Abbey. (C) I will not marry you.
\end{enumerate}
\end{quote}

\textbf{Step 2:} Restructure the argument vertically
1. If I marry you, then you must watch Downton Abbey
2. You refuse to watch Downton Abbey
3. Thus, I will not marry you.
\end{reflect}

\hypertarget{optional-video-10}{%
\subsection*{Optional Video}\label{optional-video-10}}
\addcontentsline{toc}{subsection}{Optional Video}

\begin{reflect}
In this 6 minute optional video, you will be introduced to some further techniques for identifying arguments.

\href{https://www.youtube.com/watch?v=lYiEj5z8le8}{Watch: \emph{Lesson 2. Identifying Arguments}}
\end{reflect}

\hypertarget{identifying-and-practicing-difficult-arguments}{%
\section*{3.4 Identifying and Practicing Difficult Arguments}\label{identifying-and-practicing-difficult-arguments}}
\addcontentsline{toc}{section}{3.4 Identifying and Practicing Difficult Arguments}

Follow and complete the steps below to accomplish your learning for this topic.

\begin{enumerate}
\def\labelenumi{\arabic{enumi}.}
\tightlist
\item
  Read Topic Notes
\item
  Watch Topic Video
\item
  Complete Logic Exercise
\item
  Complete Practice Exercises
\item
  Watch Optional Video
\end{enumerate}

\hypertarget{topic-notes-11}{%
\subsection*{Topic Notes}\label{topic-notes-11}}
\addcontentsline{toc}{subsection}{Topic Notes}

In the previous topics, we have attempted to build our foundation for identifying arguments. The real challenge is when we're confronted with long speeches, texts, reports, stories, and accusations whereby the argument may often seem unclear. Our task is to use the tools we've learned to identify the argument(s) in more difficult contexts. In this topic, we practice these skills by following these steps: (i) ignore non-statements and label the premises and conclusion; (ii) restructure the argument vertically with premises on top and the conclusion beneath. Notice that the difficult step is (i). We must sift through the content to locate keywords, interpret ambiguous statements, possibly use the ``Why/Because'' strategy in cases of doubt, decipher unusual ``if/then'' statements, and so on. Sometimes there is no argument at all; other times there may exist more than one argument, such that there is a main argument, and a secondary argument. The point of this entire exercise is to learn the skill that so many of us fail to properly master: to learn what the opposing person is trying to say. What is their position? What is their argument if there is one? Only after we've successfully identified the argument can we then properly raise objections and evaluate their argument.

\hypertarget{topic-video-11}{%
\subsection*{Topic Video}\label{topic-video-11}}
\addcontentsline{toc}{subsection}{Topic Video}

In this video, you will learn how to use the previous tools and strategies to identify difficult arguments. These arguments are often a challenge to locate because they exist within a longer text or speech. These texts and speeches often include irrelevant information and non-statements that make identification difficult.

\textbf{Identifying and Practicing Difficult Arguments Unit 3 Topic 4} Video (8 min 03 sec)

\hypertarget{logic-exercise}{%
\subsection*{Logic Exercise}\label{logic-exercise}}
\addcontentsline{toc}{subsection}{Logic Exercise}

\begin{reflect}
In the previous topics, we learned about identifying arguments. In this final topic, we apply those tools to identifying more difficult arguments. The examples below are more challenging because the arguments are not only greater length, but the arguments can be difficult to identify. One must carefully sift through the statements to decipher (i) does an argument exist, and (ii) if yes, then what is the conclusion(s) and the premises supporting that conclusion(s)? One must employ all the methods to achieve those tasks: such as ignoring non-statements, search for keywords that indicate premises and conclusions (highlight them, if that helps), label the premises and conclusions in parentheses, interpret what the author intended to say, and restructure the argument vertically in its strongest form. Note: more complex arguments often include multiple conclusions and those conclusions may operate as premises of the same argument. Note: when encountering a lengthy passage, read the entire passage through multiple times before attempting to identify the parts of the argument. The reason for that is to gain a broader understanding of what the author intended to say.

\textbf{Checklist}

\begin{itemize}
\tightlist
\item
  Focus on statements and ignore non-statements
\item
  Locate the premises and conclusion using keywords
\item
  Label the premises and conclusion using letters and numerals in parentheses
\item
  Restructure the argument vertically, with premises on top and the conclusion beneath
\end{itemize}

Example 3 is more challenging. In these types of cases, read through the entire passage several times to decipher the argument exactly. Take time to draft multiple arguments vertically and observe which version best represents what the author intended to say.
\end{reflect}

\hypertarget{practice-exercises-1}{%
\subsection*{Practice Exercises}\label{practice-exercises-1}}
\addcontentsline{toc}{subsection}{Practice Exercises}

\begin{reflect}
These practice exercises are optional and not graded. However, the more you practice and become familiar with identifying arguments, the more prepared you will be for your Final Course Reflection Assignment.

If you want to keep your answers for future reference, click the Download button on the left after you've completed the exercise and save the answers to your computer.
\end{reflect}

\hypertarget{optional-video-11}{%
\subsection*{Optional Video}\label{optional-video-11}}
\addcontentsline{toc}{subsection}{Optional Video}

\begin{reflect}
In this 5 minute 34 second optional video, you will be introduced to strategies for locating terms that describe parts of the argument in difficult passages, such as terms that pick out conclusions and premises.

\href{https://www.youtube.com/watch?v=07mehbgE5jc}{Watch: \emph{Identifying Premises and Conclusions}}
\end{reflect}

  \bibliography{book.bib}

\end{document}
