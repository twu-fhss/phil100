% Options for packages loaded elsewhere
\PassOptionsToPackage{unicode}{hyperref}
\PassOptionsToPackage{hyphens}{url}
%
\documentclass[
]{book}
\usepackage{amsmath,amssymb}
\usepackage{iftex}
\ifPDFTeX
  \usepackage[T1]{fontenc}
  \usepackage[utf8]{inputenc}
  \usepackage{textcomp} % provide euro and other symbols
\else % if luatex or xetex
  \usepackage{unicode-math} % this also loads fontspec
  \defaultfontfeatures{Scale=MatchLowercase}
  \defaultfontfeatures[\rmfamily]{Ligatures=TeX,Scale=1}
\fi
\usepackage{lmodern}
\ifPDFTeX\else
  % xetex/luatex font selection
\fi
% Use upquote if available, for straight quotes in verbatim environments
\IfFileExists{upquote.sty}{\usepackage{upquote}}{}
\IfFileExists{microtype.sty}{% use microtype if available
  \usepackage[]{microtype}
  \UseMicrotypeSet[protrusion]{basicmath} % disable protrusion for tt fonts
}{}
\makeatletter
\@ifundefined{KOMAClassName}{% if non-KOMA class
  \IfFileExists{parskip.sty}{%
    \usepackage{parskip}
  }{% else
    \setlength{\parindent}{0pt}
    \setlength{\parskip}{6pt plus 2pt minus 1pt}}
}{% if KOMA class
  \KOMAoptions{parskip=half}}
\makeatother
\usepackage{xcolor}
\usepackage{longtable,booktabs,array}
\usepackage{calc} % for calculating minipage widths
% Correct order of tables after \paragraph or \subparagraph
\usepackage{etoolbox}
\makeatletter
\patchcmd\longtable{\par}{\if@noskipsec\mbox{}\fi\par}{}{}
\makeatother
% Allow footnotes in longtable head/foot
\IfFileExists{footnotehyper.sty}{\usepackage{footnotehyper}}{\usepackage{footnote}}
\makesavenoteenv{longtable}
\usepackage{graphicx}
\makeatletter
\def\maxwidth{\ifdim\Gin@nat@width>\linewidth\linewidth\else\Gin@nat@width\fi}
\def\maxheight{\ifdim\Gin@nat@height>\textheight\textheight\else\Gin@nat@height\fi}
\makeatother
% Scale images if necessary, so that they will not overflow the page
% margins by default, and it is still possible to overwrite the defaults
% using explicit options in \includegraphics[width, height, ...]{}
\setkeys{Gin}{width=\maxwidth,height=\maxheight,keepaspectratio}
% Set default figure placement to htbp
\makeatletter
\def\fps@figure{htbp}
\makeatother
\setlength{\emergencystretch}{3em} % prevent overfull lines
\providecommand{\tightlist}{%
  \setlength{\itemsep}{0pt}\setlength{\parskip}{0pt}}
\setcounter{secnumdepth}{5}
\usepackage{booktabs}
\usepackage{amsthm}
\makeatletter
\def\thm@space@setup{%
  \thm@preskip=8pt plus 2pt minus 4pt
  \thm@postskip=\thm@preskip
}
\makeatother

\usepackage{tcolorbox}


\newtcolorbox{blackbox}{
  colback=black,
  coltext=white,
  colframe=black,
  boxsep=5pt,
  arc=4pt}
\newtcolorbox{bonus}{
  colback=blue!15,
  colframe=blue!15,
  coltext=black!80,
  boxsep=5pt,
  arc=4pt}
\newtcolorbox{reflect}{
  colback=green!5,
  colframe=green!5,
  coltext=black!80,
  boxsep=5pt,
  arc=4pt}
\newtcolorbox{assessment}{
  colback=blue!5,
  colframe=blue!5,
  coltext=black!80,
  boxsep=5pt,
  arc=4pt}
\newtcolorbox{progress}{
  colback=purple!10,
  colframe=purple!10,
  coltext=black!80,
  boxsep=5pt,
  arc=4pt}
\newtcolorbox{video}{
  colback=yellow!5,
  colframe=yellow!5,
  coltext=black!80,
  boxsep=5pt,
  arc=4pt}
\newtcolorbox{caution}{
  colback=red!5,
  colframe=red!5,
  coltext=black!80,
  boxsep=5pt,
  arc=4pt}
\newtcolorbox{feedback}{
  colback=black!5,
  colframe=black!5,
  coltext=black!80,
  boxsep=5pt,
  arc=4pt}
\ifLuaTeX
  \usepackage{selnolig}  % disable illegal ligatures
\fi
\usepackage[]{natbib}
\bibliographystyle{apalike}
\IfFileExists{bookmark.sty}{\usepackage{bookmark}}{\usepackage{hyperref}}
\IfFileExists{xurl.sty}{\usepackage{xurl}}{} % add URL line breaks if available
\urlstyle{same}
\hypersetup{
  pdftitle={Philosophy for Life},
  pdfauthor={Andrew Brigham},
  hidelinks,
  pdfcreator={LaTeX via pandoc}}

\title{Philosophy for Life}
\author{Andrew Brigham}
\date{2023-08-24}

\begin{document}
\maketitle

{
\setcounter{tocdepth}{1}
\tableofcontents
}
\hypertarget{welcome}{%
\chapter*{Welcome}\label{welcome}}
\addcontentsline{toc}{chapter}{Welcome}

This course explores philosophy as a way of life accessible to all, in order to think more truthfully, act more justly, and live more faithfully. This course focuses on critical thinking as an invaluable ethical tool for interpreting current events. Students will learn to analyze and evaluate the claims of contemporary culture and religious faith.

\hypertarget{course-learning-outcomes}{%
\section*{Course Learning Outcomes}\label{course-learning-outcomes}}
\addcontentsline{toc}{section}{Course Learning Outcomes}

This course will enhance leadership skills and facilitate learning whereby a student will:

\begin{enumerate}
\def\labelenumi{\arabic{enumi}.}
\tightlist
\item
  Develop skills for living a life of wisdom and justice\\
\item
  Cultivate skills of logical argumentation.\\
\item
  Develop and enhance skills of written communication.\\
\item
  Practice the skills of wisdom and reason, such as humility, reflection, charity, rhetoric, and argumentation.\\
\item
  Write a journal that expresses the learned skills concisely and effectively.\\
\item
  Enhance skills of wisdom and reason to live life within the context of justice and faith.\\
\item
  Understand how to become a well-rounded person and citizen.
\end{enumerate}

\hypertarget{texts-and-reading-resources}{%
\section*{Texts and Reading Resources}\label{texts-and-reading-resources}}
\addcontentsline{toc}{section}{Texts and Reading Resources}

\textbf{Required}
- A variety of resources will be provided throughout each unit.

\hypertarget{course-notes}{%
\section*{Course Notes}\label{course-notes}}
\addcontentsline{toc}{section}{Course Notes}

\hypertarget{how-to-navigate-this-book}{%
\subsection*{How To Navigate This Book}\label{how-to-navigate-this-book}}
\addcontentsline{toc}{subsection}{How To Navigate This Book}

Take a moment to experiment with the controls in the toolbar at the top of the page. You can search this book for a word or phrase (for example, to look up a definition). To move quickly to different portions of the book, click on the appropriate chapter or section in the table of contents on the left. The buttons at the top of the page allow you to show/hide the table of contents, search the book, adjust the typeface, the font size, and the background colour to make the text easier to read.

\includegraphics{assets/course-intro/menu.png}

The faint left and right arrows at the sides of each page (or bottom of the page if it's narrow enough) allow you to step to the next/previous section. Here's what they look like:

\includegraphics{assets/course-intro/left_arrow.png} \includegraphics{assets/course-intro/right_arrow.png}

You can also download an offline copy of this books in a pdf format. If you are having any accessibility or navigation issues with this book, please reach out to your instructor or our online team at \href{mailto:elearning@twu.ca}{\nolinkurl{elearning@twu.ca}}

\hypertarget{course-units}{%
\subsection*{Course Units}\label{course-units}}
\addcontentsline{toc}{subsection}{Course Units}

This course is organized into 3 units. Each unit of the course will provide you with the following information:

\begin{itemize}
\tightlist
\item
  A general overview of the key concepts that will be addressed during the unit.\\
\item
  Specific learning outcomes and topics for the unit.\\
\item
  Learning activities to help you engage with the concepts. These often include key readings, videos, and reflective prompts.\\
\item
  The Assessment section provides details on assignments you will need to complete throughout the course to demonstrate your understanding of the course learning outcomes.
\end{itemize}

\begin{caution}
Note that assessments, including assignments and discussion posts will
be submitted in Moodle. See the Assessment tab in Moodle for assignment
details and dropboxes.
\end{caution}

\hypertarget{course-activities}{%
\subsection*{Course Activities}\label{course-activities}}
\addcontentsline{toc}{subsection}{Course Activities}

Below is some key information on features you will see throughout the course.

\begin{reflect}
\textbf{\emph{Learning Activity}}\\
This box will prompt you to engage in course concepts, often by viewing
resources and reflecting on your experience and/or learning. Most
learning activities are ungraded and are designed to help prepare you
for the assessment in this course.
\end{reflect}

\begin{assessment}
\textbf{\emph{Assessment}}\\
This box will signify an assignment you will submit in Moodle. Note that
assignments demonstrate your understanding of the course learning
outcomes. Be sure to review the grading rubrics for each assignment.
\end{assessment}

\begin{progress}
\textbf{\emph{Checking Your Learning}}\\
This box is for checking your understanding, to make sure you are ready
for what follows.
\end{progress}

\begin{feedback}
\textbf{\emph{Note}}\\
This box signifies key notes, important quotes, or case students. It may
also warn you of possible problems or pitfalls you may encounter!
\end{feedback}

\hypertarget{wisdom}{%
\chapter{Wisdom}\label{wisdom}}

\includegraphics{assets/u1/Unit_1_Overview.png}

CCO Creative Commons by \href{https://isorepublic.com/photo/sun-reflecting-on-waves/}{ISO Republic}

\hypertarget{overview}{%
\section*{Overview}\label{overview}}
\addcontentsline{toc}{section}{Overview}

This unit focuses on wisdom. In this unit, we will (i) introduce some concepts of wisdom found in the literature; and (ii) introduce the ``Life Skills'' to journal and practice. Note that the goal is not to provide a comprehensive summary of each reading, but rather to gain some advice about wisdom from each reading and then practice and journal that advice.

\hypertarget{topics}{%
\subsection*{Topics}\label{topics}}
\addcontentsline{toc}{subsection}{Topics}

This unit is divided into the following topics:

\begin{enumerate}
\def\labelenumi{\arabic{enumi}.}
\tightlist
\item
  Wisdom and Action\\
\item
  Wisdom and Knowledge\\
\item
  Wisdom and Humility\\
\item
  Wisdom and Bias\\
\item
  Wisdom and Charity\\
\item
  Wisdom and Rhetoric
\end{enumerate}

\hypertarget{learning-outcomes}{%
\subsection*{Learning Outcomes}\label{learning-outcomes}}
\addcontentsline{toc}{subsection}{Learning Outcomes}

When you have completed this unit, you should be able to:

\begin{itemize}
\tightlist
\item
  Develop the skills to live a wise and just life\\
\item
  Develop wisdom, reason, humility, and charity by practicing these skills\\
\item
  Enhance skills of wisdom and reason to live life within the context of justice and faith.\\
\item
  Understand how to become a well-rounded person and citizen.
\end{itemize}

\hypertarget{activity-checklist}{%
\subsection*{Activity Checklist}\label{activity-checklist}}
\addcontentsline{toc}{subsection}{Activity Checklist}

{An Activity Checklist has not been provided}

\hypertarget{resources}{%
\subsection*{Resources}\label{resources}}
\addcontentsline{toc}{subsection}{Resources}

\begin{itemize}
\tightlist
\item
  Nozick, R., 1989, ``What is Wisdom and Why Do Philosophers Love it So?'' in The Examined Life, New York: Touchstone Press, Chapter 23.\\
\item
  \href{https://www.earlymoderntexts.com/assets/pdfs/descartes1644part1.pdf}{Descartes, R., Miller, V. R., \& Miller, R. P. (1983). Principles of philosophy. Dordrecht, Holland: Reidel.}\\
\item
  \href{https://twu.idm.oclc.org/login?url=https://search.ebscohost.com/login.aspx?direct=true\&db=nlebk\&AN=1085368\&site=eds-live\&scope=site\&ebv=EB\&ppid=pp_1}{Plato, \& Jowett, B. (1999). Apology. Project Gutenberg.}\\
\item
  {[}Stanovish, K. (2020). The Bias that divides us. Quillette.{]} (\url{https://quillette.com/2020/09/26/the-bias-that-divides-us/})\{target=``\_blank''\}. (Free subscription)\\
\item
  Johnson, Ralph. 1981. ``Charity Begins at Home,'' Informal Logic Newsletter 3: 4-9.\\
\item
  Heinrichs, J., 2007 ``Set Your Goals,'' chapter 2 in ``Thank you for Arguing: What Aristotle, Lincoln, and Homer Simpson can teach us about the Art of Persuasion. New York : Three Rivers Press,''
\end{itemize}

\hypertarget{wisdom-and-action}{%
\section{Wisdom and Action}\label{wisdom-and-action}}

\hypertarget{activity-reading}{%
\subsection*{Activity: Reading}\label{activity-reading}}
\addcontentsline{toc}{subsection}{Activity: Reading}

\begin{reflect}
Pre-reading: Robert Nozick, \href{assets/u1/PAT_10018662_What_is_Wisdom.pdf}{\emph{``What Is Wisdom And Why Do Philosophers Love It So?''}}
\end{reflect}

This topic examines wisdom and action. According to Nozick, wisdom is practical in that wisdom is required to live life well, to cope with and respond to the many challenges of life, and, if possible, avoid some of those challenges. For living life well, wisdom requires both affirming the right beliefs and practicing the right actions. \textbf{How do we know which beliefs and actions are appropriate for living life well?} Nozick argues that a wise person holds beliefs about many aspects of life, including beliefs about life's most important values, beliefs about achieving one's goals, and following the necessary steps to achieve these goals. Likewise, the right actions for living a fulfilling life are derived from applying the aforementioned beliefs. If a person knows about the most important values and goals in life, but never applies them to their behavior, then they are not considered wise. In other words, having knowledge does not entail wisdom. Nozick is also aware that applying the appropriate beliefs and actions is not sufficient to live life well. Sometimes bad things occur for no apparent reason, which may undermine our efforts. A wise person is aware of this problem, and while pursuing their goals, prepares for failure, learns how to avoid it, and how to respond to it if failure occurs.

\hypertarget{highlights-from-the-reading}{%
\subsubsection*{Highlights from the reading}\label{highlights-from-the-reading}}
\addcontentsline{toc}{subsubsection}{Highlights from the reading}

\begin{enumerate}
\def\labelenumi{\arabic{enumi}.}
\tightlist
\item
  ``What is wisdom? Wisdom is an understanding of what is important, where this understanding informs a (wise) person's thought and action. Things of lesser importance are kept in proper perspective. (p.~267)\\
\item
  ``Wisdom is practical; it helps. Wisdom is what you need to understand in order to live well and cope with the central problems and avoid the dangers in the predicaments human beings find themselves in.'' (p.~267)\\
\item
  ``Wisdom is not just one type of knowledge, but diverse. What a wise person needs to know and understand constitutes a varied list: the most important goals and values of life - the ultimate goal, if there is one; what means will reach these goals without too great a cost; what kinds of dangers threaten the achieving of these goals; how to recognize and avoid or minimize these dangers\ldots{}''(p.~269)\\
\item
  ``There also will be bits of negative wisdom: certain things are NOT important.'' (p.~269) ``A wise person knows these diverse things and lives them. Someone who only knew them, who offered good advice to others yet who lived foolishly himself, would not be termed wise.'' (p.~270)\\
\item
  ``Wisdom does guarantee success in achieving life's important goals\ldots{}''(p.~270)\\
\item
  ``A wise person will have gone in the right direction, and if the world thwarts his journey, he will have known how to respond to that too.'' (p.~270)\\
\item
  ``Moreover, the process of living wisely, pursuing or opening oneself to what is important, taking account of a range of circumstances and utilizing one's fullest capacities to steer skillfully through them, is itself a way of being deeply connected to reality. (p.~276) ``The person who lives wisely connects to reality more thoroughly than someone who moves through life spoon-fed by circumstances, even if what they try to feed is reality.'' (p.~276)\\
\item
  ``Wisdom is not simply knowing how to steer one's way through life, cope with difficulties etc.. It also is knowing the deepest story, being able to see and appreciate the deepest significance of whatever occurs\ldots(p.~276)
\end{enumerate}

\hypertarget{activity-watch-and-reflect}{%
\subsection*{Activity: Watch and Reflect}\label{activity-watch-and-reflect}}
\addcontentsline{toc}{subsection}{Activity: Watch and Reflect}

\begin{reflect}
Please take a moment to watch the following video, a look at the definition of wisdom from philosophers Valerie Tiberius and Philip Kitcher, as well as psychologist Lisa Feldman Barret. The video duration is 5:55 minutes and is an optional resource.

\href{https://www.youtube.com/watch?v=obqedyeUcwk}{Watch: What is Wisdom?}
\end{reflect}

This topic contains a learning activity, its purpose is to examine one application of wisdom and setting a personal goal that may improve one's life. Along with setting a personal goal, one must outline the good habits necessary to achieve that goal, and devise strategies to avoid the bad habits that would undermine it. Additionally, one needs to prepare for failure in terms of developing the right habits and reaching the end goal, as well as developing a strategy for avoiding failure and addressing it, if it happens.

\hypertarget{activity-take-notes-for-your-life-journal}{%
\subsection*{Activity : Take notes for your Life journal}\label{activity-take-notes-for-your-life-journal}}
\addcontentsline{toc}{subsection}{Activity : Take notes for your Life journal}

\begin{reflect}
Answering the following questions is optional. Answering these questions and saving them as a document will allow you to take notes on the readings in preparation for the Life Journal assignment due at the end of Unit 1.
\end{reflect}

\hypertarget{wisdom-and-knowledge}{%
\section{Wisdom and Knowledge}\label{wisdom-and-knowledge}}

\hypertarget{activity-reading-1}{%
\subsection*{Activity: Reading}\label{activity-reading-1}}
\addcontentsline{toc}{subsection}{Activity: Reading}

\begin{reflect}
Pre-reading: \href{assets/u1/PrinciplesofPhilosophy.pdf}{Preface to Principles of Philosophy; Principles 1, 2, 3, 75}
\end{reflect}

In the previous topic, we established that knowledge is not sufficient for wisdom because wisdom requires action. That is, knowledge must be applied correctly in order to live life well. While Descartes endorses that view of the limitations of knowledge and the importance of application, he maintains that one should pursue knowledge in the highest domains, such as knowledge of God, science, and morality, because that knowledge benefits the one pursuing wisdom and a fulfilling life.

There are two ideas associated with wisdom and knowledge to discuss. The first idea is that \textbf{acquiring knowledge requires the practice of doubting}. Do not believe everything you read and hear. The second idea is that a \textbf{wise person obtains knowledge of the most valuable things in life, such as knowledge about God, science, and morality}. According to Descartes, pursuing knowledge of these higher domains is not only a good in itself, but that higher knowledge can enable one to live a better life.

Having doubt about something does not entail rejecting its truth. For our purposes, doubting something means applying careful consideration, such as asking yourself: Why should I think this is true? What is the best way to verify the claim? Who can give me an informed opinion? Are there any concerns about this content? Which aspects of the content are strong and which are weak? Healthy doubt is valuable for acquiring knowledge and living life well. For example, whenever we detect incoming scams, such as phone and computer scams, we employ doubt. In this regard, one intends to uncover the nature of the scam and warn others of its harmful potential. The hesitation (or doubt) to believe and act according to the computer scam is similar to the hesitation one should apply when listening to the news, a podcast, a sermon, or reading an opinion on Facebook or Twitter. However, particularly in ordinary affairs, Decartes also warns that extreme doubt may be harmful. Neither the need for basic necessities nor the importance of friendship should be questioned. When individuals are faced with situations such as these, it is often necessary that they act even if they lack all the information about the situation. Descartes argues that remaining paralyzed by doubt is an inadvisable course.

To pursue a higher knowledge about God, science, and morality, Descartes imagines the structure of a tree, whose roots are metaphysics, the trunk is physics, and the branches represent the remainder of scientific inquiry, including morality (note that `science' in this context encompasses more domains than a contemporary definition of science). Descartes believed that in order to understand the human condition, one must have an understanding of metaphysics and science. Descartes' point about seeking this type of valuable knowledge about God, science, and morality is relevant because of the human obsession with entertainment. We often seek entertainment and pleasure rather than searching for the truth about the most important ideas in life. That obsession with entertainment and the neglect of seeking valuable knowledge undermines one's potential to gain wisdom and seek a fulfilling life.

\hypertarget{highlights-from-the-reading-1}{%
\subsubsection*{Highlights from the reading}\label{highlights-from-the-reading-1}}
\addcontentsline{toc}{subsubsection}{Highlights from the reading}

\begin{enumerate}
\def\labelenumi{\arabic{enumi}.}
\tightlist
\item
  For example, the word `philosophy' means the study of wisdom, and by `wisdom' it means not only prudence in our everyday affairs but also a perfect knowledge of all things that mankind is capable of knowing\ldots'' (179)\\
\item
  ``\ldots we should not doubt the probable truths which concern the conduct of life\ldots{}''(182).\\
\item
  ``Thus, the whole of philosophy is like a tree. The roots are metaphysics, the trunk is physics, and the branches emerging from the trunk are all the other sciences\ldots{}'' (186)\\
\item
  ``\ldots For although the truth often does not touch our imagination as much as falsehood and pretense, because it seems less striking and more plain, nevertheless the satisfaction it produces is always more durable and more solid.'' (188)\\
\item
  ``The second benefit {[}of studying Descartes' principles of philosophy{]} is that the study of these principles will accustom people little by little to form better judgments about all the things they come across, and hence will make them wiser.'' (188)
\end{enumerate}

\hypertarget{activity-watch-and-reflect-1}{%
\subsection*{Activity: Watch and Reflect}\label{activity-watch-and-reflect-1}}
\addcontentsline{toc}{subsection}{Activity: Watch and Reflect}

\begin{reflect}
Please take a moment to view this optional video. In this video, the Honorable Richard Holloway, who is a writer, broadcaster, and cleric from Scotland, seeks to initiate a debate around the more profound spiritual questions of life. The duration of this video is 4:45 minutes.

\href{https://www.youtube.com/watch?v=LERi_Xjfio4}{Watch: \emph{Why doubt is a good thing}}
\end{reflect}

The aim of the following learning activities is to explore and learn about these two applications of knowledge: the practice of doubt and the acquisition of valuable knowledge over entertainment.

\hypertarget{activity-learning-to-doubt---simulation-exercise-1}{%
\subsection*{Activity : Learning to doubt - Simulation Exercise 1}\label{activity-learning-to-doubt---simulation-exercise-1}}
\addcontentsline{toc}{subsection}{Activity : Learning to doubt - Simulation Exercise 1}

\begin{reflect}
The simulation exercises are meant to simulate a conversation and introduce students to the various topics in Unit 1. Students are encouraged to not only review the simulation exercises but to practice the applications of wisdom in their daily lives.
\end{reflect}

\hypertarget{activity-take-notes-for-your-life-journal-1}{%
\subsection*{Activity: Take notes for your Life Journal}\label{activity-take-notes-for-your-life-journal-1}}
\addcontentsline{toc}{subsection}{Activity: Take notes for your Life Journal}

\begin{reflect}
Answering the following questions is optional. Answering these questions and saving them as a document will allow you to take notes on the readings in preparation for the Life Journal assignment due at the end of Unit 1.
\end{reflect}

\hypertarget{wisdom-and-humility}{%
\section{Wisdom and Humility}\label{wisdom-and-humility}}

\hypertarget{activity-reading-2}{%
\subsection*{Activity: Reading}\label{activity-reading-2}}
\addcontentsline{toc}{subsection}{Activity: Reading}

\begin{reflect}
Pre-reading: \href{http://classics.mit.edu/Plato/apology.html}{Apology - Plato}
\end{reflect}

The third topic focuses on wisdom and humility. Our previous two topics discussed the importance of knowledge when seeking wisdom. Nevertheless, even Descartes recognizes that knowledge in some domains is difficult to obtain. Plato believed that a wise person understands their limitations about what they know and do not know. In other words, \textbf{a wise person is aware of their ignorance}. The word ``ignorance'' in this case does not have a negative connotation, but rather describes a lack of knowledge about a particular thing.

In ``the Apology,'' Plato introduces the character Socrates, who is on trial for his life. During his defense, Socrates describes the account of his friend, Chaerephon, seeking answers from the Oracle of Delphi, wondering if there is any person wiser than Socrates. In reply to Chaerephon, the oracle says that no one is wiser than Socrates. Socrates is puzzled by the revelation about his wisdom since he considers himself to be unwise. Socrates then asks questions of different people in the city, such as politicians, poets, and artisans, in an attempt to solve the mystery about the oracle's revelation. Socrates discovers that while these individuals claim to know a great deal about issues such as the nature of beauty and what is good, he observes that they know very little about such issues. These people think they possess knowledge, when in fact they do not, and that is why they lack wisdom. Their way of thinking contrasts with Socrates, who thinks he lacks knowledge about these issues; more specifically, Socrates is aware of his ignorance about the nature of beauty and what is good, while the others are unaware of their own ignorance about these same issues. Because of this, Socrates is the wisest person.

One application of being aware of ignorance for seeking wisdom is to practice a humble attitude towards the complex issues in life, such as how to interpret the Bible, evolution, the nature of sexuality, the implications of social media for our lives, and was Donald Trump an effective president? These types of questions represent complex ideas that even some of the most specialized scholars cannot answer. A wise person demonstrates their humility towards these complex issues by asking thoughtful questions about the opposing view, knowing when to withhold judgment and opinion, and with a willingness to change their mind.

\hypertarget{highlights-from-the-reading-2}{%
\subsubsection*{Highlights from the reading}\label{highlights-from-the-reading-2}}
\addcontentsline{toc}{subsubsection}{Highlights from the reading}

``When I heard the answer, I said to myself, What can the god mean? and what is the interpretation of this riddle? for I know that I have no wisdom, small or great. What can he mean when he says that I am the wisest of men? And yet he is a god and cannot lie; that would be against his nature. After a long consideration, I at last thought of a method of trying the question. I reflected that if I could only find a man wiser than myself, then I might go to the god with a refutation in my hand. I should say to him,''Here is a man who is wiser than I am; but you said that I was the wisest.'' Accordingly I went to one who had the reputation of wisdom, and observed to him - his name I need not mention; he was a politician whom I selected for examination - and the result was as follows: When I began to talk with him, I could not help thinking that he was not really wise, although he was thought wise by many, and wiser still by himself; and I went and tried to explain to him that he thought himself wise, but was not really wise; and the consequence was that he hated me, and his enmity was shared by several who were present and heard me. So I left him, saying to myself, as I went away: Well, although I do not suppose that either of us knows anything really beautiful and good, I am better off than he is - for he knows nothing, and thinks that he knows. I neither know nor think that I know. In this latter particular, then, I seem to have slightly the advantage of him. Then I went to another, who had still higher philosophical pretensions, and my conclusion was exactly the same. I made another enemy of him, and of many others besides him.''

``After this I went to one man after another, being not unconscious of the enmity which I provoked, and I lamented and feared this: but necessity was laid upon me - the word of God, I thought, ought to be considered first. And I said to myself, Go I must to all who appear to know, and find out the meaning of the oracle. And I swear to you, Athenians, by the dog I swear! - for I must tell you the truth - the result of my mission was just this: I found that the men most in repute were all but the most foolish; and that some inferior men were really wiser and better. I will tell you the tale of my wanderings and of the''Herculean'' labors, as I may call them, which I endured only to find at last the oracle irrefutable. When I left the politicians, I went to the poets; tragic, dithyrambic, and all sorts. And there, I said to myself, you will be detected; now you will find out that you are more ignorant than they are. Accordingly, I took them some of the most elaborate passages in their own writings, and asked what was the meaning of them - thinking that they would teach me something. Will you believe me? I am almost ashamed to speak of this, but still I must say that there is hardly a person present who would not have talked better about their poetry than they did themselves. That showed me in an instant that not by wisdom do poets write poetry, but by a sort of genius and inspiration; they are like diviners or soothsayers who also say many fine things, but do not understand the meaning of them. And the poets appeared to me to be much in the same case; and I further observed that upon the strength of their poetry they believed themselves to be the wisest of men in other things in which they were not wise. So I departed, conceiving myself to be superior to them for the same reason that I was superior to the politicians.''

``At last I went to the artisans, for I was conscious that I knew nothing at all, as I may say, and I was sure that they knew many fine things; and in this I was not mistaken, for they did know many things of which I was ignorant, and in this they certainly were wiser than I was. But I observed that even the good artisans fell into the same error as the poets; because they were good workmen they thought that they also knew all sorts of high matters, and this defect in them overshadowed their wisdom - therefore I asked myself on behalf of the oracle, whether I would like to be as I was, neither having their knowledge nor their ignorance, or like them in both; and I made answer to myself and the oracle that I was better off as I was.''

The aim in the following learning activities is to practice humility about what we know and don't know. One way to practice humility is to ask questions and limit how often we share our opinions.

\hypertarget{activity-watch-and-reflect-2}{%
\subsection*{Activity: Watch and Reflect}\label{activity-watch-and-reflect-2}}
\addcontentsline{toc}{subsection}{Activity: Watch and Reflect}

\begin{reflect}
Please take a moment to watch this optional video with Dr.~Ian Church, What is intellectual humility? A doxastic account. Here is the first of three parts. If you have the time, you should watch them all. Please note that this video lasts for approximately 2:27 minutes.

\href{https://www.youtube.com/watch?v=8CZIkGEJYRY}{Watch: \emph{What is intellectual humility?}}
\end{reflect}

\hypertarget{activity-practicing-humility---simulation-exercise-2}{%
\subsection*{Activity: Practicing Humility - Simulation Exercise 2}\label{activity-practicing-humility---simulation-exercise-2}}
\addcontentsline{toc}{subsection}{Activity: Practicing Humility - Simulation Exercise 2}

\begin{reflect}
The simulation exercises are meant to simulate a conversation and introduce students to the various topics in Unit 1. Students are encouraged to not only review the simulation exercises but to practice the applications of wisdom in their daily lives.
\end{reflect}

\hypertarget{activity-take-notes-for-your-life-journal-2}{%
\subsection*{Activity: Take notes for your Life Journal}\label{activity-take-notes-for-your-life-journal-2}}
\addcontentsline{toc}{subsection}{Activity: Take notes for your Life Journal}

\begin{reflect}
Answering the following questions is optional. Answering these questions and saving them as a document will allow you to take notes on the readings in preparation for the Life Journal assignment due at the end of Unit 1.
\end{reflect}

\hypertarget{wisdom-and-bias}{%
\section{Wisdom and Bias}\label{wisdom-and-bias}}

\hypertarget{activity-reading-3}{%
\subsection*{Activity: Reading}\label{activity-reading-3}}
\addcontentsline{toc}{subsection}{Activity: Reading}

\begin{reflect}
Pre-reading: \href{https://quillette.com/2020/09/26/the-bias-that-divides-us/}{The Bias that Divides Us}

\textbf{Note:} In order to access this article you will need to join the free email list.
\end{reflect}

In the previous topic, we discussed the importance of humility when seeking knowledge. Topic 4 explores the relationship between wisdom and bias and this relationship is valuable for seeking humility. Recall that a wise person seeking humility recognizes their own ignorance regarding the complex issues in life. Identifying personal ignorance also requires acknowledging one's biases. When one recognizes the deep influence of human biases on knowledge acquisition, it becomes easier to recognize one's ignorance.

There are many types of biases, both social and cognitive. \textbf{Social bias} is a kind of negative attitude and opinion toward other members in society, usually without good reason. Examples of social bias include prejudice toward other members of society based on their race, gender, age, and so forth. A \textbf{cognitive bias} refers more generally to human intelligence, and, in some cases, to human reasoning. Specifically, cognitive bias occurs when personal interest conflicts with the acquisition of knowledge; that is, humans tend to make errors of judgment due to their bias towards self-interest. One form of cognitive bias is called \textbf{confirmation bias}. Confirmation bias is when people attempt to acquire knowledge by consulting evidence that only supports their own agenda. An example of this would be a person who believes that God exists and supports their position by viewing sources that affirm the existence of God and ignoring or misinterpreting sources that deny this belief. That person seeks information that reinforces their current opinions and generally avoids information that undermines their current opinions.

A more precise way to think about confirmation bias is called ``myside'' bias. While myside bias includes the error of favoring evidence that supports one's previously held opinions, myside bias locates that error within the context of group membership. Those previously held opinions often reflect convictions to agree with one's social group and discontentment towards agreeing with opposing social groups. For example, Stanovich describes a person who commits what is perceived as a virtuous act. If that person is a member of your social group, you will be more inclined to affirm the act as virtuous. If that person is not a member of your social group, you will be less inclined to affirm the act as virtuous. This error of reasoning comes from affirming the virtue of the act only by association with a particular social group. Stanovich contends that myside bias affects everyone, especially intellectual elites, and those who believe they are immune to such biases are more susceptible to them.

\hypertarget{highlights-from-the-reading-3}{%
\subsubsection*{Highlights from the reading}\label{highlights-from-the-reading-3}}
\addcontentsline{toc}{subsubsection}{Highlights from the reading}

\begin{enumerate}
\def\labelenumi{\arabic{enumi}.}
\tightlist
\item
  ``\ldots myside bias: people evaluate evidence, generate evidence, and test hypotheses in a manner biased toward their own prior beliefs, opinions, and attitudes.''\\
\item
  ``Research has shown that myside bias is displayed in a variety of experimental situations: people evaluate the same virtuous act more favourably if committed by a member of their own group and evaluate a negative act less unfavourably if committed by a member of their own group; they evaluate an identical experiment more favourably if the results support their prior beliefs than if the results contradict their prior beliefs; and when searching for information, people select information sources that are likely to support their own position. Even the interpretation of a purely numerical display of outcome data is tipped in the direction of the subject's prior belief. Likewise, judgments of logical validity are skewed by people's prior beliefs. Valid syllogisms with the conclusion ``therefore, marijuana should be legal'' are easier for liberals to judge correctly and harder for conservatives; whereas valid syllogisms with the conclusion ``therefore, no one has the right to end the life of a fetus'' are harder for liberals to judge correctly and easier for conservatives.''\\
\item
  ``myside bias is one of the most ubiquitous of biases because it is exhibited by the vast majority of subjects studied. Myside bias is also not limited to individuals with certain cognitive or demographic characteristics. It is one of the most universal of cognitive biases.''\\
\item
  ``In short, the convictions that determine your side when you think in a mysided fashion, often don't come from rational thought.''\\
\item
  ``However, one particular bias---myside bias---sets a trap for the cognitively sophisticated. Regarding most biases, they are used to thinking---rightly---that they are less biased. However, myside thinking about your political beliefs represents an outlier bias where this is not true.''\\
\item
  ``This may lead to a particularly intense bias blind spot among certain cognitive elites. If you are a person of high intelligence, if you are highly educated, and if you are strongly committed to an ideological viewpoint, you will be highly likely to think you have thought your way to your viewpoint.''\\
\item
  ``University faculty in the social sciences fit this bill perfectly. And the opening for a massive bias blind spot occurs when these same faculty think that they can objectively study, within the confines of an ideological monoculture, the characteristics of their ideological opponents.''\\
\item
  ``Anything that makes us more skeptical about our beliefs will tend to decrease the myside bias that we display (by preventing beliefs from turning into convictions).''\\
\item
  ``In short, just as we are gorging on fat-laden food that is not good for us because our bodies were built by genes with a selfish replicator survival logic, so we are gorging on memes that fit our resident beliefs because cultural replicators have a similar survival logic. And just as our overconsumption of fat-laden fast foods has led to an obesity epidemic, so our overconsumption of congenial memes has made us memetically obese as well.''\\
\item
  ``When I talk to lay audiences about different types of cognitive processes, I use the example of broccoli and ice cream. Some cognitive processes are demanding but necessary. They are the broccoli. Other thinking tendencies come naturally to us and they are not cognitively demanding processes. They are the ice cream. In lectures, I point out that broccoli needs a cheerleader, but ice cream does not. This is why education rightly emphasizes the broccoli side of thinking---why it stresses the psychologically demanding types of thinking that people need encouragement to practice.''\\
\item
  When you first wake up, remember that you are biased. As you discuss controversies with fellow students, colleagues, family, and friends, your biases are hard at work, practically bullying you into thinking a certain way. The moment you deny this reality, you succumb to yet another bias. The aim in the following learning activity is to become more aware of one's cognitive biases and how that awareness can promote an attitude of humility.
\end{enumerate}

\hypertarget{activity-watch-and-reflect-3}{%
\subsection*{Activity: Watch and Reflect}\label{activity-watch-and-reflect-3}}
\addcontentsline{toc}{subsection}{Activity: Watch and Reflect}

\begin{reflect}
Watch the optional video below. Jonathan Haidt, a renowned social psychologist, shares a personal example of confirmation bias on social media and emphasizes the importance of listening to ideas from different political perspectives. Please note that the video is 2:13 minutes in length.

\href{https://www.youtube.com/watch?v=XO0AVyF6EVg}{Watch: \emph{Jonathan Haidt on Why We're Convinced We're Right (and everyone else is wrong!)}}
\end{reflect}

\hypertarget{activity-awareness-of-bias---simulation-exercise-3}{%
\subsection*{Activity: Awareness of bias - Simulation Exercise 3}\label{activity-awareness-of-bias---simulation-exercise-3}}
\addcontentsline{toc}{subsection}{Activity: Awareness of bias - Simulation Exercise 3}

\begin{reflect}
The simulation exercises are meant to simulate a conversation and introduce students to the various topics in Unit 1. Students are encouraged to not only review the simulation exercises but to practice the applications of wisdom in their daily lives.
\end{reflect}

\hypertarget{activity-take-notes-for-your-life-journal-3}{%
\subsection*{Activity : Take notes for your Life Journal}\label{activity-take-notes-for-your-life-journal-3}}
\addcontentsline{toc}{subsection}{Activity : Take notes for your Life Journal}

\begin{reflect}
Answering the following questions is optional. Answering these questions and saving them as a document will allow you to take notes on the readings in preparation for the Life Journal assignment due at the end of Unit 1.
\end{reflect}

\hypertarget{wisdom-and-charity}{%
\section{Wisdom and Charity}\label{wisdom-and-charity}}

\hypertarget{activity-reading-4}{%
\subsection*{Activity: Reading}\label{activity-reading-4}}
\addcontentsline{toc}{subsection}{Activity: Reading}

\begin{reflect}
Pre-reading: \href{assets/u1/CharityBeginsatHome.pdf}{Ralph H. Johnson, ``Charity Begins at Home''}
\end{reflect}

In Topic 5, the emphasis is on wisdom and charity. The term ``charity'' in this case refers to \textbf{a principle of treating a person's argument and/or opinion in its strongest form}. In the previous two topics, we discussed humility and bias, both of which are relevant to practicing charity. For example, when employing humility, one is aware of their ignorance and the need to ask questions about the opponent's view. That questioning can be a type of charity when done for the purpose of gaining understanding. Similarly, one is aware of myside bias, and that bias influences the interpretation and treatment of the opponent, especially if that opponent is from another social group. Recognizing the powerful influence of myside bias promotes charity by helping constrain one's immediate rejection of the opponent's view.

Charity involves more than humility and awareness of bias: it requires willpower. One must willfully, and temporarily, embrace the opponent's view, assist with constructing that view in its best possible light, and then express the view better than the opponent. The act of embracing an opposing argument is challenging for many reasons, but, as stated in the previous chapter on bias, humans are not built to act with charity toward those with whom they disagree, especially when those convictions are strong. Put in a different way, humans struggle to, on the one hand, affirm a viewpoint in which they hold strong convictions, and then, on the other hand, compartmentalize that view away while attempting to charitably evaluate objections to that conviction.

In his article, ``Charity Begins at Home,'' Johnson outlines the general principle of charity and then discusses some of the challenges with that principle. Practicing charity involves three steps:

\begin{enumerate}
\def\labelenumi{\arabic{enumi}.}
\tightlist
\item
  Identifying the argument: Does the argument even exist? According to Johnson, the first step is difficult in its own right because the art of constructing a well-organized argument is in decline.\\
\item
  Reconstructing the argument: This step requires that one re-build the argument to follow the rules of logical inference. Sometimes arguments include additional hidden premises, irrelevant information, and are simply weak or invalid. One must reconstruct the argument in its strongest form.\\
\item
  Presenting the main objections to the argument: Now that the argument has been identified and properly reconstructed, one should focus on the criticisms that undermine the main premises of the argument and avoid criticizing irrelevant information and implications of the argument.
\end{enumerate}

According to Johnson, a major problem with the principle of charity is that it requires too much effort and time to apply, as few people can construct an argument correctly in the first place. His solution to this problem is to apply the principle only when the following conditions are met: (i) the argument is fully expressed; (ii) the argument is brought forward by a serious person; (iii), the argument is about a serious issue. One implication of those criteria for the principle is that one should apply charity only when there is reason to think that the opponent will benefit from that charity.

\hypertarget{highlights-from-the-reading-4}{%
\subsubsection*{Highlights from the reading}\label{highlights-from-the-reading-4}}
\addcontentsline{toc}{subsubsection}{Highlights from the reading}

\begin{enumerate}
\def\labelenumi{\arabic{enumi}.}
\tightlist
\item
  ``The Principle of Charity which governs all levels of argument analysis is that the critic should provide the best possible interpretation of the material under consideration.'' (p.~5)\\
\item
  ``The justification of the principle is\ldots ethical. One is under the general obligation to be fair in one's dealings with others\ldots{}'' (p.5)\\
\item
  ``There is also a prudential reason for adhering to the principle\ldots{} It tells you that you want to interpret the argument's meaning in whatever way makes the most sense and force out of it, because otherwise, it can easily be reformulated\ldots to meet your objections.'' (p.5)\\
\item
  ``The first step in argument analysis must be to decide whether or not the passage under scrutiny contains an argument\ldots{}''(p.5)\\
\item
  ``Once a passage has been identified as containing an argument we are then faced with the task of identifying and setting forth its premises and conclusions in their logical relationships.'' (p.5)\\
\item
  ``Once the argument has been identified, extracted, and reconstructed, the job of the critic can begin\ldots one should look for the strongest possible criticisms of the argument\ldots{}''(p.6)\\
\item
  ``Effective argument analysis can be a grueling task.'' (p.8)\\
\item
  ``Many times I have found myself wrestling with diffuse and poorly expressed arguments that seem to have been sloppily put together - something dashed off in five minutes by a loose reasoner with scant knowledge of the demands of the argumentative process.'' (p.8)\\
\item
  ``In defense of this proposal {[}the proposal that the only arguments deserving charity are those that meet the three criteria{]}, let me make two points. First, the individual who dashes off a poorly expressed argument is, in all probability, not going to derive any profit from the critics' laborious undertaking. His or her commitment to the rational process is too slight and powers of reasoning too underdeveloped (else we would not be in this situation in the first place). (p.~8).\\
\item
  The aim in the following learning activity is to learn about practicing charity. The goal is to embrace your opponent's position, assist them in making their position as strong as possible, and only then introduce objections to their position.
\end{enumerate}

\hypertarget{activity-watch-and-reflect-4}{%
\subsection*{Activity: Watch and Reflect}\label{activity-watch-and-reflect-4}}
\addcontentsline{toc}{subsection}{Activity: Watch and Reflect}

\begin{reflect}
You are invited to watch the following optional video. John Corvino who is a speaker, writer, philosophy professor, and Dean of the Irvin D. Reid Honors College at Wayne State University in Detroit, is sharing about The Principle of Charity. Please note that this video is 2:07 minutes long.

\href{https://www.youtube.com/watch?v=LZZ7tQnI2-M}{Watch: \emph{The Principle of Charity}}
\end{reflect}

\hypertarget{activity-acting-with-charity---simulation-exercise-4}{%
\subsection*{Activity: Acting with charity - Simulation Exercise 4}\label{activity-acting-with-charity---simulation-exercise-4}}
\addcontentsline{toc}{subsection}{Activity: Acting with charity - Simulation Exercise 4}

\begin{reflect}
The simulation exercises are meant to simulate a conversation and introduce students to the various topics in Unit 1. Students are encouraged to not only review the simulation exercises but to practice the applications of wisdom in their daily lives.
\end{reflect}

\hypertarget{activity-take-notes-for-your-life-journal-4}{%
\subsection*{Activity: Take notes for your Life Journal}\label{activity-take-notes-for-your-life-journal-4}}
\addcontentsline{toc}{subsection}{Activity: Take notes for your Life Journal}

\begin{reflect}
Answering the following questions is optional. Answering these questions and saving them as a document will allow you to take notes on the readings in preparation for the Life Journal assignment due at the end of Unit 1.
\end{reflect}

\hypertarget{wisdom-and-rhetoric}{%
\section{Wisdom and Rhetoric}\label{wisdom-and-rhetoric}}

\begin{reflect}
Pre-reading:\href{assets/u1/PAT-10018663_Thank_you_for.pdf}{Jay Heinrichs, \emph{``Set Your Goals,''} chapter 2 in \emph{``Thank you for Arguing: What Aristotle, Lincoln, and Homer Simpson can teach us about the Art of Persuasion''}}
\end{reflect}

Topic 6 is about wisdom and rhetoric. Rhetoric refers to \textbf{the art of persuasion} or influence. Previously we discussed how wisdom requires an awareness and application of our actions, knowledge, humility, biases, and charity. The art of persuasion employs all of these skills. Along with those skills of wisdom, rhetoric also requires some tools of argumentation. A clarification regarding rhetoric is that the terms ``argument'', ``argue'', and ``argumentation'' can mean different things in different contexts. Students will learn the technical definitions of ``argument'' and ``argumentation'' in Unit 2, which may differ from how those terms are used in this topic of rhetoric. The student will also have the opportunity to develop some of the tools of technical argumentation in Unit 2, which will hone their rhetorical skills.

Arguments with other people, such as family, friends, or colleagues, are part of everyday life. A wise person knows how to argue without fighting. According to Heinrichs, a necessary skill for arguing productively is knowing the appropriate goal you wish to bring about. Are you trying to influence someone's decision? Is that feasible? Are you hoping to feel like a winner? Is that an appropriate goal? It may be your goal to change the mood of your opponent, or simply to inform your opponent of a different point of view. Would you like to end the argument quickly? After the argument, would you like to maintain the relationship with the person? Should you pursue either of these goals? How will you go about accomplishing your goal? In what way are you prepared to compromise? Are you prepared to experience defeat from your opponent's point of view in order to achieve your goal? What is the purpose of the argument and how should it be achieved?

Exercising the previous skills of wisdom when arguing with other people is helpful for practicing rhetoric. Wise actions, for example, require setting reasonable goals and avoiding the pitfalls that may undermine these goals. Similarly, when applying rhetoric in an argument with someone else, focus on the goal you wish to bring about and be willing to concede as much as possible to get there. Furthermore, remember that wisdom requires knowledge and humility. When those skills are applied to rhetoric, demonstrate the ability to challenge your own views to show your intention to understand the opponent's position. Knowledge of one's own bias is also essential to rhetoric. Ensure that the opponent understands that you are aware of your own biases and that you will listen to and learn from them. And in everything, act with charity. When arguing with others, assist them with their arguments, and, if possible, suggest ways in which their arguments can be strengthened. Demonstrate your comfort with the exchange by temporarily embracing their viewpoint. Charity of this kind encourages your opponent to act in a similar manner, and, in doing so, may reduce the defensiveness of each person. Depending on your rhetorical goal when arguing with another person, the tactics above may be useful.

\hypertarget{highlights-from-the-reading-5}{%
\subsubsection*{Highlights from the reading}\label{highlights-from-the-reading-5}}
\addcontentsline{toc}{subsubsection}{Highlights from the reading}

\begin{enumerate}
\def\labelenumi{\arabic{enumi}.}
\tightlist
\item
  ``This chapter will help you distinguish between an argument and a fight, and to choose what you want to get out of an argument.'' (p, 16) ``You succeed in an argument when you persuade your audience.'' (p.~16)\\
\item
  ``\ldots argument by the stick - fighting - is no argument. It never persuades, it only inspires revenge or retreat.'' (p.17)\\
\item
  ``In a fight, one person takes out his aggression on another. Donald Trump was fighting when he said of Rosie O'Donnell, ``I mean, I'd look at her in that fat, ugly face of hers, I'd say `Rosie, you're fired.' On the other hand, when Geroge Foreman tries to sell you a grill, he makes an argument: persuasion that tries to change your mood, your mind, or your willingness to do something.'' (p.17)\\
\item
  ``The basic difference between an argument and a fight: an argument, done skillfully, gets people to want what you want. You fight to win; you argue to achieve agreement.'' (p.~17)\\
\item
  ``Changing the mood is the easiest goal, and usually the one you work on first {[}\#1 goal{]}.'' (p.~21) `` \ldots making them decide what you want.'' {[}\#2 goal{]} (p.~21)\\
\item
  ``Goal number three - in which you get an audience to do something or stop doing it - is the most difficult.'' {[}\#3 goal{]} (p.22).
\end{enumerate}

The aim in the following learning activity is to practice a skill of rhetoric: set a goal of changing the mood of the argument and try to outline the way of achieving that goal. One way to change the mood of the argument is by (implicitly) affirming your opponent's view. Say something like this: ``Fair enough. I need to think about this. Can you explain that in more detail?'' OR ``That's good advice. Can you tell me a bit more?'' Be careful not to patronize, but sincerely affirm your opponent's view to change the mood of the debate, from each person being defensive to each person being receptive.

\hypertarget{activity-watch-and-reflect-5}{%
\subsection*{Activity: Watch and Reflect}\label{activity-watch-and-reflect-5}}
\addcontentsline{toc}{subsection}{Activity: Watch and Reflect}

\begin{reflect}
Take a moment to watch this optional video. Here, Camille A. Langston, a specialist in rhetoric and nineteenth-century women, explains the basics of deliberative rhetoric and shares tips for appealing to the ethos, logos, and pathos of your audience. This video is 4:29 minutes long.

\href{https://www.youtube.com/watch?v=3klMM9BkW5o}{Watch: \emph{How to use rhetoric to get what you want}}
\end{reflect}

\hypertarget{activity-the-art-of-persuasion---simulation-exercise-5}{%
\subsection*{Activity: The art of persuasion - Simulation Exercise 5}\label{activity-the-art-of-persuasion---simulation-exercise-5}}
\addcontentsline{toc}{subsection}{Activity: The art of persuasion - Simulation Exercise 5}

\begin{reflect}
The simulation exercises are meant to simulate a conversation and introduce students to the various topics in Unit 1. Students are encouraged to not only review the simulation exercises but to practice the applications of wisdom in their daily lives.
\end{reflect}

\hypertarget{activity-take-notes-for-your-life-journal-5}{%
\subsection*{Activity: Take notes for your Life Journal}\label{activity-take-notes-for-your-life-journal-5}}
\addcontentsline{toc}{subsection}{Activity: Take notes for your Life Journal}

\begin{reflect}
Answering the following questions is optional. Answering these questions and saving them as a document will allow you to take notes on the readings in preparation for the Life Journal assignment due at the end of Unit 1.
\end{reflect}

\hypertarget{unit-1-summary}{%
\section*{Unit 1 Summary}\label{unit-1-summary}}
\addcontentsline{toc}{section}{Unit 1 Summary}

In Unit 1, we introduced some concepts and applications of wisdom. One insight about wisdom is that wisdom is a guide to living life well, and the topics covered in Unit 1 expounded that idea of how to live life well.

Living life well often requires setting reasonable goals and practicing good habits to achieve these goals. Setting reasonable goals and practicing good habits includes obtaining knowledge by way of healthy doubt, both knowledge of the day-to-day and knowledge of the highest order, such as knowledge about God, science, and morality. Knowledge can be difficult to obtain, however. And, thus, a wise person recognizes their ignorance of the complex issues in life. Furthermore, admitting one's ignorance is necessary for acknowledging human biases. Human minds are wired to think certain ways and those biases may hinder the quest for acquiring knowledge and living life well.

Humans are social creatures and learning to live life well often occurs within social contexts. One must learn to live well with other people. A wise person gains knowledge from their opponents by acting with charity. A charitable person does not interrupt, insult, or demean their opponent or their argument. A charitable person attempts to restructure the opponent's argument in its strongest form and then respectfully engage that argument with objections.

The final topic in Unit 1 introduced rhetoric, the art of persuasion. A wise person knows how to get along in life and that ``getting along'' requires some degree of tact. When one applies for a job, volunteers for a charitable organization, deals with a frustrating professor or employer, chooses to get married, and negotiates a mortgage, one must be able to persuade others (and oneself) to some extent. Learning to practice these concepts of wisdom can help with learning how to live life well.

\hypertarget{assessment}{%
\section*{Assessment}\label{assessment}}
\addcontentsline{toc}{section}{Assessment}

{Content missing}

\hypertarget{checking-your-learning}{%
\section*{Checking Your Learning}\label{checking-your-learning}}
\addcontentsline{toc}{section}{Checking Your Learning}

{Content missing}

\hypertarget{reason}{%
\chapter{Reason}\label{reason}}

\includegraphics{assets/u2/Unit2Overview.jpg}

CCO Creative Commons by \href{https://isorepublic.com/photo/flying-kick/}{ISO Republic}

\hypertarget{overview-1}{%
\section*{Overview}\label{overview-1}}
\addcontentsline{toc}{section}{Overview}

The second unit focuses on reason. In this unit, the goal is to learn (i) what is an argument, and (ii), how does one identify an argument. The skill of identifying arguments applies to many areas of life and it's important to practice this skill. When you listen to Joe Rogan, Jimmy Kimmel, people on TikTok, Facebook, Fox, and CNN, one should search for the argument and then restructure the argument in its strongest form. Only after that process is complete does one then focus on evaluating the truth of the argument. (We do not cover evaluating the truth of arguments in this unit. Consider enrolling in Phil 103, 105, or 210 to explore how to evaluate arguments).

\hypertarget{topics-1}{%
\subsection*{Topics}\label{topics-1}}
\addcontentsline{toc}{subsection}{Topics}

This unit is divided into the following topics:

\begin{enumerate}
\def\labelenumi{\arabic{enumi}.}
\tightlist
\item
  The Parts of an Argument\\
\item
  The Keywords of an Argument\\
\item
  Labeling an Argument\\
\item
  Restructuring an Argument\\
\item
  Interpreting an Argument\\
\item
  Arguments with ``If\ldots Then''\\
\item
  Identifying Difficult Arguments
\end{enumerate}

\hypertarget{learning-outcomes-1}{%
\subsection*{Learning Outcomes}\label{learning-outcomes-1}}
\addcontentsline{toc}{subsection}{Learning Outcomes}

When you have completed this unit, you should be able to:

\begin{itemize}
\tightlist
\item
  Cultivate skills of logical argumentation.\\
\item
  Develop skills for living a life of wisdom and justice.\\
\item
  Know how to identify arguments by locating the premises and conclusion.\\
\item
  Practice the skills of wisdom and reason, such as humility.
\end{itemize}

\hypertarget{activity-checklist-1}{%
\subsection*{Activity Checklist}\label{activity-checklist-1}}
\addcontentsline{toc}{subsection}{Activity Checklist}

{An Activity Checklist has not been provided}

\hypertarget{resources-1}{%
\subsection*{Resources}\label{resources-1}}
\addcontentsline{toc}{subsection}{Resources}

For this unit, there are no assigned readings.

\hypertarget{the-parts-of-an-argument}{%
\section{The Parts of an Argument}\label{the-parts-of-an-argument}}

An argument is a set of statements organized in a specific way, for example:

\begin{enumerate}
\def\labelenumi{\arabic{enumi}.}
\tightlist
\item
  All humans are mortal. (Premise)\\
\item
  Lucy is human. (Premise)\\
\item
  Therefore, Lucy is mortal. (Conclusion)
\end{enumerate}

The statements above are organized as an argument because one of the statements, the conclusion, is supported by the other set of statements, the premises. By reading these premises and the conclusion carefully, one can see that the premises support the truth of the conclusion. \textbf{For our purposes, the order of the premises does not matter}. For example, the following argument is the same as the argument above:

\begin{enumerate}
\def\labelenumi{\arabic{enumi}.}
\tightlist
\item
  Lucy is human.\\
\item
  All humans are mortal.\\
\item
  Therefore, Lucy is mortal.
\end{enumerate}

The space between the premises and the conclusion is called the \textbf{inference}. The line illustrates when the premises end and where the conclusion follows. The term ``inference'' refers to the relationship between premises and conclusion, and that relationship changes depending on the type of argument in use. Put in a different way, the inference refers to how one moves from premise to conclusion. Sometimes that inference is one of certainty, in that the truth of the premises guarantees the truth of the conclusion; other times that inference is one of probability, in that the truth of the premises raises the probable truth of the conclusion. \textbf{(You are not required to know about the different types of inferences in this class)}.\_

The \textbf{premises} are the ``reasons'' for why one should think the \textbf{conclusion} is true, and therefore the conclusion is being defended with those premises. The premises and conclusion will always be described with \textbf{statements}, and statements are claims that are \textbf{either true or false}. Non-statements, however, do not express claims that are either true or false, and thus non-statements are never premises or conclusions. Knowing the difference between statements and non-statements is important because that knowledge will help with identifying the parts of the argument. For example:

\begin{enumerate}
\def\labelenumi{\arabic{enumi}.}
\tightlist
\item
  Apples are red.\\
\item
  Tony Stark is Iron Man.\\
\item
  Was that a magic trick?\\
\item
  Hey, partner!
\end{enumerate}

The first two examples are statements because they express claims that are either true or false. The third example is a question and does not express a claim that is either true or false. Therefore, the third example is a non-statement. (Of course, if one answered the question by saying, ``that is a magic trick,'' then that answer to the question is a statement that is either true or false). The fourth example is an exclamatory greeting and does not express a claim that is either true or false. Therefore, the fourth example is a non-statement. Remember, then, that arguments are made up of statements and never made up of non-statements.

\begin{caution}
\textbf{Note} that a \textbf{statement} is different from a \textbf{sentence}. A sentence has a subject and a predicate. The subject is what or whom the sentence is about, and the predicate is a description of the subject and contains the verb. Sometimes a whole sentence is one statement, for example:
\end{caution}

\begin{quote}
``The cat is in the hat.''
\end{quote}

That entire sentence is a statement because the sentence is either true or false. Other times a whole sentence includes multiple statements, for example:

\begin{quote}
``If the cat is in the hat, then the fish is in the bowl.''
\end{quote}

In that example, the first part, ``the cat is in the hat,'' is the first statement, and the second part, ``the fish is in the bowl,'' is the second statement. So even though that example is one sentence, the entire sentence includes multiple statements that are either true or false.

More difficult sentences may include both statements and non-statements, for example:

\begin{quote}
``Stop, the train is approaching.''
\end{quote}

In that example, the first part, ``Stop,'' is a non-statement, and the second part, ``the train is approaching,'' is a statement. The reason is because the word ``Stop'' does not express a claim that is either true or false, and the phrase ``the train is approaching'' does express a claim that is either true or false. \textbf{When identifying the premises and conclusion of an argument, one should only focus on statements and always ignore non-statements.}

\hypertarget{activity-identifying-statements-and-non-statements}{%
\subsection*{Activity : Identifying statements and non-statements}\label{activity-identifying-statements-and-non-statements}}
\addcontentsline{toc}{subsection}{Activity : Identifying statements and non-statements}

\begin{reflect}
Which of the following are statements? Which of the following are non-statements?. Identify the statements and non-statements by dragging down the appropriate statement and non-statement to the corresponding box below.
\end{reflect}

\hypertarget{the-keywords-of-an-argument}{%
\section{The Keywords of an Argument}\label{the-keywords-of-an-argument}}

The purpose of Topic 1 was to introduce the elements of an argument, such as statements, non-statements, and sentences. We learned about the premises, conclusion, and inferences (note, again, that we do not cover inferences in this class). Topic 2 discusses the use of keywords to identify the premises and conclusion of an argument.

Some arguments are difficult to locate because the premises and conclusion remain unclear. In those difficult cases, one strategy for identifying the argument is locating \textbf{keywords} indicating the premises and conclusion. Keywords that indicate premises include: \textbf{``All, every, most, some, none, for, because, If\ldots then.''} Keywords that indicate the conclusion are: \textbf{``therefore, so, thus, hence, consequently, it follows that\ldots{}''} Consider the following example:

\begin{enumerate}
\def\labelenumi{\arabic{enumi}.}
\tightlist
\item
  \textbf{All} humans are mortal. (Premise)\\
\item
  Lucy is a human. (Premise)\\
\item
  \textbf{Therefore}, Lucy is mortal. (Conclusion)
\end{enumerate}

In that example, the first premise follows the term ``All,'' and the conclusion follows the term ``Therefore.'' The second premise does not have a keyword, but, in this case, the reader knows that statement is not the conclusion -- because the conclusion was identified already with the keyword ``therefore'' -- and so the second statement must be a premise (this isn't always the case, but we needn't be concerned about that here). The argument is easy to identify because both the premise and conclusion have keywords.

\hypertarget{activity-identifying-premises-and-conclusions}{%
\subsection*{Activity: Identifying premises and conclusions}\label{activity-identifying-premises-and-conclusions}}
\addcontentsline{toc}{subsection}{Activity: Identifying premises and conclusions}

\begin{reflect}
Identify which of the following are premises and which are conclusions by clicking the correct answer.

\textbf{Checklist}

\begin{itemize}
\tightlist
\item
  Focus on statements and ignore non-statements\\
\item
  Locate the premises and conclusion using keywords
\end{itemize}
\end{reflect}

\hypertarget{labeling-an-argument}{%
\section{Labeling an Argument}\label{labeling-an-argument}}

In topics 1 and 2, we learned about the parts of an argument and some initial strategies for identifying arguments. The parts of an argument include statements, which are claims that are either true or false, such as premises and conclusions. We also learned that identifying an argument is simpler when there exist keywords for indicating those premises and conclusions. In Topic 3, we focus on how to label arguments by symbolizing premises and conclusions, for example:

\begin{quote}
``All humans are mortal, and Lucy is a human. So, Lucy is mortal.''
\end{quote}

Unlike the examples of arguments in the topics 1 and 2, notice that this argument is not written vertically but horizontally. Most arguments written in ordinary discourse (e.g.~the news, Facebook, etc.) will be expressed horizontally. That horizontal format already makes the argument more difficult to locate. The first sentence includes two statements and both statements are different premises of the argument. The conclusion is the second sentence indicated by the keyword ``So.'' The objective is to label the premises and conclusion with symbols so that the reader can identify the argument more accurately. In doing so, begin by first labeling the premises and conclusion by writing the letter ``p'' in parentheses at the beginning of the premises, and the letter ``c'' in parentheses next to the conclusion. For example:

\begin{quote}
\textbf{(P)} All humans are mortal, and \textbf{(P)} Lucy is a human. \textbf{(C)} So, Lucy is mortal.
\end{quote}

In that example, ``P'' stands for premise and (C) means conclusion. \textbf{Note again: the order of the premises does not matter in this Unit. One can reverse the order of the premises and produce the same argument}. When reading more complex arguments, labeling the premises and conclusion becomes essential to be sure one correctly identifies the argument.

\hypertarget{learning-label-the-premises-and-conclusions}{%
\subsection*{Learning: Label the premises and conclusions}\label{learning-label-the-premises-and-conclusions}}
\addcontentsline{toc}{subsection}{Learning: Label the premises and conclusions}

\begin{reflect}
Identify which of the following are premises and which are conclusions. Label by filling the blanks using the appropriate letters.

{Checklist:}

\begin{itemize}
\tightlist
\item
  Focus on statements and ignore non-statements.\\
\item
  Locate the premises and conclusion using keywords.\\
\item
  Label the premises and conclusion by inserting the appropriate letter in the blank in front of the statements: P for premise, and C for Conclusion. Be sure to ignore non-statements by inserting an ``X'' in the blank in front of the non-statement.
\end{itemize}

{Example}

\textbf{Argument:}

\begin{quote}
``All conservatives are Christian. Charlie is conservative. Therefore, Charlie is Christian.''
\end{quote}

\textbf{Solution:}

\begin{quote}
\begin{enumerate}
\def\labelenumi{(\Alph{enumi})}
\setcounter{enumi}{15}
\tightlist
\item
  All conservatives are Christian. (P) Charlie is conservative. (C) Therefore, Charlie is Christian.
\end{enumerate}
\end{quote}
\end{reflect}

\hypertarget{restructuring-an-argument}{%
\section{Restructuring an Argument}\label{restructuring-an-argument}}

A major part of our learning in Topics 1-3 focused on the premises and conclusions of an argument. Additionally, we learned about the keywords for identifying those parts, and how to label the premises and conclusion with letters in parentheses. As illustrated in the previous topics, notice that writing an argument in vertical form, with premises on top and the conclusion on the bottom, is the most effective way to illustrate the final argument. For example,

\begin{enumerate}
\def\labelenumi{\arabic{enumi}.}
\tightlist
\item
  All humans are mortal. (P)\\
\item
  Lucy is a human. (P)\\
\item
  Therefore, Lucy is mortal. (C)
\end{enumerate}

That vertical format is valuable for the reader to know exactly what the argument is about. Unfortunately, arguments in ordinary discourse are not written in that vertical fashion - as illustrated in Topic 3, where arguments were written horizontally - and therefore one must restructure the horizontal argument format to a vertical argument format. In Topic 4, we practice all of the methods above with the aim of restructuring the argument vertically.

\hypertarget{activity-label-and-structure-the-argument}{%
\subsection*{Activity: Label and structure the argument}\label{activity-label-and-structure-the-argument}}
\addcontentsline{toc}{subsection}{Activity: Label and structure the argument}

\begin{reflect}
This activity has two steps. The first step is to click/drag the appropriate letter into the box to label the premises and conclusion. In the second step, click/drag the appropriate premise and conclusion to complete the vertical argument.

{Checklist:}

\begin{itemize}
\tightlist
\item
  Focus on statements and ignore non-statements\\
\item
  Locate the premises and conclusion using keywords\\
\item
  Label the premises and conclusion using letters in parentheses.\\
\item
  Restructure the argument vertically, with premises on top and the conclusion on the bottom.
\end{itemize}

{Example:}

\begin{quote}
``Lucy is a teacher. All teachers drink coffee. Therefore, Lucy drinks coffee.''
\end{quote}

\textbf{Step 1:} Label the argument

\begin{quote}
\begin{enumerate}
\def\labelenumi{(\Alph{enumi})}
\setcounter{enumi}{15}
\tightlist
\item
  Lucy is a teacher. (P) All teachers drink coffee. (C) Therefore, Lucy drinks coffee.
\end{enumerate}
\end{quote}

\textbf{Step 2:} Restructure the argument vertically

\begin{enumerate}
\def\labelenumi{\arabic{enumi}.}
\tightlist
\item
  Lucy is a teacher. (P)\\
\item
  All teachers drink coffee. (P)\\
\item
  Therefore, Lucy drinks coffee. (C)
\end{enumerate}

{Practice Exercises}
\end{reflect}

\hypertarget{interpreting-an-argument}{%
\section{Interpreting an Argument}\label{interpreting-an-argument}}

In Topics 1-4, we learned about the parts of the argument, how to locate keywords, label the parts of the argument with letters in parentheses, and how to restructure the argument vertically with premises at the top and conclusion at the bottom.

Another challenge with arguments written in ordinary discourse is not only that they're written horizontally but they lack keywords for identifying premises and conclusions. When there are no keywords for identifying premise and conclusion, the reader must interpret what the author intended to say and then determine which statements are the premises and conclusion. Consider two examples, the first simpler and the second more complex. The first example is the following:

\begin{quote}
``I think Lucy is mortal. She is human and humans are mortal.''
\end{quote}

This example does not include a keyword for the conclusion but rather uses the phrase ``I think.'' Furthermore, note that the conclusion comes first, ``I think Lucy is mortal.'' That informal way of writing the conclusion makes identification more challenging. Premise (1) also includes the pronoun ``she'', which in this case refers to ``Lucy.'' Note also that premises (1) and (2) are statements within the same sentence, respectively, and lack keywords for identification. In most situations, arguments resemble this example in that they are unclear, and one must interpret the argument according to what the author intended to say. If we apply the steps from the previous examples, first label the premises and conclusion with letters:

\begin{quote}
\begin{enumerate}
\def\labelenumi{(\Alph{enumi})}
\setcounter{enumi}{2}
\tightlist
\item
  I think that Lucy is mortal. (P) She is human, and (P) humans are mortal.
\end{enumerate}
\end{quote}

Now that the argument is labeled, order the argument vertically with abbreviations in parentheses, and then \textbf{observe} the added keywords that provide a clear interpretation. For example:

\begin{enumerate}
\def\labelenumi{\arabic{enumi}.}
\tightlist
\item
  \textbf{Lucy} is human. (P)\\
\item
  \textbf{All} humans are mortal. (P)\\
\item
  \textbf{Therefore}, Lucy is mortal. (C)
\end{enumerate}

Now the final version of that argument is clearer. We ordered the premises and conclusion vertically, replaced the pronoun ``she'' with ``Lucy,'' added the keyword ``All'' to premise (2), on the assumption that the author intended to refer to ``all'' humans and not ``some'' humans, and added the conclusion keyword ``Therefore.''

The second more complex example is the following:

\begin{quote}
``Dogs must be man's best friend.'' Dogs are loyal and kind.''
\end{quote}

In this example, there are no relevant keywords, and the conclusion may not be obvious. The task, then, is to carefully interpret the sentences and their relationship with each other to discover (1) does an argument even exist? and (2), if yes, then what are the premises and the conclusion?

Identifying the argument can be difficult. One technique for identifying obscure arguments like the one above is using the ``Why/Because'' strategy. A conclusion can always be followed by asking a hypothetical ``Why?'' and a premise can always be introduced with the term ``Because.'' Consider the following:

\begin{quote}
``Dogs must be man's best friend. (\textbf{Why})?\ldots(\textbf{Because}) Dogs are loyal and (\textbf{Because}) dogs are kind.''
\end{quote}

The first statement, ``Dogs must be man's best friend,'' can be followed by asking ``\textbf{Why} do you think that claim is true?'' The second and third statements can be introduced with the term ``because''. This process demonstrates that the first statement, followed by the term ``why,'' is the conclusion, and the second and third statements, preceded with the term ``because,'' are the premises. So, in difficult cases with a lack of keywords, locate the statements and note which statement best follows with a ``Why?'' (likely the conclusion), and the statements that best precede with a ``Because'' (likely the premises supporting that conclusion).

However, at times that strategy may create confusion because most statements could in principle be followed with a hypothetical ``Why?'' For example:

\begin{quote}
``Dogs must be man's best friend. Dogs are loyal \textbf{(Why?)} and dogs are kind.''
\end{quote}

In that example, the statement followed by the hypothetical ``why?'' is not the conclusion, but one could technically apply the strategy in that case, thereby creating confusion. How does one resolve this confusion? One strategy is to answer the ``Why'' question with the other statements involved, and test and compare each statement with another until the conclusion becomes more apparent. For example, suppose one wishes to test the statement, ``Dogs are loyal,'' to discover if that statement is the conclusion:

\begin{quote}
``Dogs are loyal.'' \textbf{Why?}
\end{quote}

Answer that question by using only the other two statements involved, for example:

\begin{enumerate}
\def\labelenumi{\arabic{enumi}.}
\tightlist
\item
  Dogs are loyal. \textbf{Why? Because} dogs must be man's best friend.\\
\item
  Dogs are loyal. \textbf{Why? Because} dogs are kind.
\end{enumerate}

Neither answer to the question seems accurate, because dog loyalty has to do with other characteristics absent from those statements. That process indicates that the statement, ``dogs are loyal,'' is likely not the conclusion. Repeat the process with the other two statements until a conclusion becomes evident. If there's no apparent conclusion, then that may show the argument is poorly constructed.

A final comment is the terms ``why'' and ``because'' may be used in sentences without invoking an argument. For example,

\begin{quote}
``I'm at the birthday party because I was invited.''
\end{quote}

Those statements do not (really) constitute an argument even though the term ``because'' is used and the term ``why'' is implied.

\hypertarget{activity-label-and-structure-the-arguments-vertically}{%
\subsection*{Activity: Label and structure the arguments vertically}\label{activity-label-and-structure-the-arguments-vertically}}
\addcontentsline{toc}{subsection}{Activity: Label and structure the arguments vertically}

\begin{reflect}
This activity has two steps. The first step is to click/drag the appropriate letter into the box to label the premises and conclusion. In the second step, click/drag the appropriate premise and conclusion to complete the vertical argument.

{Checklist:}

\begin{itemize}
\tightlist
\item
  Focus on statements and ignore non-statements.\\
\item
  Locate the premises and conclusion using keywords.
\end{itemize}

\textbf{NOTE: use the ``Why/Because'' strategy for difficult cases.}

\begin{itemize}
\tightlist
\item
  Label the premises and conclusion using letters in parentheses.\\
\item
  Restructure the argument vertically, with premises on top and the conclusion on the bottom.
\end{itemize}

\textbf{Note the keywords that are added to the premises and conclusions in step 2.}

{Example:}

\begin{quote}
``I think all children are artists. Marcie is a child. Marcie must be an artist.''
\end{quote}

\textbf{Step 1:} Label the arguments

\begin{quote}
\begin{enumerate}
\def\labelenumi{(\Alph{enumi})}
\setcounter{enumi}{15}
\tightlist
\item
  I think all children are artists. (P) Marcie is a child. (C) Marcie must be an artist.
\end{enumerate}
\end{quote}

\textbf{Step 2:} Restructure the argument vertically

\begin{enumerate}
\def\labelenumi{\arabic{enumi}.}
\tightlist
\item
  \textbf{All} children are artists. (P)
\item
  Marcie is a child. (P)
\item
  \textbf{Therefore}, Marcie is an artist. (C)
\end{enumerate}

{Practice Exercises}
\end{reflect}

\hypertarget{arguments-with-ifthen}{%
\section{Arguments with ``If\ldots Then''}\label{arguments-with-ifthen}}

In topics 1-5, we learned about the parts of the argument, and, in doing so, learned some strategies for identifying arguments. However, as arguments become more complicated, one must learn new techniques.

Some arguments use sentences with the terms ``If\ldots Then.'' These types of sentences are called conditional sentences (or conditionals) and often appear in arguments. For example:

\begin{quote}
``\textbf{IF} Willy is a whale, \textbf{THEN} Willy is a mammal.''
\end{quote}

As discussed briefly in an earlier section, note that some sentences may include multiple statements and thus include multiple premises. However, note that conditional sentences, while they may include multiple statements, \textbf{always constitute one premise}. That is, both ``If\ldots Then'' statements in a conditional sentence are not different premises of the argument but are parts of the same premise.

For example:

\begin{enumerate}
\def\labelenumi{\arabic{enumi}.}
\tightlist
\item
  \textbf{If Willy is a whale, then Willy is a mammal.} (P)\\
\item
  Willy is a whale. (P2)\\
\item
  Therefore, Willy is a mammal. (C1)
\end{enumerate}

\textbf{The first premise includes both ``If\ldots Then'' statements}, and both statements in the conditional can be defined. The first statement following the term ``If'' is called the \textbf{antecedent}. That means the thing that comes before. The second statement following the term ``Then'' is called the \textbf{consequent}. That means the thing that comes after, or the thing that follows from the antecedent, for example:

``If Willy is a whale (antecedent), then Willy is a mammal (consequent).''

Conditional sentences are tricky business. While we needn't be concerned with those difficulties in this Unit, three factors deserve attention for the purpose of identifying arguments: (i) notice that the terms ``if'\,' and ``then'' are keywords that may indicate a conditional premise; (ii), notice the conclusion in the argument above, ``Willy is a mammal,'' is smuggled into the second part (the consequent) of the first premise. When identifying a conditional sentence, the consequent of that conditional may be a clue for locating the conclusion (not always, but sometimes); and (iii), conditional sentences can be expressed in English in different ways. So one must learn to identify some of these expressions and practice how to convert them into standard ``If\ldots then'' form.

\textbf{a. Sentences with the term ``if'' in the middle}

\begin{quote}
``Willy is a mammal \textbf{IF} Willy is a whale.''
\end{quote}

\textbf{The term ``if'' introduces the antecedent, and the antecedent always comes first in the conditional sentence}. So, rewrite the sentence above to standard conditional form by moving the term ``if'' and the antecedent, ``Willy is a whale,'' to the beginning of the sentence, and add the term ``then'' to the beginning of the consequent, ``Willy is a mammal'':

\begin{quote}
``\textbf{If} Willy is a whale, \textbf{then} Willy is a mammal.''
\end{quote}

Now the argument can be structured vertically, for example:

\begin{enumerate}
\def\labelenumi{\arabic{enumi}.}
\tightlist
\item
  If Willy is a whale, then Willy is a mammal. (P)\\
\item
  Willy is a whale. (P)\\
\item
  Therefore, Willy is a mammal. (C)
\end{enumerate}

\textbf{b. Sentences with the phrase ``only if'' at the beginning}

\begin{quote}
``\textbf{Only if} Willy is a mammal, is Willy a whale.''
\end{quote}

\textbf{The term ``only if'' introduces the consequent, and the consequent is always located at the end of the conditional sentence}. So, change the term ``only if'' to ``then'' and move its consequent, ``Willy is a mammal,'' to the end. Then move the last statement, ``Willy is a whale,'' to the beginning with the term ``If'' added in front because that statement is the antecedent.

\begin{quote}
``If Willy is a whale, then Willy is a mammal.''
\end{quote}

\textbf{c.~Sentences with the phrase ``only if'' in the middle:}

\begin{quote}
``Willy is a whale only if Willy is a mammal.''
\end{quote}

Note, again, that the phrase ``\textbf{only if}'' introduces the consequent. Rewrite the sentence by converting ``only if'' to ``then,'' and add the term ``If'' to the beginning.

\begin{quote}
``If Willy is a whale, then Willy is a mammal.''
\end{quote}

\textbf{d.~Sentences with the term ``unless'' at the beginning}

\begin{quote}
``\textbf{Unless} you eat vegetables, there's no dessert.''
\end{quote}

\textbf{The term ``unless'' introduces the antecedent with a negation} (i.e.~``not'') and is translated to ``\textbf{if not.}'' So, rewrite by replacing ``unless'' with ``if not'' and add the term ``then'' to the beginning of the consequent.

\begin{quote}
``\textbf{If not} you eat vegetables, \textbf{then} there's no dessert.''
\end{quote}

Obviously, that sentence above, although logically correct, must be rewritten to make sense:

\begin{quote}
``\textbf{If} you do \textbf{not} eat vegetables, then there's no dessert.''
\end{quote}

\textbf{e. Sentences with the term ``unless'' in the middle}

\begin{quote}
``There's no dessert UNLESS you eat vegetables.''
\end{quote}

Rewrite by changing ``unless'' to ``if not,'' move ``if not'' and its statement, ``you eat vegetables,'' to the beginning to make the antecedent; add ``then'' to the second statement, ``there's no dessert,'' to make the consequent.

\begin{quote}
``\textbf{If} you do not eat vegetables, \textbf{then} there's no dessert.''
\end{quote}

\textbf{f.~Sentences with the phrase ``provided that''}

\begin{quote}
``You may eat dessert \textbf{provided that} you eat vegetables.''
\end{quote}

\textbf{The phrase ``provided that'' introduces the antecedent}. Rewrite by changing ``provided that'' to ``if'' and move ``if'' and its statement, ``you eat vegetables,'' to the beginning to make the antecedent; add the term ``then'' to the beginning of the other statement, ``You may eat dessert,'' to make the consequent, for example:

\begin{quote}
``\textbf{If} you eat vegetables, \textbf{then} you may eat dessert.''
\end{quote}

Now you know how to identify different ways of expressing ``If\ldots then,'' and how to rewrite those expressions to simplify ordering the argument vertically.

\textbf{Conditional Sentences Cheat Sheet:}

\begin{enumerate}
\def\labelenumi{\arabic{enumi}.}
\tightlist
\item
  Antecedent: the first statement in a conditional sentence introduced with the term ``If''\\
\item
  Consequent: the second statement in a conditional sentence introduced with the term ``then''\\
\item
  ``If'' = always introduces the antecedent, regardless of location\\
\item
  ``Only if'' = always introduces the consequent, regardless of location\\
\item
  ``Unless'' = always introduces the antecedent with a negation, regardless of location, translated as ``if not''\\
\item
  ``Provided that'' = always introduces the antecedent regardless of location
\end{enumerate}

\hypertarget{activity-arguments-with-conditionals}{%
\subsection*{Activity: Arguments with conditionals}\label{activity-arguments-with-conditionals}}
\addcontentsline{toc}{subsection}{Activity: Arguments with conditionals}

\begin{reflect}
This activity has two steps. The first step is to click/drag the appropriate letter into the box to label the premises and conclusion. In the second step, click/drag the appropriate premise and conclusion to complete the vertical argument.

{Checklist}

\begin{itemize}
\tightlist
\item
  Focus on statements and ignore non-statements.\\
\item
  Locate the premises and conclusion using keywords.

  \begin{itemize}
  \tightlist
  \item
    \textbf{NOTE:} the terms ``if'' and ``then'' may be keywords for indicating a conditional sentence and a premise of an argument. NOTE: expressions such as ``only if'' and ``unless'' (in green) indicate conditionals. Note how those expressions (in green) will change to standard ``If/Then'' conditionals when ordering the argument vertically.\\
  \end{itemize}
\item
  Label the premises and conclusion using letters in parentheses.\\
\item
  Restructure the argument vertically, with premises on top and the conclusion beneath.

  \begin{itemize}
  \tightlist
  \item
    \textbf{Notice the changes to the wording in some of the premises and conclusions.}
  \end{itemize}
\end{itemize}

{Example 1:}

\begin{quote}
``If that thing walks like a duck, then that thing is a duck. That thing walks like a duck. Therefore, that thing is a duck.
\end{quote}

\textbf{Step 1:} Label the argument

\begin{quote}
\begin{enumerate}
\def\labelenumi{(\Alph{enumi})}
\setcounter{enumi}{15}
\tightlist
\item
  If that thing walks like a duck, then that thing is a duck. (P) That thing walks like a duck. (C) Therefore, that thing is a duck.
\end{enumerate}
\end{quote}

\textbf{Step 2:} Restructure the argument vertically

\begin{enumerate}
\def\labelenumi{\arabic{enumi}.}
\tightlist
\item
  If that animal walks like a duck, then that animal is a duck.\\
\item
  That animal walks like a duck.\\
\item
  Therefore, that animal is a duck.
\end{enumerate}

{Example 2:}

\begin{quote}
``I'll marry you only if you watch Downton Abbey. What!? You refuse to watch Downton Abbey. You're kidding me! Well, I refuse to marry you.''
\end{quote}

\textbf{Step 1:} Label the argument (notice the change from ``only if'' to ``if then'')

\begin{quote}
\begin{enumerate}
\def\labelenumi{(\Alph{enumi})}
\setcounter{enumi}{15}
\tightlist
\item
  If I marry you, then you must watch Downton Abbey. (P) You refuse to watch Downton Abbey. (C) I will not marry you.
\end{enumerate}
\end{quote}

\textbf{Step 2:} Restructure the argument vertically

\begin{enumerate}
\def\labelenumi{\arabic{enumi}.}
\tightlist
\item
  If I marry you, then you must watch Downton Abbey\\
\item
  You refuse to watch Downton Abbey\\
\item
  Thus, I will not marry you.
\end{enumerate}

Practice Exercises \{-\}
\end{reflect}

\hypertarget{identifying-difficult-arguments}{%
\section{Identifying Difficult Arguments}\label{identifying-difficult-arguments}}

In the previous topics, we learned about identifying arguments. In this final topic, we apply those tools to identifying more difficult arguments. The examples below are more challenging because the arguments are not only greater length, but the arguments can be difficult to identify. One must carefully sift through the statements to decipher (i) does an argument exist, and (ii) if yes, then what is the conclusion(s) and the premises supporting that conclusion(s)? One must employ all the methods to achieve those tasks: such as ignoring non-statements, search for keywords that indicate premises and conclusions (highlight them, if that helps), label the premises and conclusions in parentheses, interpret what the author intended to say, and restructure the argument vertically in its strongest form. Note: more complex arguments often include multiple conclusions and those conclusions may operate as premises of the same argument. Note: when encountering a lengthy passage, read the entire passage through multiple times before attempting to identify the parts of the argument. The reason for that is to gain a broader understanding of what the author intended to say.\\
\#\#\# Activity: Difficult Arguments \{-\}

\begin{reflect}
{Checklist:}

\begin{itemize}
\tightlist
\item
  Focus on statements and ignore non-statements\\
\item
  Locate the premises and conclusion using keywords.\\
\item
  Label the premises and conclusion using letters and numerals in parentheses.\\
\item
  Restructure the argument vertically, with premises on top and the conclusion beneath
\end{itemize}

\begin{center}\rule{0.5\linewidth}{0.5pt}\end{center}

\begin{center}\rule{0.5\linewidth}{0.5pt}\end{center}

\begin{center}\rule{0.5\linewidth}{0.5pt}\end{center}

Example 3 is more challenging. In these types of cases, read through the entire passage several times to decipher the argument exactly. Take time to draft multiple arguments vertically and observe which version best represents what the author intended to say.
\end{reflect}

\hypertarget{activity-practice-exercise}{%
\subsection*{Activity: Practice Exercise}\label{activity-practice-exercise}}
\addcontentsline{toc}{subsection}{Activity: Practice Exercise}

\textbf{Instructions:} Read the following excerpts and then locate the argument.

\begin{enumerate}
\def\labelenumi{\arabic{enumi}.}
\item
  ``Be smart about the shrimp you eat. Thankfully in Canada this is easier than in many places. Most of Canada's shrimp fisheries are considered to be ecologically sustainable with minimal bycatch, although some use otter trawls which can severely damage sea floor habitats. Canada is home to one of the most sustainable prawn fisheries in the world -- the B.C spot prawn fishery. This fishery uses traps that do not result in as much bycatch or habitat damage. We also have programs like Oceans Wise that will tell you if the shrimp you want to buy for the barbecue or order in a restaurant won't harm the oceans they come from. Shrimp should be something special we eat in celebration of special events like World Oceans Day! Fortunately the timing coincides with the B.C spot prawn season. Yes, you will pay more for the shrimp you eat but the oceans will pay less for your choices. Your long-term gain will be appreciating and eating other marine life for much longer.'' (Sarah Foster, 2014)
\item
  ``Imagine you wake up in the morning and find yourself back-to-back in bed with an unconscious violinist. A famous unconscious violinist. He has been found to have a fatal kidney ailment, and the Society of Music Lovers has canvassed all the available medical records and found that you alone have the right blood type to help. They therefore have kidnapped you, and last night the violinist's circulatory system was plugged into yours, so that your kidneys can be used to extract poisons from his blood as well as your own. If he is unplugged from you now, he will die; but in nine months he will have recovered from his ailment, and can safely be unplugged from you. Since most people would agree that you would be morally justified in unplugging yourself from the violinist, it also follows that it would be morally justified for a woman to abort (i.e.~unplug herself from) her baby.'' (Thomson, 1971)
\item
  ``My son, if you accept my words and store up my commands, turning your ear to wisdom and applying your heart to understanding - indeed, if you call out for insight and cry aloud for understanding, and if you look for it as for silver and search for it as for hidden treasure, then you will understand the fear of the Lord and find knowledge of God.'' You did all those things. So, you understand the fear of the Lord and have found knowledge of God.
\item
  Dr.~Jeckel {[}is{]} the world's leading expert on schizophrenia and has suggested that the mental disorder is a direct result of psychological trauma suffered during early childhood. He sounds like he knows what he's talking about, so that must be the cause of schizophrenia.
\item
  Not all forms of gender discrimination are unethical. There are number of exclusively male or female fitness clubs around the country utilize by religious individuals who shun the meat market scene. If a woman wants to pare herself the embarrassment of being ogled in her sports bra while doing thigh-thrusts, it is her right to work out with women only. Similarly, if a man wants to spare himself the temptation of working out with lingerie models, he should be allowed membership to strictly male fitness clubs. It would be unreasonable to require non-discrimination of these private clubs, or to make them build separate families to accommodate everyone.'' (Arizona Daily Wildcat)
\item
  ``I'm seriously thinking about voting for trump, and here is why. I firmly believe that our system of government is deeply flawed, if not completely broken. Yet we still keep voting for the same type of people. If Trump wins, there's a good chance the whole thing will collapse from his absurdity. Then maybe we could start over and build something better that works. A vote for trump is a vote for full system breakdown, which I believe is exactly what we need.''
\item
  ``Is there scientific evidence that prayer really works?\ldots The problem with\ldots any so-called controlled experiment regarding prayer is that there can be no such thing as a controlled experiment concerning prayer. You can never divide people into groups that received prayer and those that did not. The main reason is that there is no way to know that someone did no receive prayer. How would anyone know that some distant relative was not praying for a member of the group\ldots identified as having received no prayer?'' There is no scientific evidence that prayer words. (Free Inquiry, 1997)
\item
  ``Current-day Christians use violence to spread their right-to-life message. These Christians, often referred to as the religious right, are well-known for violent demonstrations against Planned Parenthood and other abortion clinics. Doctors and personal are threatened with death, clinics have been bombed there have even been cases of doctors being murdered.'' (Letter to the Editor, Daily Wildcat, 17, September, 2002).
\item
  ``Life on Earth today is better than it has ever been. We have technologies that our grandparents could only dream of. Life expectancies keep going up and up. And, despite all the criticisms, I just don't think there's anything wrong with our consumer culture.'' (MacDonald and Vaughn, 2016).
\item
  ``Allow me to explain to you why I think that hockey is the greatest sport ever in the history of the entire world. It is incredibly fast paced since skating allows the players to move at great speeds. Also, there is a lot of skill involved in controlling such a small puck with something like a hockey stick with so much precision. Lastly, the checking and even fighting makes the sport very physical and exciting to watch.'' (MacDonald and Vaughn, 2016).
\item
  ``The worst calamity that will befall the world in the next 20 years will be the use of small nuclear weapons by terrorists or rogue states. The death toll from such a state of affairs is likely to be higher than that of any other kind of human devastation. The United Nations just issues a report that comes to the same conclusion. We should act now to prevent the proliferation of nuclear weapons and nuclear-weapons-grade material from falling into the wrong hands'' (MacDonald and Vaughn, 2016)
\item
  ``Suffering and death from a lack of food, shelter, and medical care are bad. If it is in your power to prevent something bad from happening, without sacrificing anything nearly as important, it is wrong not to do so. By donating to aid agencies, you can prevent suffering and death from lack of food, shelter, and medical care, without sacrificing anything nearly as important. Therefore, if you do not donate to aid agencies, you are doing something wrong.'' (Singer, 2009).
\item
  Policies and precedents on assisted death may have oppressive consequences for women. First, almost all of the legal cases involving assisted death in Canada and the United States have involved women. Moreover, the majority of Dr Kevorkian's ``clients'' have been women. Further, women's wishes concerning the withholding and withdrawal of life-sustaining treatment have been treated differently than men's by American courts.'' (Downie and Sherwin, 1996)
\item
  ``Look around the world: contemplate the whole and every part of it: you will find it to be nothing but one great machine, subdivided into an infinite number of lesser machines, which again admit of subdivisions, to a degree beyond what humans senses and faculties can trace and explain. All these various machines, and even their most minute parts, are adjusted to each other with an accuracy, which ravishes into admiration all men, who have ever contemplated them. The curious adapting of means to ends, throughout all nature, resembles exactly, though it much exceeds, the production of human contrivance; of human design, thought, wisdom, and intelligence. Since therefore the effects resemble each other, we are led to infer, by all the rules of analogy, that the causes also resemble; and that the Author of Nature is somewhat similar to the mind of men; though possessed of much larger factulies, proportioned to the grandeur of the work, which he has executed. By this argument a posteriori, and by this argument alone, do we prove at once the existence of a Deity, and his similarity to human mind and intelligence.'' (Cleanthes, in David Hume's Dialogues Concerning Natural Religion)
\item
  ``If we see a house, Cleanthes, we conclude, with the greatest certainty, that it had an architect or builder because this is precisely the species of effect which we have experienced to proceed from that species of cause. But surely you will not affirm that the universe bears such a resemblance to a house that we can with the same certainty infer a similar cause, or that the analogy here is entire and perfect. The dissimilitude is so striking that the utmost you can here pretend to is a guess\ldots{}'' (Philo, in David Hume's Dialogues Concerning Natural Religion)
\end{enumerate}

\hypertarget{unit-2-summary}{%
\section*{Unit 2 Summary}\label{unit-2-summary}}
\addcontentsline{toc}{section}{Unit 2 Summary}

Unit 2 focused on how to identify arguments. Learning to identify arguments is valuable because sometimes we fail to both identify and understand an opponent's argument before raising objections. Arguments are statements structured in a specific way, such that one the of the statements, the conclusion, is supported by the other statements, the premises. Statements are claims that are either true or false and can be expressed using sentences, subordinate clauses, and conditionals (to name a few), but never exclamatory sentences and questions. Students should know how to locate keywords for identifying premises and conclusion, know how to label both with letters in parentheses, and then order the premises and conclusion vertically. Students should have some knowledge about how to interpret and identify arguments in more complicated passages by using the ``why/because'' strategy.

\hypertarget{assessment-1}{%
\section*{Assessment}\label{assessment-1}}
\addcontentsline{toc}{section}{Assessment}

{Content missing}

\hypertarget{checking-your-learning-1}{%
\section*{Checking Your Learning}\label{checking-your-learning-1}}
\addcontentsline{toc}{section}{Checking Your Learning}

{Content missing}

\hypertarget{faith}{%
\chapter{Faith}\label{faith}}

\includegraphics{assets/u3/Unit3Overview.jpg}

CCO Creative Commons by \href{https://isorepublic.com/photo/hand-reach/}{ISO Republic}

\hypertarget{overview-2}{%
\section{Overview}\label{overview-2}}

The third unit examines the idea of faith. The aim in this unit is to (i) consider some definitions of faith; and (ii), inquire if, and in what way, faith coincides with wisdom and reason. One idea associated with the term faith is belief. To say that one has ``faith in God'' may imply that one believes that God exists. That view of faith centers on intellectual assent by affirming a statement about God. However, the term faith can also refer to a personal trust in something or someone. Thus, the phrase ``faith in God'' may also refer to trusting God for some state of affairs. Those two initial definitions of faith are different and yet may apply in the same instance. One may possess ``faith in God'' and in doing so believe both that God exists and trust God to bring about some state of affairs. Or one may believe that God exists, but maintain that God cannot be trusted to bring about some state of affairs. The term faith can also mean hope. According to that view, to have ``faith in God'' means to desperately hope that God exists and will bring about some good state of affairs, even if one lacks belief that God exists. That third definition of faith as hope is often distinguished from faith as belief. Finally, the term faith may refer to other ideas besides belief and hope. To live a life of faith may require that a person act a certain kind of way; if that person fails to live a life of faith, then they in fact lack faith (even if they ``believe'' accordingly). This final view of defining faith focuses less on intellectual assent - the types of beliefs and hopes one affirms - and more on behavior.

\hypertarget{topics-2}{%
\subsection*{Topics}\label{topics-2}}
\addcontentsline{toc}{subsection}{Topics}

This unit is divided into the following topics:

\begin{enumerate}
\def\labelenumi{\arabic{enumi}.}
\tightlist
\item
  Faith, Hope, Doubt\\
\item
  Ethics in the Christian Scriptures
\end{enumerate}

\hypertarget{learning-outcomes-2}{%
\subsection*{Learning Outcomes}\label{learning-outcomes-2}}
\addcontentsline{toc}{subsection}{Learning Outcomes}

When you have completed this unit, you should be able to:

\begin{itemize}
\tightlist
\item
  Develop skills for living a life of wisdom and justice\\
\item
  Develop and enhance skills of written communication\\
\item
  Practice the skills of wisdom and reason, such as humility\\
\item
  Enhance skills of wisdom and reason to live life within the context of justice and faith.\\
\item
  Understand how to become a well-rounded person and citizen.
\end{itemize}

\hypertarget{activity-checklist-2}{%
\subsection*{Activity Checklist}\label{activity-checklist-2}}
\addcontentsline{toc}{subsection}{Activity Checklist}

{An Activity Checklist has not been provided}

\hypertarget{resources-2}{%
\subsection*{Resources}\label{resources-2}}
\addcontentsline{toc}{subsection}{Resources}

\begin{itemize}
\tightlist
\item
  Pojman, L. P., \& Rea, M. C. (2008). ``Faith, Hope and Doubt'' in Philosophy of religion: An anthology.\\
\item
  Boyd, C et. al.,(2018) ``Ethics in the Christian Scriptures'' chapter 3 in Christian Ethics and Moral Philosophy: An Introduction to Issues and Approaches
\end{itemize}

\hypertarget{faith-hope-doubt}{%
\section{Faith, Hope, Doubt}\label{faith-hope-doubt}}

\hypertarget{activity-reading-5}{%
\subsection*{Activity: Reading}\label{activity-reading-5}}
\addcontentsline{toc}{subsection}{Activity: Reading}

\begin{reflect}
{Specific reading assignment is not given.}
\end{reflect}

In this topic, we explore a first way to think about faith. For Pojman, religious faith - particularly in the midst of doubt - entails a hope that the claims of that faith are true (e.g.. in the case of Christianity, that creation is restored, forgiveness, and resurrection). Pojman's view of faith differs from some traditional definitions of faith that focus on affirming correct beliefs. According to Pojman, faith requires hope and not belief. One reason Pojman affirms that position is because the acquisition of belief is largely beyond one's control and thus one cannot be held responsible for those beliefs. For example, the belief that ``I see a tree'' is something the world generates in the mind through perception. One cannot help but believe that they see a tree in the appropriate circumstances. Belief in God may resemble the example of perceiving the tree in that one cannot help but believe in God or disbelieve in God, depending on the appropriate circumstances. By contrast, Pojman argues that hope is within one's control and thus one can be held responsible for their hopes. So, for Pojman, a person who desperately hopes in the truth of Christianity, who takes appropriate action in light of that hope, even though they do not believe the truth of Christianity, may find themselves in God's good company.

\hypertarget{highlights-from-the-reading-6}{%
\subsubsection*{Highlights from the reading}\label{highlights-from-the-reading-6}}
\addcontentsline{toc}{subsubsection}{Highlights from the reading}

\begin{enumerate}
\def\labelenumi{\arabic{enumi}.}
\tightlist
\item
  ``Traditionally, orthodox Christianity has claimed (1) that faith in God and Christ entails belief that God exists and that Christ is God incarnate and (2) that without faith we are damned to eternal hell. Thus, doubt is an unacceptable propositional attitude.''
\item
  ``Beliefs are not chosen but occur involuntarily as responses to states of affairs in the world.''\\
\item
  ``When a person acquires a belief, the world forces itself upon him.''\\
\item
  ``There is something incoherent in stating that one can obtain or sustain a belief in full consciousness simply by a bais act of the will - that is, purposefully disregarding the evidence\ldots{}''\\
\item
  Suppose I say that I believe I have \$1000,000 in my checking account, and suppose that when you point out to me that there is no reason to believe this, I respond, ``I know that there is not the slightest reason to suppose that there is \$1000,000 in my checking account, but I believe it anyway, simply because I want to.''\\
\item
  ``If we have a moral duty not to volit {[}that is, to hold beliefs only because of our will to do so{]} but to seek the truth impartially and passionately, then we ought not to obtain religious belief by willing to have them; instead we should follow the best evidence we can get.''\\
\item
  ``The point may be put more simply. Suppose you are fleeing a murderous gang of desperados who are betn on your annihilation. You come to the edge of a cliff that overlooks a yawning gorge. You find a rope spanning the gorge - tied to a tree on the cliff on the opposite side - and a man who announces that he is a tight-rope walker and can carry you over the gorge on the rope. He doesn't look as if he can do it, so you wonder whether he is insane or simply overconfident. He takes a few steps on the rope to assure you that he can balance himself. You agree that it's possible that he can navigate the rope across the gorge, but you have grave doubts about whether he can carry you. But your options are limited. Soon your pursuers will be upon you. You must decide. Whereas you still don't believe that the `tight-rope walker' can save you, you decide to trust him. You place your faith in his ability, climb on his back, close your eyes, and do your best to relax and obey his commands in adjusting your body as he steps onto the rope. You have a profound, even desperate, hope that he will be successful. This is how I see religious hope functioning in the midst of doubt.''
\end{enumerate}

\hypertarget{activity-watch-and-reflect-6}{%
\subsection*{Activity: Watch and Reflect}\label{activity-watch-and-reflect-6}}
\addcontentsline{toc}{subsection}{Activity: Watch and Reflect}

\begin{reflect}
Please take a moment to watch the following video on Doubt and Faith by Professor Richard Swinburne. The video duration is 6:17 minutes long and is an optional resource.

\href{Swinburne:\%20On\%20Doubt\%20and\%20Faith}{Watch: \emph{Swinburne: On Doubt and Faith}}
\end{reflect}

\hypertarget{activity-take-notes-for-your-final-paper}{%
\subsection*{Activity: Take Notes for your Final Paper}\label{activity-take-notes-for-your-final-paper}}
\addcontentsline{toc}{subsection}{Activity: Take Notes for your Final Paper}

\begin{reflect}
Answering the following questions is optional. Answering these questions and saving them as a document will allow you to take notes on the readings in preparation for your final paper due at the end of Unit 3.
\end{reflect}

\hypertarget{ethics-in-the-christian-scriptures}{%
\section{Ethics in the Christian Scriptures}\label{ethics-in-the-christian-scriptures}}

\hypertarget{activity-reading-6}{%
\subsection*{Activity: Reading}\label{activity-reading-6}}
\addcontentsline{toc}{subsection}{Activity: Reading}

\begin{reflect}
{Specific reading assignment is not given.}
\end{reflect}

The second topic focuses on another definition of faith. While Boyd and Thorsen's chapter deals largely with Christian ethics, a valuable topic in its own right, especially noting the relationship between ethics and wisdom, the chapter includes a valuable discussion about the nature of faith. In some Christian traditions, faith coincides with salvation. That is, faith in God is motivated in part to receive forgiveness and secure the afterlife. In some cases, that commitment to faith for salvation aligns with understanding faith as belief and trust - the first two definitions of faith described above. However, Boyd and Thorsen briefly summarize a different view of faith (and, indeed, salvation) such that faith requires living a certain kind of life - a life devoted to Christian virtue. Christian virtue focuses on the ethics of love towards God, other individuals (particularly the neediest) and to society as a whole. Thus, Boyd and Thorsen's chapter provides valuable insight into Christian wisdom. A Christian should focus not only on the principles of wisdom described in Unit 1, but should reflect on how those principles of wisdom align with Christian faith and moral responsibility towards God, individuals, and society.

\hypertarget{highlights-from-the-reading-7}{%
\subsection*{Highlights from the reading}\label{highlights-from-the-reading-7}}
\addcontentsline{toc}{subsection}{Highlights from the reading}

\begin{enumerate}
\def\labelenumi{\arabic{enumi}.}
\tightlist
\item
  ``\ldots the sermon {[}on the mount{]} culminates in what is known as the golden rule: ``In everything do to others as you would have them do to you.'' (41)\\
\item
  ``From this brief survey of how Christians have interpreted the Sermon on the Mount, we can see that there is no unanimity with regard to how to understand Jesus's teachings in general and his ethics in particular. This does mean that we are without guidance with regard to learning about Jesus's beliefs, values, and practices.'' (44-5)\\
\item
  ``But Christians should be humble when making their claims about what Jesus and the Christian Scriptures state precisely about any particular ethical issue.''
\item
  ``The so-called new perspective on Paul emphasizes a greater continuity between Paul's view of grace and faith, including God's role in providing for people's salvation, and the conditionality of their choosing to assent, repent, and act faithfully'' (46).\\
\item
  ``Although Christians have had divergent opinions of the degree to which obedience to God's laws relates to the gospel message of salvation, they agree that people are responsible for their moral choices.'' (47)\\
\item
  ``The greatest ethical command that Jesus gave had to do with love, and this principle is echoed throughout the Christian Scriptures.'' (48)
\end{enumerate}

\hypertarget{activity-watch-and-reflect-7}{%
\subsection*{Activity: Watch and Reflect}\label{activity-watch-and-reflect-7}}
\addcontentsline{toc}{subsection}{Activity: Watch and Reflect}

\begin{reflect}
Please take a moment to watch the following video on Why Consider Other Christian Views of the Bible? by Don Thorsen. Please consider thet the video duration is 1:17 minutes long and is an optional resource.

\href{https://www.youtube.com/watch?v=pHQiSwX-dCY}{Watch: \emph{Why Consider Other Christian Views of the Bible?}}
\end{reflect}

\hypertarget{activity-take-notes-for-your-final-paper-1}{%
\subsection*{Activity: Take notes for your Final Paper}\label{activity-take-notes-for-your-final-paper-1}}
\addcontentsline{toc}{subsection}{Activity: Take notes for your Final Paper}

\begin{reflect}
Answering the following questions is optional. Answering these questions and saving them as a document will allow you to take notes on the readings in preparation for your final paper due at the end of Unit 3.
\end{reflect}

\hypertarget{unit-3-summary}{%
\section*{Unit 3 Summary}\label{unit-3-summary}}
\addcontentsline{toc}{section}{Unit 3 Summary}

Unit 3 focused on faith. The term ``faith'' is ambiguous and it's important to explore a range of definitions and reflect how those definitions apply to life. Students explored two ways to think about faith. Pojman's article defines faith as hope and not belief. Some people find believing in God difficult, either because of the evidence against God's existence or because they do not trust God, or both. The second article focused on defining faith as action and not only belief and trust. A person of faith acts a certain way, and this definition of faith is related to Christian salvation. A person inherits the afterlife by how they live and not only what they believe.

\hypertarget{assessment-2}{%
\section*{Assessment}\label{assessment-2}}
\addcontentsline{toc}{section}{Assessment}

{Content missing}

\hypertarget{checking-your-learning-2}{%
\section*{Checking Your Learning}\label{checking-your-learning-2}}
\addcontentsline{toc}{section}{Checking Your Learning}

{Content missing}

\hypertarget{assessment-3}{%
\chapter*{Assessment}\label{assessment-3}}
\addcontentsline{toc}{chapter}{Assessment}

The following assignments are opportunities for learners to demonstrate their understanding of the course outcomes. Please confirm assignment details with your instructor, referring to the course syllabus.

Note that Assignment dropboxes are located in Moodle. Also refer to the Course Schedule in Moodle for the specific due dates.

\hypertarget{assignment}{%
\section*{Assignment:}\label{assignment}}
\addcontentsline{toc}{section}{Assignment:}

\hypertarget{grading-criteria}{%
\subsection*{Grading Criteria}\label{grading-criteria}}
\addcontentsline{toc}{subsection}{Grading Criteria}

See the following rubric that explains how your assignment will be evaluated. Also available as a \href{assets/assessment/Identity-as-a-Teacher-RUBRIC.pdf}{pdf}

\textbf{APA/WRITING}

\textbf{Unsatisfactory:} Paper does not model language and conventions used in scholarly literature. Writing is not well-organized. Several errors in grammar or composition. Sources are not cited. APA citations are not appropriately formatted.

\textbf{Developing:} Paper partially models language and conventions used in scholarly literature. Writing is somewhat well organized and includes some errors in grammar or composition. Not all sources cited. APA citations are generally formatted correctly, with several errors.

\textbf{Proficient:} \emph{Paper consistently models language and conventions used in scholarly literature. Writing is well-organized and includes few (if any) errors in grammar or composition. All resources are appropriately cited (including in-text citations and bibliography information). Few (if any) errors in APA citations.}

\textbf{Exemplary:} Paper is an exemplar of language and conventions used in scholarly literature. Writing is well-organized and free of errors in grammar or composition. All resources are appropriately cited. No errors in APA format.

\hypertarget{statement-of-teaching-identity}{%
\subsubsection*{STATEMENT OF TEACHING IDENTITY}\label{statement-of-teaching-identity}}
\addcontentsline{toc}{subsubsection}{STATEMENT OF TEACHING IDENTITY}

\textbf{Unsatisfactory:} Does not provide a statement about identity as a teacher/facilitator

\textbf{Developing:} Provides an unclear statement about identity as a teacher/facilitator.

\textbf{Proficient:} \emph{Provides a clear, concise, and powerful statement about identity as a teacher/facilitator.}

\textbf{Exemplary:} Provides a clear, concise, and powerful statement about identity as a teacher/facilitator. Statement incorporates theory or research from course materials.

\hypertarget{developing-a-cohesive-and-logical-academic-argument}{%
\subsubsection*{DEVELOPING A COHESIVE AND LOGICAL ACADEMIC ARGUMENT}\label{developing-a-cohesive-and-logical-academic-argument}}
\addcontentsline{toc}{subsubsection}{DEVELOPING A COHESIVE AND LOGICAL ACADEMIC ARGUMENT}

\textbf{Unsatisfactory:} Does not make a focused, cohesive, or logical academic argument. Paper is confusing, and is missing an introduction, body, or conclusion. Transitions between sections and ideas are missing.

\textbf{Developing:} Makes an academic argument that is only partially focused, cohesive and logical. Paper is generally organized, but is missing an introduction, body, or conclusion. Transitions between sections and ideas are unclear.

\textbf{Proficient:} \emph{Makes a focused, cohesive, logical academic argument. Paper is effectively organized and includes an introduction, body, and conclusion. Transitions between sections and ideas are clear.}

\textbf{Exemplary:} Makes a focused, cohesive, logical and compelling academic argument. Paper is effectively organized and includes an introduction, body, and conclusion. Transitions between sections and ideas are clear, and build on each other.

\hypertarget{analysis-of-identity-as-a-teacher}{%
\subsubsection*{ANALYSIS OF IDENTITY AS A TEACHER}\label{analysis-of-identity-as-a-teacher}}
\addcontentsline{toc}{subsubsection}{ANALYSIS OF IDENTITY AS A TEACHER}

\textbf{Unsatisfactory:} Does not include three important aspects of identity as a teacher/facilitator. Does not include an analysis.

\textbf{Developing:} Lists but does not discuss three important aspects of identity as a teacher/facilitator. Includes a partial analysis.

\textbf{Proficient:} \emph{Includes a detailed discussion of three important aspects of identity as a teacher/facilitator. Includes thoughtful analysis of each of the three elements.}

\textbf{Exemplary:} Includes a detailed discussion of three important aspects of identity as a teacher/facilitator. Includes a thoughtful analysis, integrating scholarly literature to support analysis and furthering scholarly thinking related to teacher identity.

\hypertarget{scholarly-integration}{%
\subsubsection*{SCHOLARLY INTEGRATION}\label{scholarly-integration}}
\addcontentsline{toc}{subsubsection}{SCHOLARLY INTEGRATION}

\textbf{Unsatisfactory:} Does not integrate references to support claims and assertions made in the paper.

\textbf{Developing:} Integrates references to support some of the claims and assertions made in the paper.

\textbf{Proficient:} \emph{Integrates references to support claims and assertions made in the paper.}

\textbf{Exemplary:} Integrates references to support claims and assertions made in the paper, effectively synthesizing different perspectives and research results from scholarly sources.

\begin{longtable}[]{@{}
  >{\raggedright\arraybackslash}p{(\columnwidth - 8\tabcolsep) * \real{0.2000}}
  >{\raggedright\arraybackslash}p{(\columnwidth - 8\tabcolsep) * \real{0.2000}}
  >{\raggedright\arraybackslash}p{(\columnwidth - 8\tabcolsep) * \real{0.2000}}
  >{\raggedright\arraybackslash}p{(\columnwidth - 8\tabcolsep) * \real{0.2000}}
  >{\raggedright\arraybackslash}p{(\columnwidth - 8\tabcolsep) * \real{0.2000}}@{}}
\toprule\noalign{}
\begin{minipage}[b]{\linewidth}\raggedright
\textbf{TOTAL}
\end{minipage} & \begin{minipage}[b]{\linewidth}\raggedright
\textbf{0 = 0\% (F)}
\end{minipage} & \begin{minipage}[b]{\linewidth}\raggedright
\textbf{10 = 50\% (C)}
\end{minipage} & \begin{minipage}[b]{\linewidth}\raggedright
\textbf{15 = 75 (B)}
\end{minipage} & \begin{minipage}[b]{\linewidth}\raggedright
\textbf{20 = 100\% (A+)}
\end{minipage} \\
\midrule\noalign{}
\endhead
\bottomrule\noalign{}
\endlastfoot
\end{longtable}

\begin{center}\rule{0.5\linewidth}{0.5pt}\end{center}

\hypertarget{assignment-company-website-analysis}{%
\section*{Assignment: Company Website Analysis}\label{assignment-company-website-analysis}}
\addcontentsline{toc}{section}{Assignment: Company Website Analysis}

\begin{assessment}
Investigate the Human Resources or Faculty Development portion of a
company's website, a higher education institution or adult learning
facility, preferably one with which you are familiar. Focus on the
faculty or employee development part of the website. In this assignment,
you will apply the theory of teaching in/for/with depth by analyzing the
learning culture of an organization.

In a 4-5 page APA formatted paper, analyze the website by responding to
the following questions in your report:

\begin{enumerate}
\def\labelenumi{\arabic{enumi}.}
\tightlist
\item
  What can you infer about the company's learning culture?\\
\item
  From what is visible on the public website, would you say it is an
  authentic learning community? Why or why not? Discuss whether the
  website reflects aspects of one or more of the learning community
  models explored in previous lessons.\\
\item
  Do you see evidence that interconnectedness and integrity are valued?
  Explain.\\
\item
  What traits and skills seem to be valued in employees?\\
\item
  How does the company develop skills in its employees (e.g., workshops,
  seminars, mentoring)? Are the methods based on the principles of
  andragogy? (see Smith YouTube video). What specific adult learning
  strategies do you see reflected in the development/training
  opportunities for employees?
\end{enumerate}

Your paper should be 4-5 pages and should incorporate references to at
least five scholarly sources you have studied in this course, or other
scholarly sources you have identified.

The paper should include:

\begin{enumerate}
\def\labelenumi{\arabic{enumi}.}
\tightlist
\item
  Introduction\\
\item
  Analysis (responding to the prompts)\\
\item
  Conclusion\\
\item
  Reference List
\end{enumerate}
\end{assessment}

\hypertarget{company-website-analysis-rubric}{%
\subsection*{Company Website Analysis Rubric}\label{company-website-analysis-rubric}}
\addcontentsline{toc}{subsection}{Company Website Analysis Rubric}

See the following rubric that explains how your assignment will be evaluated. Also available as a \href{assets/assessment/Company-Website-Analysis-RUBRIC.pdf}{pdf}

\hypertarget{apa-formatting}{%
\subsubsection*{APA Formatting}\label{apa-formatting}}
\addcontentsline{toc}{subsubsection}{APA Formatting}

\textbf{Unsatisfactory:} Paper does not model language and conventions used in scholarly literature.
Writing is not well-organized. Several errors in grammar or composition. Sources
are not cited. APA citations are not appropriately formatted.

\textbf{Developing:} Paper partially models language and conventions used in scholarly literature.
Writing is somewhat well organized and includes some errors in grammar or
composition. Not all sources cited. APA citations are generally formatted
correctly, with several errors.

\textbf{Proficient:} \emph{Paper consistently models language and conventions used in scholarly
literature. Writing is well-organized and includes few (if any) errors in
grammar or composition. All resources are appropriately cited (including in-text
citations and bibliography information). Few (if any) errors in APA citations.}

\textbf{Exemplary:} Paper is an exemplar of language and conventions used in scholarly literature.
Writing is well-organized and free of errors in grammar or composition. All
resources are appropriately cited. No errors in APA format.

\hypertarget{developing-a-cohesive-and-logical-academic-argument-1}{%
\subsubsection*{DEVELOPING a COHESIVE and LOGICAL ACADEMIC ARGUMENT}\label{developing-a-cohesive-and-logical-academic-argument-1}}
\addcontentsline{toc}{subsubsection}{DEVELOPING a COHESIVE and LOGICAL ACADEMIC ARGUMENT}

\textbf{Unsatisfactory:} Does not make a focused, cohesive, or logical academic argument. Paper is
confusing, and is missing an introduction, body, or conclusion. Transitions
between sections and ideas are missing.

\textbf{Developing:} Makes an academic argument that is only partially focused, cohesive and logical.
Paper is generally organized, but is missing an introduction, body, or
conclusion. Transitions between sections and ideas are unclear.

\textbf{Proficient:} \emph{Makes a focused, cohesive, logical academic argument. Paper is effectively
organized and includes an introduction, body, and conclusion. Transitions
between sections and ideas are clear.}

\textbf{Exemplary:} Makes a focused, cohesive, logical and compelling academic argument. Paper is
effectively organized and includes an introduction, body, and conclusion.
Transitions between sections and ideas are clear and build on each other.

\hypertarget{analysis-of-learning-culture}{%
\subsubsection*{ANALYSIS of LEARNING CULTURE}\label{analysis-of-learning-culture}}
\addcontentsline{toc}{subsubsection}{ANALYSIS of LEARNING CULTURE}

\textbf{Unsatisfactory:} Does not include an analysis of the company learning culture, and no evaluation
of the authenticity of the learning community.

\textbf{Developing:} Includes a partial analysis of the company learning culture, including a limited
evaluation of the authenticity of the learning community.

\textbf{Proficient:} \emph{Includes a detailed analysis of the company learning culture, including an
evaluation of the authenticity of the learning community.}

\textbf{Exemplary:} Includes a detailed analysis of the company learning culture, including an
evaluation of the authenticity of the learning community. Includes a thoughtful
analysis, integrating scholarly literature to support analysis and furthering
scholarly thinking related to teacher identity.

\hypertarget{evaluation-of-interconnectedness-and-integrity}{%
\subsubsection*{EVALUATION of INTERCONNECTEDNESS and INTEGRITY}\label{evaluation-of-interconnectedness-and-integrity}}
\addcontentsline{toc}{subsubsection}{EVALUATION of INTERCONNECTEDNESS and INTEGRITY}

\textbf{Unsatisfactory:} Does not include an evaluation of evidence of interconnectedness and integrity
on the company website. Does not integrate scholarly sources in the evaluation.

\textbf{Developing:} Includes a partial evaluation of evidence of interconnectedness and integrity on
the company website. Evaluation includes only limited reference to scholarly
sources.

\textbf{Proficient:} \emph{Includes a detailed evaluation of evidence of interconnectedness and integrity
on the company website. Evaluation integrates scholarly sources.}

\textbf{Exemplary:} Includes a detailed evaluation of evidence of interconnectedness and integrity
on the company website. Includes recommendations for ways in which to integrate
interconnectedness and integrity into employee development.

\hypertarget{analysis-of-adult-learning-strategies}{%
\subsubsection*{ANALYSIS of ADULT LEARNING STRATEGIES}\label{analysis-of-adult-learning-strategies}}
\addcontentsline{toc}{subsubsection}{ANALYSIS of ADULT LEARNING STRATEGIES}

\textbf{Unsatisfactory:} Does not include a detailed analysis of valued skills and evidence of adult
learning theory in employee development. Does not integrate scholarly sources.

\textbf{Developing:} Includes a partial analysis of valued skills and evidence of adult learning
theory in employee development. Analysis integrates few, if any, scholarly
sources.

\textbf{Proficient:} \emph{Includes a detailed analysis of valued skills and evidence of adult learning
theory in employee development. Analysis integrates scholarly sources.}

\textbf{Exemplary:} Includes a detailed analysis of valued skills and evidence of adult learning
theory in employee development. Includes recommendations for ways in which to
integrate adult learning theory into employee development.

\hypertarget{scholarly-integration-1}{%
\subsubsection*{SCHOLARLY INTEGRATION}\label{scholarly-integration-1}}
\addcontentsline{toc}{subsubsection}{SCHOLARLY INTEGRATION}

\textbf{Unsatisfactory:} Does not integrate scholarly references to support claims and assertions made in
the paper.

\textbf{Developing:} Integrates scholarly references to support some of the claims and assertions
made in the paper.

\textbf{Proficient:} \emph{Integrates scholarly references to support claims and assertions made in the
paper.}

\textbf{Exemplary:} Integrates scholarly references to support claims and assertions made in the
paper, effectively synthesizing different perspectives and research results from
scholarly sources.

\begin{longtable}[]{@{}
  >{\raggedright\arraybackslash}p{(\columnwidth - 8\tabcolsep) * \real{0.2000}}
  >{\raggedright\arraybackslash}p{(\columnwidth - 8\tabcolsep) * \real{0.2000}}
  >{\raggedright\arraybackslash}p{(\columnwidth - 8\tabcolsep) * \real{0.2000}}
  >{\raggedright\arraybackslash}p{(\columnwidth - 8\tabcolsep) * \real{0.2000}}
  >{\raggedright\arraybackslash}p{(\columnwidth - 8\tabcolsep) * \real{0.2000}}@{}}
\toprule\noalign{}
\begin{minipage}[b]{\linewidth}\raggedright
\textbf{TOTAL}
\end{minipage} & \begin{minipage}[b]{\linewidth}\raggedright
\textbf{0 = 0\% (F)}
\end{minipage} & \begin{minipage}[b]{\linewidth}\raggedright
\textbf{10 = 50\% (C)}
\end{minipage} & \begin{minipage}[b]{\linewidth}\raggedright
\textbf{15 = 75 (B)}
\end{minipage} & \begin{minipage}[b]{\linewidth}\raggedright
\textbf{20 = 100\% (A+)}
\end{minipage} \\
\midrule\noalign{}
\endhead
\bottomrule\noalign{}
\endlastfoot
\end{longtable}

\begin{center}\rule{0.5\linewidth}{0.5pt}\end{center}

\hypertarget{assignment-platform-paper}{%
\section*{Assignment: Platform Paper}\label{assignment-platform-paper}}
\addcontentsline{toc}{section}{Assignment: Platform Paper}

\begin{assessment}
For this assignment, you will write a contextualized Platform Paper in
which you discuss your ideal learning community and your role as
teacher/leader of that learning community. Select a context for your
paper (i.e.~facilitating in a FAR Centre in a specific country, teaching
adult learners, facilitating employee development workshops, etc.). Your
paper should be written and referenced in APA format and include
references to a minimum of 10 scholarly sources (this can include
literature you read in this course). You will write a draft of the
Platform Paper in Unit 8 and post for Peer Review. In Unit 9, you will
provide feedback to another learner on their paper. You will make
revisions based on the Peer Review and, in Unit 10, you will submit the
final Platform Paper. Peer reviewers will be assigned in advance.

{Paper Outline}

This paper will be 12-15 pages long, and should include:

\begin{enumerate}
\def\labelenumi{\arabic{enumi}.}
\tightlist
\item
  Introduction (1-2 pages)\\
\item
  Section 1: Ideal Learning Environment (5-7 pages)\\
\item
  Section 2: Your Role as Teacher and Leader (5-7 pages)\\
\item
  Conclusion (1-2 pages)
\end{enumerate}

{Paper Guidelines}

\begin{itemize}
\tightlist
\item
  \textbf{Introduction}: Introduce the two sections in your paper,
  providing a brief description of the key points you will make in each
  section.\\
\item
  \textbf{Section 1}: In section one, you will describe your ideal
  education learning environment. This section should demonstrate your
  learning about authentic learning communities, incorporating scholarly
  sources and your own analysis to depict your ideal learning
  environment. Incorporate a discussion of the learning community
  environment, learning experiences, student learning outcomes, and
  personal beliefs about teaching and learning.\\
\item
  \textbf{Section 2}: In this section, describe your role as a teacher
  or leader within an authentic learning community. Incorporating
  scholarly literature, analyze your role as a facilitator/leader in
  planning learning experiences, facilitating student learning, and
  assessing student learning. Describe the actions, practices, and
  strategies you will engage in to achieve your vision of the learning
  community you described in section one.\\
\item
  \textbf{Conclusion}: Summarize the key points you made in each
  section.\\
\item
  \textbf{References}: Include a reference list with references to at
  least 10 scholarly sources.
\end{itemize}
\end{assessment}

\hypertarget{platform-paper-rubric}{%
\subsection*{Platform Paper Rubric}\label{platform-paper-rubric}}
\addcontentsline{toc}{subsection}{Platform Paper Rubric}

See the following rubric that explains how your assignment will be evaluated. Also available as a \href{assets/assessment/Platform-Paper-RUBRIC.pdf}{pdf}

\hypertarget{apawriting}{%
\subsubsection*{APA/WRITING}\label{apawriting}}
\addcontentsline{toc}{subsubsection}{APA/WRITING}

\textbf{Unsatisfactory:} Paper does not model language and conventions used in scholarly literature. Writing is not well-organized. Several errors in grammar or composition. Sources are not cited. APA citations are not appropriately formatted.

\textbf{Developing:} Paper partially models language and conventions used in scholarly literature. Writing is somewhat well organized and includes some errors in grammar or composition. Not all sources cited. APA citations are generally formatted correctly, with several errors.

\textbf{Proficient:} \emph{Paper consistently models language and conventions used in scholarly literature. Writing is well-organized and includes few (if any) errors in grammar or composition. All resources are appropriately cited (including in-text citations and bibliography information). Few (if any) errors in APA citations.}

\textbf{Exemplary:} Paper is an exemplar of language and conventions used in scholarly literature. Writing is well-organized and free of errors in grammar or composition. All resources are appropriately cited. No errors in APA format.

\hypertarget{developing-a-cohesive-and-logical-academic-argument-2}{%
\subsubsection*{DEVELOPING a COHESIVE and LOGICAL ACADEMIC ARGUMENT}\label{developing-a-cohesive-and-logical-academic-argument-2}}
\addcontentsline{toc}{subsubsection}{DEVELOPING a COHESIVE and LOGICAL ACADEMIC ARGUMENT}

\textbf{Unsatisfactory:} Does not make a focused, cohesive, or logical academic argument. Paper is confusing, and is missing an introduction, body, or conclusion. Transitions between sections and ideas are missing.

\textbf{Developing:} Makes an academic argument that is only partially focused, cohesive and logical. Paper is generally organized, but is missing an introduction, body, or conclusion. Transitions between sections and ideas are unclear.

\textbf{Proficient:} \emph{Makes a focused, cohesive, logical academic argument. Paper is effectively organized and includes an introduction, body, and conclusion. Transitions between sections and ideas are clear.}

\textbf{Exemplary:} Makes a focused, cohesive, logical and compelling academic argument. Paper is effectively organized and includes an introduction, body, and conclusion. Transitions between sections and ideas are clear, and build on each other.

\hypertarget{ideal-learning-environment}{%
\subsubsection*{IDEAL LEARNING ENVIRONMENT}\label{ideal-learning-environment}}
\addcontentsline{toc}{subsubsection}{IDEAL LEARNING ENVIRONMENT}

\textbf{Unsatisfactory:} Does not include a description of your ideal learning environment. Does not reference scholarly sources. Does note analyze key elements of an authentic learning community. Does not mention or describe the learning community environment, student learning outcomes, learning outcomes and personal beliefs about teaching and learning.

\textbf{Developing:} Includes a partial description of your ideal learning environment, referencing few scholarly sources and including a partial analysis of key elements of an authentic learning community. Mentions some elements, but does not fully describe the learning community environment, student learning outcomes, learning outcomes and personal beliefs about teaching and learning.

\textbf{Proficient:} \emph{Includes a detailed description of your ideal learning environment, referencing scholarly sources and analyzing key elements of an authentic learning community. Describes the learning community environment, student learning outcomes, learning outcomes and personal beliefs about teaching and learning.}

\textbf{Exemplary:} Includes a detailed description of your ideal learning environment, referencing scholarly sources and analyzing key elements of authentic learning communities. Provides a rationale for key elements of the learning community environment, student learning outcomes, learning outcomes and personal beliefs about teaching and learning. Advances scholarly thinking about authentic learning communities.

\hypertarget{your-role-as-teacher-and-leaders}{%
\subsubsection*{YOUR ROLE AS TEACHER AND LEADERS}\label{your-role-as-teacher-and-leaders}}
\addcontentsline{toc}{subsubsection}{YOUR ROLE AS TEACHER AND LEADERS}

\textbf{Unsatisfactory:} Does not include a description of your role as a teacher or leader within an authentic learning community, incorporating scholarly literature. Does not include an analysis of your role as a facilitator/leader in planning learning experiences, facilitating student learning, and assessing student learning. Does not include a description of the actions, practices, and strategies you will engage in to achieve your vision of the learning community you described in section one.

\textbf{Developing:} Includes a partial description of your role as a teacher or leader within an authentic learning community, incorporating scholarly literature. Describes but does not analyze your role as a facilitator/leader in planning learning experiences, facilitating student learning, and assessing student learning. Lists but does not describe the actions, practices, and strategies you will engage in to achieve your vision of the learning community you described in section one.

\textbf{Proficient:} \emph{Includes a detailed description of your role as a teacher or leader within an authentic learning community, incorporating scholarly literature. Includes a detailed analysis of your role as a facilitator/leader in planning learning experiences, facilitating student learning, and assessing student learning. Includes a detailed description of the actions, practices, and strategies you will engage in to achieve your vision of the learning community you described in section one.}

\textbf{Exemplary:} Includes a detailed analysis of your role as a teacher or leader within an authentic learning community, incorporating scholarly literature. Includes a detailed analysis of your role as a facilitator/leader in planning learning experiences, facilitating student learning, and assessing student learning. Includes a detailed description of the actions, practices, and strategies you will engage in to achieve your vision of the learning community you described in section one. Synthesizes scholarly thinking about the role of the teacher/leader.

\hypertarget{scholarly-integration-2}{%
\subsubsection*{SCHOLARLY INTEGRATION}\label{scholarly-integration-2}}
\addcontentsline{toc}{subsubsection}{SCHOLARLY INTEGRATION}

\textbf{Unsatisfactory:} Does not integrate many references to support the arguments made in the paper.

\textbf{Developing:} Integrates fewer than 10 scholarly sources to support arguments made in the paper.

\textbf{Proficient:} \emph{Integrates a minimum of 10 scholarly sources to support arguments made in each section of the paper.}

\textbf{Exemplary:} Integrates a minimum of 10 references to support the arguments made in each section, including several scholarly sources not included in course materials.

\begin{longtable}[]{@{}
  >{\raggedright\arraybackslash}p{(\columnwidth - 8\tabcolsep) * \real{0.2000}}
  >{\raggedright\arraybackslash}p{(\columnwidth - 8\tabcolsep) * \real{0.2000}}
  >{\raggedright\arraybackslash}p{(\columnwidth - 8\tabcolsep) * \real{0.2000}}
  >{\raggedright\arraybackslash}p{(\columnwidth - 8\tabcolsep) * \real{0.2000}}
  >{\raggedright\arraybackslash}p{(\columnwidth - 8\tabcolsep) * \real{0.2000}}@{}}
\toprule\noalign{}
\begin{minipage}[b]{\linewidth}\raggedright
\textbf{TOTAL}
\end{minipage} & \begin{minipage}[b]{\linewidth}\raggedright
\textbf{0 = 0\% (F)}
\end{minipage} & \begin{minipage}[b]{\linewidth}\raggedright
\textbf{10 = 50\% (C)}
\end{minipage} & \begin{minipage}[b]{\linewidth}\raggedright
\textbf{15 = 75 (B)}
\end{minipage} & \begin{minipage}[b]{\linewidth}\raggedright
\textbf{20 = 100\% (A+)}
\end{minipage} \\
\midrule\noalign{}
\endhead
\bottomrule\noalign{}
\endlastfoot
\end{longtable}

\hypertarget{references}{%
\chapter*{References}\label{references}}
\addcontentsline{toc}{chapter}{References}

The following are key references used in this course. \textbf{\emph{Check with your course syllabus for required readings.}}

  \bibliography{book.bib}

\end{document}
